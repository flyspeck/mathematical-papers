% Authors: Julia Gordon and Thomas Hales
% December 2014
% Title: Endoscopic transfer of orbital integrals in large residual characteristic

% XX Issues: 
% E as unramified extension, E as splitting field.
% Harish-Chandra descent in pos. char.
% nonstandard endoscopy.
% group transfer factor
% group transfer factor tamely ram. char trivial on top unipo
% Fix theorem numbering by section.
% Measure on unipotent by exponentiation nilpotent?
% Mock exp. vs no mock.

% Notation:
% Zent = Center
% Norm = Normalizer
% Topu = Topological Unipotent
% Rad(B) = Nilpotent radical of reductive group.

\documentclass[12pt]{amsart}
\usepackage{amssymb,amscd,amsmath,amsthm,color}
\usepackage{calc}
\usepackage{amsrefs}
\usepackage{color}
\usepackage[T1]{fontenc}
\usepackage{mathtools}
\usepackage[normalem]{ulem}
\usepackage{tikz}
\usetikzlibrary{matrix,arrows,decorations.pathmorphing}
\usepackage{setspace}
\usepackage{verbatim}
\usepackage{mathrsfs}
\usepackage{lineno}
%\usepackage[notcite, notref]{showkeys}
%\usepackage[left=2cm,top=2cm,right=2cm,nohead]{geometry}

\textwidth 6.5 in
\oddsidemargin 0 in
\evensidemargin 0 in
\topmargin -.125 in
\textheight 8.75 in

\definecolor{bettergreen}{rgb}{0,.7,0}
%\newcommand\blue[1]{{\color{blue}{#1}}}
%\newcommand\green[1]{{\color{bettergreen}{#1}}}
%\newcommand\red[1]{{\color{red}{#1}}}

%\long\def\comment#1{\marginpar{{\footnotesize\color{red} #1\par}}}
%\long\def\commentimmi#1{\marginpar{{\footnotesize\color{bettergreen} #1\par}}}
%\long\def\change#1{{\color{blue} #1}}
%newcommand\green[1]{{\color{green} #1}}
%\newcommand   [1]{{\color{red}\small #1}}

%\let\immi=\green
%\let\raf=\blue

\newcommand\todo[1]{\ \vspace{5mm}\par \noindent\framebox{\begin{minipage}[c]{0.95 \textwidth} \tt #1\end{minipage}} \vspace{5mm} \par}


%%%%%%%%%%%%%%%%%%%%%%%%%
% Defs from Hales

\makeatletter
\newcommand*{\rom}[1]{\text{\expandafter\@slowromancap\romannumeral #1@}}
\makeatother

\newcommand{\op}[1]{\operatorname{#1}}
\newcommand{\ring}[1]{{\mathbb #1}}
\newcommand{\cal}[1]{\mathcal{#1}}
\newcommand{\locus}[1]{\op{Iva}(#1)}
%\def\rtie{\times}
\def\NF{\op{NF}}
\def\UF{\op{UF}}
\def\VF{{\op{VF}}}
\def\Lie{\op{Lie}}

%\def\cG{{ {G}}}
%\def\cH{{ {H}}}

\def\R{\cal{R}}
\def\Y{\Upsilon}
\def\s{{\mathfrak{f}}}
%\def\glocus{{g_{0\text{-set}}}}

%%%%%%%%%%%%%%%%%%%%%%%%%

%\DeclarePairedDelimiter\ceil{\lceil}{\rceil}
%\DeclarePairedDelimiter\floor{\lfloor}{\rfloor}


%%%%  Definitions  re: typeface  %%%%%%%%%%

\newcommand{\A}{\mathbb{A}}
\newcommand{\Q}{{\mathbb Q}}
\newcommand{\C}{{\mathbb C}}
%\newcommand{\R}{{\mathbb R}}
\newcommand{\Z}{{\mathbb Z}}
\newcommand{\N}{{\mathbb N}}
\newcommand{\LL}{{\mathbb L}}
\newcommand{\TT}{{\mathbb T}}
\newcommand{\CC}{{\mathbb C}}
\newcommand{\ZZ}{{\mathbb Z}}

\newcommand{\cF}{\mathcal{F}}
\newcommand{\cB}{\mathcal{B}}

\newcommand{\fg}{\mathfrak{g}}
\newcommand{\fb}{\mathfrak{b}}
\newcommand{\fn}{\mathfrak{n}}
\newcommand{\fc}{\mathfrak{c}}
\newcommand{\fh}{\mathfrak{h}}
\newcommand{\fF}{\mathfrak{F}}

\newcommand{\reg}{\mathrm{rss}}


%%%%  Motivic  definitions  %%%%%%%


\newcommand\ord{\mathrm{ord}}
\newcommand\ac{\overline{\mathrm{ac}}}
\newcommand\lef{\mathbb L}
\newcommand\cP{{\mathcal P}}
\newcommand\cC{{\mathcal C}}
\newcommand{\Loc}{\mathrm{Loc}}


%%%%%%%%%%%%% Theorem declarations %%%%%%%%%%%%%

\theoremstyle{plain}
\newtheorem{thm}{Theorem}
\newtheorem{theorem}[thm]{Theorem}
\newtheorem{lem}[thm]{Lemma}
\newtheorem{cor}[thm]{Corollary}
%\newtheorem{defn}[thm]{Definition}
%\newtheorem{rem}[thm]{Remark}
\newtheorem{prop}[thm]{Proposition}

\theoremstyle{definition}
\newtheorem{rem}[thm]{Remark}
\newtheorem{defn}[thm]{Definition}
\newtheorem{example}[thm]{Example}

\title{Endoscopic transfer of orbital integrals in large residual characteristic}

\author{Julia Gordon and Thomas Hales}



\begin{document}

\begin{abstract} This article constructs Shalika germs in the context
  of motivic integration, both for ordinary orbital integrals and
  $\kappa$-orbital integrals.  Based on transfer principles in motivic
  integration and on Waldspurger's endoscopic transfer of smooth
  functions in characteristic zero, we deduce the the endoscopic
  transfer of smooth functions in sufficiently large residual
  characteristic.
\end{abstract}



\maketitle
\linenumbers


We dedicate this article to the memory of Jun-Ichi Igusa.  The second
author wishes to acknowledge the deep and lasting influence that
Igusa's research has had on his work, starting with his work as a
graduate student that used Igusa theory to study the Shalika germs of
orbital integrals, and continuing today with themes in motivic
integration that have been inspired by the Igusa zeta function.

\bigskip

This article establishes the endoscopic matching of smooth functions
in sufficiently large residual characteristic.  The results are based
on four fundamental results: Langlands-Shelstad descent for transfer
factors~\cite{LSxf}, Ng\^o's proof of the fundamental
lemma~\cite{NBC}, Waldspurger's proof that the fundamental lemma
implies endoscopic matching of smooth functions in characteristic
zero~\cite{W}, and the Cluckers-Loeser version of motivic
integration~\cite{CL}, including transfer principles for deducing
results for one nonarchimedean field from another nonarchimedean field
with the same residue field~\cite{CLe}.  We use recent extensions of
the transfer principle to transfer linear dependencies from one field
to another \cite{CGH2}.

We note that the term transfer is used in with two separate meanings
in this paper.  {\it Endoscopic matching}\footnote{It is called
  endoscopic transfer in the literature, title, and abstract, but we
  prefer to call it endoscopic matching in the body of the paper
  because of the other uses of the word transfer. We avoid the awkward
  but apt phrase ``transfer of transfer.''}
refers to the matching of $\kappa$-orbital integrals on a reductive
group with stable orbital integrals on an endoscopic group, in a form
made precise by the Langlands-Shelstad transfer factor.  On the other
hand, {\it transfer principles} refer to the transfer of first order
statements or properties of constructible functions from one
nonarchimedean field to another nonarchimedean field with the same
residue characteristic.  In this article, we will refer to endoscopic
matching,
transfer factors, and transfer principles.

In our main results, the constraints on the size of the residual
characteristic are not effective.  This means that our results are not
known to apply to any particular local field of positive
characteristic. This seems to be a serious limitation of our methods.
Nonetheless, we hope that our results about the constructibility of
Shalika germs can serve as a further illustration of the close
connection between harmonic analysis of $p$-adic groups and motivic
integration.

\section{Statement of Results}


This introductory section surveys the main results of this article.
Unexplained terminology and notation are explained in the main body of
the paper.  Our first theorem establishes the existence of motivic
constructible functions that represent Shalika germs.

We use the concepts of {\it definable subassignments} and {\it
  constructible motivic functions} (or constructible function for
short) from \cite{CL}.  All constructible functions in this paper are
understood in this sense.  

In this article, by nonarchimidean field we mean a non-discrete
locally compact nonarchimedean valued field; that is, a field
isomorphic to a finite extension of $\ring{Q}_p$ or $\ring{F}_p((t))$
for some prime $p$.  Let $\op{Loc}_m$ denote the set of all
nonarchimedean fields whose residue field characteristic at least
$m\in \ring{N}=\{0,1,2,\ldots\}$.  To avoid set-theoretic issues, we
assume that (the carrier of) each field is a subset of some fixed
cardinal number (that is sufficiently large to obtain every such field
up to isomorphism).  If $S$ is a definable set, and $F\in \op{Loc}_m$,
we write $S(F)$ for the interpretation of $S$ in $F$.  If $f$ is a
constructible function on $S$, we write $f_F$ for the corresponding
function $f_F:S(F)\to \ring{C}$.

Let $G_{/Z}$ be a definable reductive group over a cocycle space $Z$
in the sense of \cite{CGH}.  For each $k\in\ring{N}$, there is a
definable set $\NF^k_G$ representing $k$-tuples of Barbasch-Moy
pairs $\Y=(N,\s)$, in the sense of Section~\ref{sec:nilpotent}, with
pairwise non-conjugate nilpotent elements $N$ in the Lie algebra $\fg$
of $G$.  We give a motivic constructible function $1_\Y$
parametrized by $\Y$.

For $F\in \op{Loc}_m$ and $z\in Z(F)$, we have $F$-points $G_z(F)$ of
a connected reductive group over $F$, obtained by twisting the split
group by the cocycle $z$. We also have $F$-points $\fg_z(F)$ of a Lie
algebra. There is a definable set $\fg^\reg$ of regular semisimple
elements in $\fg$.




\begin{theorem} [Lie algebra Shalika germs]\label{thm:lie-shalika} For each $k\in \ring{N}$,
  there exists a $k$-tuple $\Gamma$ of motivic constructible functions
  with domain $\fg^\reg\times_Z \NF^k_G$ and a constructible
  function $d_k$ with domain $\NF^k_G$.  There exists $m\in
  \ring{N}$, such that for every field $F\in \op{Loc}_{m}$, and for
  all $z\in Z(F)$, there exists $\Y\in \NF^{k_z}_{G_z}(F)$ such that
  $d_{k_z,F}(\Y)\ne 0$, where $k = k_z$ is the number of nilpotent
  conjugacy classes in $\fg_z(F)$.  Moreover, for every $z\in Z(F)$
  and every $\Y\in \NF^{k_z}_{G_z}(F)$ such that $d_{k_z,F}(\Y)\ne 0$,
  the function $X\mapsto \Gamma(X,\Y)_{i,F}/d_{k_z,F}(\Y)$ is the
  Shalika germ (for some normalization of measures) at $X\in
  \fg^\reg_z(F)$ on the orbit $N_i$, the first component of $\Y_i=(N_i,\s)$, for $i=1,\ldots,k_z$.
\end{theorem}

By the definition of $\NF^k_{G}$ and the choice $k=k_z$, the orbits of
$N_i$, the first components of
$\Y_i=(N_i,\s_i)$, for $i=1,\ldots,k_z$, give a complete non-redundant
enumeration of nilpotent classes of $G_z$.  The theorem realizes all
Shalika germs as constructible functions.  In a sense that can be made
precise (the residual characteristic must be large with respect to
fixed choices of the field-independent data such as the root data),
every large residual characteristic reductive group can be represented
as fiber $G_z$ in a definable reductive group $G_{/Z}$, so that the
theorem is general in large residual characteristic.

We obtain a similar representation as constructible functions for
Shalika germ expansions in the group in terms of unipotent conjugacy
classes.  For each $k$, there is a definable set $\UF^k_G$
representing $k$-tuples of pairwise non-conjugate Barbasch-Moy
unipotent pairs in $G$.  We give a motivic constructible function
$1_\Y$ parametrized by $\Y\in \UF^1_G$.

\begin{theorem} If we replace $\NF^k_G$ (resp. $\NF^{k_z}_{G_z}(F)$,
  $\fg^\reg$) with $\UF^k_{G}$ (resp. $\UF^{k_z}_{G_z}(F)$, $G^\reg$)
  in Theorem~\ref{thm:lie-shalika}, the same statement holds in the
  group.
\end{theorem}

Our next theorem gives a representation of the Langlands-Shelstad
transfer factor as a constructible function.  This result was
previously known for unramified groups \cite{CHL}.

\begin{theorem}[Constructibility of Lie algebra transfer
  factors]\label{thm:xfer-factor} 
Let
  $(G,H)_{/Z}$ be a definable reductive group and associated
  endoscopic group over a cocycle space $Z$.  There exists a
  constructible function $\Delta$ on
\begin{equation}\label{eqn:delta-domain}
V := \fh^{G-\reg}\times_Z
  \fg^\reg\times_Z
\fh^{G-\reg}\times_Z
  \fg^\reg
\end{equation}
and a natural number $m\in \ring{N}$ such that for every field $F\in
\op{Loc}_{m}$, every $z\in Z(F)$, and for every 
\[
(X_H,X_G,\bar X_H,\bar X_G)\in
V_z(F),\]  
the Lie algebra Langlands-Shelstad transfer factor is
$\Delta_F(X_H,X_G,\bar X_H,\bar X_G)$.
\end{theorem}

We are unable to obtain the Langlands-Shelstad
transfer factor on the full group, because there is currently no good
theory of multiplicative characters for motivic integration.

\begin{theorem}[Constructibility of group transfer factors] Let
  $(G,H)_{/Z}$ be a definable reductive group and associated
  endoscopic group over a coycle space $Z$.  We write $\op{Topu}(S)$
  for the set of topologically unipotent elements in a definable
  subset $S$ of a reductive group.  There exists a constructible
  function $\Delta$ on
\[
V := \op{Topu}(H^{G-\reg})\times_Z
  \op{Topu}(G^\reg)\times_Z
\op{Topu}(H^{G-\reg})\times_Z
  \op{Topu}(G^\reg)
\] 
and a natural number $m\in \ring{N}$ such that for every field $F\in
\op{Loc}_{m}$, every $z\in Z(F)$, and for every 
\[
(\gamma_H,\gamma_G,\bar \gamma_H,\bar \gamma_G)\in V_z(F),\] the
Langlands-Shelstad transfer factor on the group $G_z(F)$ restricted to
$V_z(F)$ is $\Delta_F(\gamma_H,\gamma_G,\bar \gamma_H,\bar
\gamma_G)$.
\end{theorem}

Based on the the constructibility of Shalika germs and the transfer
factor, we obtain a constructible function representing the Shalika
germs of $\kappa$-orbital integrals (Equation~\ref{eqn:kappa}).  As a
special case, we obtain the constructibility of stable Shalika germs
(Equation~\ref{eqn:stable}).  These constructions work both in the Lie
algebra and in the group.

\begin{theorem}[local endoscopic matching of orbital integrals]\label{thm:local}
  Let $(G,H)_{/Z}$ be a definable reductive group $G$ with definable
  endoscopic group $H$ over a cocycle space $Z$.  There exists a
  natural number $m\in \ring{N}$ such that for all $F\in
  \op{Loc}_{m}$, and all $z\in Z(F)$, the local endoscopic matching of
  orbital integrals holds for $(G_z,H_z)$.  That is, for all $f\in
  C_c^\infty(G_z(F))$, there exists a $f^H\in C_c^\infty(H_z(F))$ for
  which the $\kappa$-orbital integrals of $f$ (with the usual
  Langlands-Shelstad transfer factor) are equal to the stable orbital
  integrals of $f^H$ near $1$. More precisely,
\[
SO(\gamma_H,f^H) = \sum_{\gamma_G} \Delta_F(\gamma_H,\gamma_G,\bar
\gamma_H,\bar \gamma_G) O(\gamma_G,f),
\]
for all $\gamma_H\in H^{G-\reg}_z(F)$ near $1$.
\end{theorem}

Our wording in Theorem~\ref{thm:local} is based on definition of {\it
  local $\Delta$-matching\footnote{It is called $\Delta$-transfer in
    \cite{LSd}.} at the identity} in \cite{LSd}.  It can be
expressed equivalently in terms of Shalika germs.  The wording in
Theorem~\ref{thm:xfer} is based on the definition of {\it
  $\Delta$-matching} in \cite{LSd}.  Recall that $\Delta$-matching
(and the definition of the Langlands-Shelstad transfer factor away
from $1$) requires a choice of central extension $\tilde H$ of the
endoscopic group $H$ and a character $\lambda$ on the center of
$\tilde H(F)$ \cite[Sec.4.4]{LSxf}.

\begin{theorem}[endoscopic matching of orbital integrals]\label{thm:xfer}
  Let $(G,H)_{/Z}$ be a definable reductive group $G$ with definable
  endoscopic group $H$ over a cocycle space $Z$.  There exists a
  natural number $m\in \ring{N}$ such that for all $F\in
  \op{Loc}_{m}$, and all $z\in Z(F)$, the endoscopic matching of
  orbital integrals holds for $(G_z,H_z)$.  That is, for all $f\in
  C_c^\infty(G_z(F))$, there exists a $f^{\tilde H}\in
  C_c^\infty(\tilde H_z(F),\lambda)$ for which the $\kappa$-orbital
  integrals of $f$ (with the usual Langlands-Shelstad transfer factor)
  are equal to the stable orbital integrals of $f^{\tilde H}$. More
  precisely,
\[
SO(\gamma_{\tilde H},f^{\tilde H}) = \sum_{\gamma_G}
\Delta_F(\gamma_{\tilde H},\gamma_G,\bar
\gamma_{\tilde H},\bar \gamma_G) O(\gamma_G,f),
\]
for all $\gamma_{\tilde H}\in {\tilde H}^{G-\reg}_z(F)$.
\end{theorem}

In a sense that can be made precise (the residual characteristic must
be large with respect to fixed choices of the field-independent data
such as the root data), every large residual characteristic reductive
group with endoscopic group can be represented as a fiber $(G_z,H_z)$
of a definable pair $(G,H)_{/Z}$, so that the theorem is comprehensive in
large residual characteristic.

We prove other theorems that we do not state in this introduction that
may be of interest.  We give a transfer principle for asymptotic
expansions and prove a uniformity result for the asymptotics as the
nonarchimedean field varies.  We give a classification of definable
reductive groups in terms of fixed data.  This classification departs in
noteworthy ways from the classification of reductive groups
over nonarchimedean fields.  This leads to the conclusion, for
example, that nonisomorphic unitary groups of the same rank over a
nonarchimdean field can have Shalika germs given by the same formula.
A final section lists some open problems.

\section{Review of Motivic Integration in Representation Theory}

In this section we review basic constructs of motivic integration,
applied to representation
theory and harmonic analysis, as developed in \cite{CHL} and
\cite{CGH}.  We assume familiarity with definable subassignments and
constructible motivic functions, as developed in \cite{CL}.

We work in the Denef-Pas language, a three-sorted first-order
language.  The three sorts are called the valued field sort, the
residue field sort, and the value group sort.  The valued field sort
and residue field sort both contain the first order language of rings,
and the value group sort contains the first order language of ordered
groups.  There is function symbol $\op{ord}$ from the valued field
sort to the value group sort, and function symbol $\ac$ from the
valued field sort to the residue field sort, which are interpreted as
the valuation map and angular component map respectively.

By a {\it fixed choice}, we mean a fixed set that does not depend
on the the Denef-Pas language or its variables in any way.  Fixed
choices are assumed to be made at the outermost level, and
will sometimes be dropped from the notation.  Various formulas
in the Denef-Pas language will be constructed from fixed choices.

A free parameter (or simply parameter) refers to a collection of free
variables of the same sort in a formula in the Denef-Pas language,
ranging over a definable set.  A bound parameter is similar, except
that the variables are all bound by a contiguous block of existential
or a contiguous block of universal quantifiers.  By a parameter in
$\VF$ (for valued field), we mean a variable of valued field sort.


\subsection{fixed choices}\label{sec:fixed}

We let $G^{**}$ be a split connected reductive group over $\ring{Q}$,
with Borel $B^{**}$ and Cartan $T^{**}\subset B^{**} \subset G^{**}$
all over $\ring{Q}$.  We let $(X^*,X_*,\Phi,\Phi^\vee)$ (characters,
cocharacters, roots, and coroots) be the root
datum for $G^{**}$,
all with respect to
$(B^{**},T^{**})$.  

We let $\Sigma$ be a (large) abstract finite group (that plays the
role of a Galois group), with a fixed enumeration
$1=\sigma_1,\ldots,\sigma_n$ of its elements, where $n =
\op{card}{\Sigma}$.
We assume that $\Sigma$ fits into a fixed short exact sequence
\[
1 \to \Sigma^t \to \Sigma \to \Sigma^{unr} \to 1,
\]
where $\Sigma^t$ (called the tame inertia) and $\Sigma^{unr}$ (called
the unramified quotient) are both assumed to be cyclic.  We fix a
generator $\op{qFr}$ of $\Sigma^{unr}$, called the quasi-Frobenius
element.


We choose an action of $\Sigma$ on the root datum stabilizing the set
of simple roots.  Specifically, we fix a homomorphism
\[
\rho_G:\Sigma\to \op{Out}(G^{**},B^{**},T^{**},\{X_\alpha\}),
\]
from $\Sigma$ to the group of automorphisms of $G^{**}$ fixing a
pinning, which we identify with the group of outer automorphisms of
$G^{**}$.  We use the action $\rho_G$ on the root datum to form the
$L$-group
\[
{}^LG = \hat G \rtimes \Sigma,
\]
where $\hat G$ is the complex Langlands dual of $G^{**}$.
We fix $(\hat T,\hat B)$ dual to $(T^{**},B^{**})$.    We assume
that the action of $\Sigma$ on $\hat G$ preserves a pinning
$(\hat T,\hat B,\{\hat X_\alpha\})$ of $\hat G$.  

We choose a semisimple element $\kappa\in \hat T^\Sigma$, with
connected centralizer $\hat H = C_{\hat G}(\kappa)^0$.  We let
$(X^*,X_*,\Phi_H,\Phi_H^\vee)$ be the root datum dual to that
of $\hat H$.  We choose as part of the endoscopic data,
a homomorphism $\rho_H:\Sigma \to C_{{}^LG}(\kappa)/\hat T$,
through which we construct
\[
{}^LH = \hat H \rtimes \Sigma.
\]
We assume that $\Sigma$ acts on $\hat H$ through automorphisms that
preserve some pinning of $\hat H$.  This choice induces an action of
$\Sigma$ on the root datum (giving two actions, $\rho_G$ and $\rho_H$,
of $\Sigma$ on $X^*$: one from $G$ and one from the endoscopic data).
Temporarily disregarding $\Sigma$, we let $H^{**}$ be the split group
over $\ring{Q}$ with root datum $(X^*,X_*,\Phi_H,\Phi_H^\vee)$.  We
fix a linear algebraic subgroup $A(G^{**})$ with finitely many
connected components of $\op{Aut}(G^{**})$ that contains the full
group of inner automorphisms and the image of $\Sigma$.

To treat the theory of reductive groups in a definable context, we fix
faithful rational representations of all reductive groups (over
$\ring{Q}$) of interest: $G^{**}$, $G^{**}_{sc}$, $G^{**}_{der}$,
$G^{**}_{\op{adj}}$, $A(G^{**})$, and even $(G^{**}_{sc}\times
G^{**}_{sc})/\op{Zent}_{sc}$, with diagonal embedding of the center
$\op{Zent}_{sc}$ of the simply connected cover of the derived group of
$G^{**}$.  We do the same for $H^{**}$ and its affiliated groups.  We
fix the obvious morphisms ($G^{**}\to G^{**}_{\op{adj}}$, and so forth)
between affiliated groups by explicit polynomial maps (expressed as
collections of polynomials with rational coefficients).  The rational
representations of the reductive groups $G^{**}$ and so forth give
representations of all of the associated Lie algebras $\fg^{**}$, and
so forth.

The root datum and pinning for $G^{**}$ give a Dynkin diagram
$\op{Dyn}(\Phi^+(G^{**}))$ whose nodes are the simple roots of
$(G^{**},B^{**},T^{**})$.  The group $\Sigma$ acts on the Dynkin
diagram by permutation of the nodes.  Let $\tau$ be an index running
over the set of connected components of the Dynkin diagram.  For each
connected component $\op{Dyn}_\tau$ of the diagram, we consider the
subgroup $\Sigma_\tau$ stabilizing the component, and the kernel
$\Sigma^0_\tau\subset \Sigma_\tau$ of the action on the component.
We let $\ell_\tau$ be the number of connected components in the
$\Sigma$ orbit of $\tau$.  We let $d_\tau =
[\Sigma_\tau:\Sigma_\tau^0]$ be the index.  We add the prefix
superscript $d_\tau$ to the symbol for the Dynkin diagram in the
usual way: ${}^1A_n$, ${}^2A_n$, and so forth.

Following \cite{Reeder}, \cite{Gross}, for each $\tau$, we form an
affine diagram ${}^e\R_\tau$ attached to the tame inertia group
$\Sigma^t_\tau = \Sigma^t\cap \Sigma_\tau$.  Its set of nodes is
$\{0\}\cup I_\tau$, where $I_\tau$ is the set of orbits of simple roots of
$\op{Dyn}_\tau$ under the action of $\Sigma^t_\tau$.  The element
$\{0\}$ represents an extended node.  The superscript prefix $e$ is
the order $[\Sigma^t_\tau:\Sigma^t_\tau\cap \Sigma_\tau^0]$ of
the group of tame inertial automorphisms of $\op{Dyn}_\tau$.  We
drop superscript $e$ from the notation when $e=1$.  The various
connected affine diagrams ${}^e\R = \cal{A}_n, {}^2\cal{A}_n,
\cal{B}_n, \cal{C}_n,\ldots, \cal{E}_8$ are listed in a table in
\cite{Gross}.

We set $\Sigma^{unr}_\tau = \Sigma_\tau/\Sigma^t_\tau$.
For each $\tau$, we choose an action
\[
\phi_\tau:\Sigma^{unr}_\tau\to \op{Aut}({}^e\R_\tau)
\]
of $\Sigma^{unr}_\tau$ on the affine diagram.  When $e\ne 1$, by
inspection the various affine diagrams we see that the order of the
automorphism group is at most $2$, so that the choice of $\phi_\tau$
amounts to at most a binary choice of whether the action is trivial or
non-trivial.  When $e=1$, the situation is only slightly more
involved.

From each orbit of $\Sigma$ on the set of components of the Dynkin
diagram $\op{Dyn}(\Phi^+(G^{**}))$ we choice a representative
$\tau$.  Let $A = \{\tau\}$ be this set of representatives.  Let
$S$ (depending on all the choices above) be the product
\[
S = \prod_{\tau\in A} \op{node}({}^e\R_\tau)/\phi_\tau,
\]
of the $\phi_\tau$-orbits of nodes in each affine diagram.  Let $\cF
= \cF(G^{**},\Sigma,\rho_G,\phi_\tau,\ldots)$ be the set of all
subsets $X$ of $S$ such that the projection of $X$ to each factor is
non-empty.  We call $\cF$ the parahoric indexing set.

The next subsection will give various parameter spaces that admit
interpretations in nonarchimedean structures.  When the fixed data
becomes associated with a reductive group $G$ over a nonarchimedean
field $F$, we either get nothing (if our data does not satisfy
required compatibility requirements) or we get the actual indexing set
for the parahoric subgroups in that reductive group.  This follows
directly from the explicit description of parahorics in \cite{Gross},
which is the starting point for our definition of $\cF$.  Parameters
$\s\in \cF$ may be identified with barycentric centers of facets in a
standard alcove in a suitable apartment of the Bruhat-Tits building.
Groups and algebras in the Moy-Prasad filtration may be associated
with these points in the building.  By \cite{CGH}, for each parameter
$\s\in \cF$, there are definable sets $G_{\s,0}$, $G_{\s,0+}$,
$\fg_{\s,0}$, $\fg_{\s,0+}$ giving parahoric subgroups, subalgebras
and their pro-unipotent radicals.


\subsection{parameter spaces}

We recall our conventions for handling Galois cohomology and the
theory of reductive groups in the Denef-Pas language.  Further details
about these constructions can be found in \cite{CHL} and \cite{CGH}.
We describe
various parameter spaces (or simply space for short) with variables of
the valued field sort.  


\subsubsection{field extensions}

The parameter space of field extensions\footnote{When we work with
  field extension, we will write $E/\VF$, rather than the more
  cumbersome $VE/\VF$, etc.} $E/\VF$ of a fixed degree $n$ is defined
to be the parameter space of $(a_0,\ldots,a_{n-1})\in \VF^n$ (that is,
$n$ variables of the valued field sort) such that
\[
p_a(t) = t^n + a_{n-1} t^{n-1} + \cdots a_0
\]
is an irreducible polynomial.  Field arithmetic is expressed in terms
of operations on $\VF^n$ by means of the identifications
\[
E = \VF[t]/(p_a(t)) \simeq \VF^n.
\]


The space of automorphisms of $E/\VF$ consists of a linear maps
(brought back to arithmetic on $\op{End}(E) = \op{End}{\VF^n} =
\VF^{n^2}$) respecting the field operations.  The space of Galois
field extensions $E/\VF$ of fixed degree $n$ is definable by the
condition that there exist $n$ distinct automorphisms of the field
$E/\VF$.  There is a space of the enumerated Galois group of Galois
field extensions of fixed degree $n$, given by tuples
\begin{equation}\label{eqn:enum-gal}
(E,\sigma_1,\sigma_2,\ldots,\sigma_n),
\end{equation}
where $E/\VF$ is a Galois field extension and $\sigma_i$ are the
distinct field automorphisms.  For fixed abstract group $\Sigma$ of
order $n$ with full enumeration $\sigma'_i\in \Sigma$, for
$i=1,\ldots,n$, there is a space of Galois groups isomorphic to
$\Sigma$ given by tuples (\ref{eqn:enum-gal}) with the additional
isomorphism requirement that
\[
\sigma_i\sigma_j = \sigma_k \Longleftrightarrow \sigma'_i\sigma'_j =
\sigma'_k.
\]

The space of unramified field extensions $E/\VF$ of a fixed degree $n$
is a definable subspace of the space of all field extensions.  There
is a space of pairs $(E,\op{qFr})$ where $E$ is an unramified extension
and $\op{qFr}$ is a fixed generator of the Galois group of $E/\VF$.  

There is a space of field extensions $K/E/\VF$, for fixed degrees
$E/\VF$ and $K/\VF$.  It is specified by irreducible polynomials
$p$ and $q$ for $E = \VF[t]/(p(t))$ and $K = \VF[s]/(q(s))$, and an
element $t'\in \VF[s]/(q(s))$ (the image of $t$ under $E\to K$) such
that $p(t') = 0$.  There is a space of Galois field extensions
$K/E/\VF$ for fixed degrees for $E/\VF$ and $K/\VF$ with enumerations
of the Galois groups of $K/\VF$ and $K/E$, together with a table
describing the homomorphism $\op{Gal}(K/\VF)$ to $\op{Gal}(E/\VF)$.

If we have a short exact sequence of enumerated groups 
\[
1 \to \Sigma^t \to \Sigma \to \Sigma^{unr}\to 1,
\]
there is a space of Galois field extensions $K/E/\VF$ with enumerated
automorphisms $\sigma_1,\ldots$, such that the automorphisms in
$\Sigma^t$ act trivially on $E$, $K/E$ is totally ramified with Galois
group $\Sigma^t$, $K/\VF$ has Galois group $\Sigma$, and $E/\VF$ is
unramified with Galois group $\Sigma^{unr}$.

\subsubsection{Galois cocycles}

The space of Galois cocycles is given as tuples
\[
(K,\sigma_1,\ldots,\sigma_n,m_1,\ldots,m_n),
\]
(where $m_i \in K^k$ or $m_i\in GL(n',K)$ or some related parameter
space as module).  Here $K$ is a Galois extension of fixed degre $n$, with
enumerated Galois group $\sigma_i$ subject to the cocycle relations:
\[
\sigma_i \sigma_j 
  = \sigma_k \Longrightarrow m_i \sigma_i(m_j) 
  = m_k\quad\text{for all } i,j,k.
\]

For fixed choices of $G^{**}$, action of $\Sigma$ on the root datum,
there is a space $(K,\sigma_1,\ldots,a_1,\ldots,G^*)$ of
quasi-split reductive groups split over $K$ with enumerated
isomorphism between the Galois group of $K/\VF$ and $\Sigma$.  Here
the elements $a_i$ enumerate the cocycle in the outer
automorphism group (identified with automorphisms fixing a pinning
$(B^{**},T^{**},\{X_\alpha\})$) defining the quasi-split form.  The
pair $(B^{**},T^{**})$ gives a pair $(B^*,T^*)$ in the quasi-split
form.

$G$ is a definable family of reductive groups parametrized by a
definable cocycle space $Z$, which is constructed as in \cite{CGH2}.  We
include in the space $Z$ an explicit choice of inner twisting 
\[
\psi:G
\times_\VF K \to G^*\times_\VF K
\] to the quasi-split inner form that agrees with an enumerated
cocycle $\sigma_i(\psi) \psi^{-1}$ with values in $G^*_{\op{adj}}(K)$, also
given as part of the data of $Z$.  In fact, we identify $G$ with $G^*$
over $K$, so that $\psi$ is the identity, but it is retained in
notation in the form of a cocycle $\sigma(\psi)\psi^{-1}$.  

Let $\fc_{\fg^{*}} = \fg/\op{ad}$ be Chevalley's adjoint quotient, which
we identify with the space of characteristic polynomials $\fg\to
\fc_{\fg^{*}}$.  The fiber in $\fg^\reg$ over each element is
$\fc_{\fg^{*}}$ is the definable stable conjugacy class with the given
characteristic polynomial.  Note that $\fc_{\fg^{*}}$ depends only on
the quasi-split form $\fc^*$. Similarly, we have a map $G\to T^*/W$,
where $T^*$ is a maximally split Cartan subgroup of the quasi-split
inner form $G^*$ of $G$, and $W$ is the absolute Weyl group for $T^*$
in $G^*$.

If $F$ is a nonarchimedean field, and $z\in Z(F)$, we have the fiber
$G_z$.  It is a connected reductive linear algebraic group with Lie
algebra $\fg_z$.  The map $\fg_z\to \fc_{\fg^{*},z}$ is the usual map
from the Lie algebra to the space of characteristic polynomials.

\subsubsection{endoscopic data}

We may similarly use the endoscopic data from our fixed choices to form
a space of quasi-split endoscopic data for $H$, again split by $K/\VF$.  
We let $\fg$ and $\fh$ be the Lie algebras of $G$ and $H$.

We
similarly construct a cocycle space for $H$ and take the fiber product
$Z$ of the cocycle spaces for $G$ and $H$, again denoting it by $Z$,
by abuse of notation.  We also include in the parameters of
the cocycle space $Z$ a large unramified extension and a choice
$\op{qFrob}$ of generator.
We also
include in $Z$ a parameter $b$ that trivializes an invariant
differential form of top degree, as explained in Section~\ref{sec:volume}.




\subsection{nilpotent elements}\label{sec:nilpotent}

The properties of nilpotent elements are well-known.  Here we
summarize the properties that we use.

\begin{theorem} 
  Let $G_{/Z}$ be a definable reductive group with Lie algebra
  $\fg_{/Z}$ over a cocycle space $Z$.  There exists $m\in\ring{N}$
  such that for all $F\in\op{Loc}_m$, and all $z\in Z(F)$, the
  following properties are equivalent properties of $N\in \fg_z(F)$:
\begin{enumerate}
\item $\rho(N)$ is nilpotent for {\it some} faithful representation $\rho$
 of  $\fg_z$;
\item $\rho(N)$ is nilpotent for {\it every} faithful representation $\rho$
 of  $\fg_z$;
\item $0$ lies in the Zariski closure of the adjoint orbit of $N$;
\item $0$ lies in the $p$-adic closure of the adjoint orbit of $N$;
\item there exists $\lambda\in X_*^F(G_z)$ such that $\lim_{t\to 0}
  \op{Ad}(\lambda(t))N = 0$; and
\item the image of $N$ under the morphism $\fg_z\mapsto \fc_{\fg^*,z}$
  is $0$.
\end{enumerate}
\end{theorem}

\begin{proof} The equivalence of the first two properties is
  well-known \cite[15.3]{humphries}.  The implications (5) $\Leftrightarrow$
  (4) $\Rightarrow$ (3) are in \cite{debacker:nilp},
  ~\cite[2.5.1]{adler-debacker:bt-lie}.  The implications (2)
  $\Leftrightarrow$ (3) $\Rightarrow$ (5) are in \cite[3.5]{mcninch},
  \cite[4.1,Prop 4]{mcninch}.

To see the equivalence of (6) with say (2), we may work over an
algebraically closed field. In particular, $\fg = \fg^{**}$.  Note that the image of
$N$ is $0$ in $\fc_{\fg}$ if and only if the image of $N$ is $0$ in
$\ft^* = \fb/\fn$ (under conjugation to a Borel), which is true if
and only if the image of $N$ is in $\fn^*$, the nilradical of $\fb$.
This holds if and only if the semisimple part of $N$ is trivial, which
is equivalent to (2).
\end{proof}

The last of the properties enumerated in the theorem is a definable
condition.  Thus, we have a definable set of all nilpotent elements in
$\fg$.  The theorem gives the compatibility of this definition with
notions of nilpotence in the various papers we cite.

\begin{theorem}[Barbasch-Moy]\label{thm:bm} 
  Let $G_{/Z}$ and $\fg_{/Z}$ be as above.  There exists $m$ such that
  for all $F\in \op{Loc}_m$, $z\in Z(F)$, and for every nilpotent
  orbit in $\fg_z(F)$, there exists $\s\in \cF$ and a nilpotent
  element $N$ in the orbit such that
\begin{enumerate}
   \item $N\in \fg_{z,\s,0}$, and
   \item if $N'$ is nilpotent and $N'\in N + \fg_{z,\s,0+}$, then $N$
     lies in the $p$-adic closure of the orbit of $N'$.
\end{enumerate}
\end{theorem}

We say that $\Y=(N,\s)$ is a Barbasch-Moy pair, with $\s\in \cF$, if $N$
is nilpotent and it satisfies the properties of the theorem.  For each
$\s\in \cF(F)$, there is a definable set consisting of all nilpotent
elements $N$ such that $\Y=(N,\s)$ is a
Barbasch-Moy pair.

\begin{cor}\label{thm:nilbound}  Let $G_{/Z}$ and $\fg_{/Z}$ be as above.  
There exists a constant $k$ such that for all $F\in \op{Loc}_m$, $z\in
Z(F)$, the number of nilpotent conjugacy classes in $\fg_z(F)$ is at
most $k$.
\end{cor}

\begin{proof} By the preceding theorem, for given $F$, the number of
  nilpotent classes is at most the sum of the numbers $k_\s$, where
  $k_\s$ is the number of nilpotent conjugacy classes in the finite
  reductive group $\fg_{z,\s,0}/\fg_{z,\s,0+}$.  Field-independent
  bounds on the number of nilpotent elements in a reductive
  group over a finite field are well-known \cite{carter}.
\end{proof}

We have corresponding versions of these theorems for the set of
unipotent elements.  We have a definable subset of $G$ consisting of
unipotent elements $u$, determined by the condition that the image of
$u$ under $G\to T^*/W$ is $1$.

\begin{theorem} Let $G_{/Z}$ be as above.  There exists
  $m$ such that for all $F\in \op{Loc}_m$, $z\in Z(F)$, and for all unipotent
  conjugacy classes in $G_z(F)$, there exists $\s\in \cF$ and an element
  $u$ in the unipotent conjugacy class such that
\begin{enumerate}
   \item $u\in G_{z,\s,0}$, and
   \item if $u'$ is unipotent and $u'\in u G_{z,\s,0+}$, then $u$
     lies in the $p$-adic closure of the conjugacy class of $u'$.
\end{enumerate}
\end{theorem}

\begin{proof} This is a first order statement in the Denef-Pas
  language.  Therefore, by a transfer principle, it is enough to prove
  the result when $F$ has characteristic zero.  We work over $F$, with
  $z\in Z(F)$ fixed, and write $\fg_{\s,r}$, $\fg_{\s,r+}$, $G_{\s,r}$,
  $G_{\s,r+}$ for the usual Moy-Prasad filtrations with
  $r\in\ring{R}$.  In particular, $\fg_{\s,0}=\fg_{z,\s,0}(F)$, and so
  forth.

  When $F$ is characteristic zero, and $m$ is large enough, we have an
  exponential map defined on the set of all topologically nilpotent
  elements with the following properties.\footnote{In characteristic
    zero, we have the mock exponential map ${\mathbf e}$, such that,
    in particular, ${\mathbf e}(\fg(F)_{x,0+})=G(F)_{x, 0+}$ for all
    $x$ (see Hypothesis 3.2.1 in \cite{debacker:homogeneity}, which is
    known to be true in characteristic zero -- need to find a better
    reference, probably a paper by Adler; set $r=0+$ in that
    hypothesis.) }
  \begin{enumerate}
    \item $\exp(\fg_{\s,r}) = G_{\s,r}$, for all $r>0$.
    \item $\exp(\fg_{\s,0+}) = G_{\s,0+}$.
    \item $\exp$ restricts to a bijection between the set of
      nilpotent elements in $\fg_{\s,0}$ and the set of unipotent
      elements in $G_{\s,0}$.
   \item Let $N$ be a nilpotent element in $\fg_{\s,0}$.  Then the
     image of $\exp(N)$ in $G_{\s,0}/G_{\s,0+}$ is $\op{fexp}(\bar N)$,
     where $\bar N$ is the image of $N$ in $\fg_{\s,0}/\fg_{\s,0+}$ and
     $\op{fexp}$ is the finite field exponential from the nilpotent
     set of $\fg_{\s,0}/\fg_{\s,0+}$ to the unipotent set of
     $G_{\s,0}/G_{\s,0+}$. The map $\op{fexp}$ is injective on the
     nilpotent set.
   \item The exponential map from the nilpotent set to the unipotent
     set preserves orbits and the partial order given by orbit closure.
 \end{enumerate}

  Pick $u$ in the given conjugacy class in such a way that $u =
  \exp(N)$, and $N$ has the properties in Theorem~\ref{thm:bm}, for
  some $\s\in cF$.
  Then $u\in G_{z,\s,0}$.  Let $u' = \exp(N') \in u G_{z,\s,0+}$.
  Reducing modulo
  $G_{\s,0+}$ gives $\op{fexp}(\bar N') = \op{fexp}(\bar N)$. By
  injectivity, we have $\bar N' = \bar N$ and $N' \in N +
  \fg_{\s,0+}$.  By Theorem~\ref{thm:bm}, $N$ lies in the closure of the
  orbit of $N'$.  Exponentiating again, $u = \exp(N)$ lies in the
  closure of the orbit of $u' = \exp(N')$.
\end{proof}


\section{Volume forms}\cite{sec:volume}

All integrals are to be computed with respect to invariant measures on
their respective orbits.  This means that the appropriate context for
motivic integration is integration with respect to volume forms, as
described in \cite[Sec. 8]{CL}.    Volume forms in the context of
reductive groups are explained in \cite{shin-templier:b}.  We follow
those sources.

We recall that each nilpotent orbit can be endowed with an invariant
motivic measure $d\mu^{\op{nil}}$ that comes from the the canonical
symplectic form on coadjoint orbits~\cite[Prop.~4.3]{CGH}.  This
measure has a free parameter $N$ that runs over the nilpotent cone.

We obtain an invariant motivic measures on the unipotent set in the
group by exponentiating the measure on the nilpotent cone.

We recall that the invariant volume forms on stable regular semisimple
orbits in the Lie algebra is constructed in \cite{CHL}.  We have the
morphism $\fg^\reg\to\fc_{\fg^*}$ that classifies the regular
semisimple stable orbits.  The Leray residue (in the sense of
\cite{CL}) of the canonical volume form\footnote{XX More precisely,
  should we be taking a relative canonical volume form with respect to
  $Z$?} on $\fg$ by the canonical volume form on $\fc_{\fg^*}$ yields
an invariant volume form $d\mu^\reg$ on each fiber of the morphism,
and hence an invariant volume form on stable regular semisimple orbits
with free parameter $X\in\fc_{\fg^*}$.  We compute all stable orbital
integrals on stable regular semisimple orbits with respect to this
family of volume forms.  By restriction to each conjugacy class in the
stable conjugacy class, we obtain an invariant measure on each regular
semisimple conjugacy class.

Let $G_{/Z}$ be a definable reductive group over a cocycle space $Z$.
Let $d$ be the relative dimension of $G$ over $Z$.  Part of the data
for $Z$ is a space of field extensions $K$ splitting $G$.  We may view
$G$ as a subset of $G_K\subset K^n = \VF^{kn}$, for some $n\in\ring{N}$, where
$k=[K:\VF]$.

We construct an algebraic differential $d$-form on $\VF^{rn}$ that
restricts to an invariant form on fibers $G_z$ over $Z$.  We start
with the split case (and $Z$ a singleton set).  If $G^{**}$ is split
over $\op{VF}$, we have a top invariant form on $G^{**}$ given on an
open cell of $G^{**}$ by
\begin{equation}
d^*t\land dn\land dn',
\end{equation}
where $d^*t$ is a top invariant form on a split torus $T^{**} \subset
B^{**}$, $dn$ a top invariant form on the unipotent
radical $\op{Rad}(B)^{**}$ of $B^{**}$, and $dn'$ on the radical of the Borel
subgroup opposite to $B^{**}$ along $T^{**}$.  We may pick root
vectors $X_\alpha$ for each root (as fixed choices over $\ring{Q}$).
Then by placing a total order on the positive roots, we may write 
\[
n = \prod_{\alpha>0}\exp(x_\alpha X_\alpha),
\]
and $dn = \land_{\alpha>0} dx_\alpha$.  Similar considerations hold
for $dn'$.

We may extend the top invariant form $\omega$ to $\omega_K$, from
$\VF$ to $K = \VF[t]/(p_a(t)) = \VF^k$, by taking each coordinate
$x_\alpha$, and so forth to be a monic polynomial of degree $k$ with
coefficients in $\VF$.  This does not change the degree $d$ of the
differential form, but expanding in $t$, it takes values in $\VF^k$.
The form $\omega_K$ is $G_K$ invariant.  It is also invariant by the
action of $G_{\op{adj}}(K)$ on $G_K$ by conjugation.

Now we allow the cocycle space $Z$ to be nontrivial.  We have a
morphism $G_K\to Z$, and $G$ is identified with the fixed point set of
the action of $\op{Gal}(K/\VF)$ on $G_K$.  Let $dz$ be any top form on
$Z$.  Since the action of the Galois group fixes a pinning, we see
that when $G$ is quasi-split, the action of the Galois group preserves
the differential form $dz\land \omega_K$ up to a cocycle with values
in $K^\times$.  Even if the group is not quasi-split, the Galois group
fixes the pinning up to a cocycle $g_\sigma$ with values in
$G_{\op{adj}}(K)$, and considering that the adjoint action of $G_{\op{adj}}$
preserves $dz\land \omega_K$, we see that the differential form is
preserved by the Galois action, up to a cocycle $a_\sigma$ with values
in $K^\times$.
When we take
nonarchimedean structures, we may split this cocycle by Hilbert's
90th.  Working at the definable level, we introduce a new free
parameter $b$ with values in $K^\times$ that is subject to the
relation $a_\sigma = \sigma(b)^{-1} b$.  

Set $\omega_0 = b\, dz\land\omega_K$.  It is Galois stable on
$G$, so on this definable set, the $d+\dim Z$-form takes values in $F$.
Also, since $\omega_K$ is invariant by $G_K$, this implies the
invariance of $\omega_0$ by $G$ (acting fiberwise over $Z$).  
%Then $\omega_0$ is an
%invariant form of (relative) top degree on the fibers $G_z$.

We have a morphism $G^\reg\to T^*/W$ that classifies stable regular
semisimple conjugacy classes in the group.  We form the Leray residue
of the Haar volume form $|\omega_0|$ on $G$ with respect to the
canonical volume form on $T^*/W$.  This yields an invariant volume
form on each fiber of the morphism, and hence an invariant volume form
on stable regular semisimple orbits in the group.



\subsection{Relation to valued fields}

To put the preceding definitions in context, we make a few comments
about reductive groups over nonarchimedean local fields $F$ in large
residual characteristic.  In applications, we assume that the residual
characteristic is sufficiently large that all Galois groups that
appear are tame (that is, wild inertia is trivial).


\subsubsection{multiplicative characters}

We recall that there is currently no good theory of multiplicative
characters for motivic integration.  In particular, since the
Langlands-Shelstand transfer factor makes uses of multiplicative
characters in the form of $\chi$-data, when working motivically,
we restrict our attention to a neighborhood of the identity on the
group.  That is, we restrict to the kernels of the multiplicative
characters.

\subsubsection{Frobenius automorphism}

Although we may speak of unramified extensions and their Galois
groups, when working motivically, we have no way to single out the
canonical Frobenius generator of the Galois group of unramified
extensions.  Indeed, we do not have access the cardinality of the
residue field $q$, until we specialize to a particular local field
$F$.

Instead, in \cite{CHL} and here, we work with an arbitrary generator
$\op{qFr}$ of cyclic group $\Sigma_0$.  In the Tate-Nakayama
isomorphism, the quasi-Frobenius element is used to identify
$\op{Gal}(E/F) = \ring{Z}/n\ring{Z}$, when $E/F$ is unramified of
degree $n$.  If we pick the ``wrong'' (i.e. non-Frobenius) generator
of $\op{Gal}(E/F)$, it has the effect of replacing the endoscopic
datum $s\in \hat T^\Sigma$ with another element $s^i$ with the same
centralizer, for some $i\in (\ring{Z}/n\ring{Z})^\times$.  This again
gives valid endoscopic data, and the arguments still work.


\section{the Langlands-Shelstad transfer factor for Lie algebras}

In this section, we give the Langlands-Shelstad transfer factor for
Lie algebras as stated in Theorem~\ref{thm:xfer-factor}.  Our definition will be
carried out entirely in terms of the Denef-Pas language and
constructible motivic functions, without reference to a nonarchimedean
field.  This construction was previously described in \cite{CHL} in
the special context of unramified groups.

We make a fixed choice of a nonzero sufficiently divisible
$k\in\ring{Z}$.  We will give the definition of the transfer factor in
backwards order starting with the top-level description, and
successively expanding and refining the definitions until all unknown
terms have been specified.  The transfer factor $\Delta(X_H,X_G,\bar
X_H,\bar X_G)$ is the product of two constructible terms:
\[
\Delta_0(X_H,X_G,\bar
X_H,\bar X_G)\quad\text{and}\quad \ring{L}^{d(X_H,X_G)}.
\]
The parameters $X_H$ and $\bar X_H$ run over the definable set of
$G$-regular semisimple elements of the Lie algebra $\fh$.  The
parameters $X_G$ and $\bar X_G$ run over the definable set of regular
semisimple elements of $\fg$; that is, $(X_G,\bar X_G)\in
\fg^\reg\times_Z\fg^\reg$.  Recall that $\psi$ is the inner isomorphism
defined over an extension of $\VF$ between $G$ and $G^*$.  The
transfer factor is defined to be $0$ unless $X_H$ and $X_G$ correspond
under the definable condition requiring the image of $\psi(X_G)$ in
the Chevalley quotient $\fc_{\fg^*}$ to equal the image of $X_H$ in
$\fc_{\fh}$ under $\fc_{\fh}\to\fc_{\fg^*}$, and similarly for $(\bar
X_H,\bar X_G)$.  We now assume that the parameters are restricted to
the definable subsets satisfying these constraints.  The factor
$\ring{L}^{d(X_H,X_G)}$ is the usual discriminant factor (called
$\Delta_{\rom4}$ by Langlands and Shelstad).  It is constructible by
\cite{CHL}.

The parameter $X_G$ is to be considered the primary parameter.  The
transfer factor depends in a subtle way on its conjugacy class within
its stable conjugacy class. The parameters $(\bar X_H,\bar X_G)$
should be viewed as secondary, only affecting the normalization of the
transfer factor by a scalar independent of $X_G$.

The constructible function $\Delta_0$ is a step function on 
\[
(X_H,X_G,\bar X_H,\bar X_G)\in \fh^{G-\reg}\times_Z\fg^\reg\times_Z \fh^{G-\reg}\times_Z\fg^\reg,
\] and it will
be defined as a finite linear combination of indicator functions of
definable sets $D_l$:
\[
\Delta_0 = \sum_{l=0}^k e^{2\pi i l/k}
1_{D_l }.
\]
Here, $D_l$ is the definable set on which $\Delta_0$ takes the value
$e^{2\pi i l/k}$.  Typically, constructible functions have integer
coefficients, but there is no harm in extending scalars to $\ring{C}$
to allow the given roots of unity.

We will give further definable sets $D_{l,{\rom{1}}}$ and $D_{l,\rom{3}}$.  (The
subscripts
correspond to the Roman numerals $\rom1$ and $\rom3_1$ in the
Langlands-Shelstad paper.) In terms
of these sets, we define $D_l$ by the union indexed over
$i,i',j\in\{0,\ldots,k-1 \}$ such that $i + j \equiv i' + l\mod k$
of the definable sets
\[
\{(X_H,X_G,\bar X_H,\bar X_G) \in D_{j,\rom{3}} \mid 
(X_H,X_G)\in D_{i,\rom1},~ (\bar X_H,\bar X_G)\in D_{ i',\rom1}\}.
\]
That is, we should associate roots of unity
$e^{2\pi i j/k}$, $e^{2\pi i i/k}$ (with two meanings of $i$), $e^{-2\pi i i'/k}$ to 
$D_{j,\rom{3}}$, $D_{i,\rom1}$ and $D_{i',\rom1}$ respectively, then multiply
the three roots of unity to obtain the contribution to $D_l$.


We have a space of regular nilpotent elements in the quasi-split Lie
algebra $\fg^*$.  The Kostant section \cite{Kott} to the Chevalley
quotient is determined by a choice of regular nilpotent element.  The
transfer factor will be independent of the choice of regular nilpotent
used to determine the Kostant section, allowing us to treat the
regular nilpotent element as a bound parameter.  The element $X_H\in
\fh$ determines an element $X\in \fg^*$ by the composite of maps
\[
\fh \to \fc_{\fh} \to \fc_{\fg^*} \to \fg^*,
\]
where the last map is the Kostant section.  

We define $a$-data to be the collection of constants $a_\alpha:=
\alpha(X)$, for $\alpha\in \Phi$.  This choice of $a$-data allows us
to do without $\chi$-data (in the sense of Langlands and Shelstad) for
the Lie algebra transfer factor.

\subsection{The cocycle $\lambda(T_{sc})$}

We recall the definition of a Galois cocycle $\lambda(T_{sc})$ with
values in $T_{sc}(K)$, attached to the centralizer $T_{sc}$ of $X$ in
$G^*$.  We take $K/\VF$ to be a splitting field of $G$, and take
$L/\VF$ to be a Galois extension that splits $T_{sc}$.  There are
bound parameters running over $L/\VF$, its enumerated Galois group,
and their relation to $K/\VF$.  We let $h$ be a bound parameter in
$G^*$ such that $(T_{sc},B_{sc}) = \op{Ad}(h)\,(T^*,B^*)$.

We have a triple $(G^*,B^*,T^*)$ with $T^*\subset B^*\subset G^*$.  If
$w$ is any element of the Weyl group of $T^*$ in $G^*$ with reduced
expression $w = s_{\alpha_1}\cdots s_{\alpha_r}$, we set $n(w) =
n(s_{\alpha_1})\cdots n(s_{\alpha_r})$ and $n(s_\alpha)$ is the image
in $\op{Norm}_{G_{sc}^*}(T_{sc}^*)$ of
\[
\begin{pmatrix}0 &1\\ -1 & 0\end{pmatrix} = \op{exp} X_\alpha \op{exp} -
  X_{-\alpha} \op{exp} X_\alpha
\]
under the homomorphism $SL(2)\to G^*_{sc}$ attached to the Lie triple
$\{X_\alpha,X_{-\alpha},H_\alpha\}$ coming from a pinning
$(B^*,T^*,\{X_\alpha\})$).  The transfer factor does not depend on the
pinning, and we treat the pinning as a bound parameter.  The
exponential map used to define $n(s_\alpha)$ is a polynomial function
on nilpotent elements.

The cocycle $\lambda(T_{sc})$ is defined using the $a$-data by an enumerated
coycle of $\op{Gal}(K/\VF)$ as described above, with
\[
\sigma \mapsto h m(\sigma_T) \sigma(h)^{-1},\quad \text{where}\quad
m(\sigma_T):= \left( \prod_{1,\sigma}^p a_\alpha^{\alpha^\vee}\right) n(\omega_T(\sigma)),
\]
and where $x^{\alpha^\vee}$ denotes the image of $x$ under the coroot
$\alpha^\vee$.  An inspection of each element of the formula shows
that this cocycle is definable.  See \cite{LSxf} for the combinatorial
description of the gauge $p$ and the choice $\omega_T$ of Weyl group
elements (coming from the twisted action of the Galois group on
$T^*_{sc}$).

\subsection{Another cocycle}

Next we recall the definition of an enumerated Galois cocycle
$\op{inv}(X_H,X_G,\bar X_H,\bar X_G)$.  Let $T$ and $\bar T$ be the
centralizers in $G$ of $X$ and $\bar X$, where $X$ and $\bar X$ are
constructed as above from a Kostant section.  We take a bound
parameter space for a Galois field extension $L/K/\VF$ that splits $T$
and $\bar T$.  The cocycle takes values in the $L$-points of $U:=
(T\times \bar T)/\op{Zent}_{sc}$, where the center $\op{Zent}_{sc}$ of the simply
connected cover is mapped diagonally.

To construct the cocycle, we take bound parameters $h$ and $\bar h$
 in $G(L)$ such that
\[
\op{Ad} h\,\psi(X_G)  = X,\quad \op{Ad} \bar h\, \psi(\bar X_G) = \bar X,
\]
with $\psi$ the inner twist given above.  As an inner twist, we get an
enumerated cocycle of $\op{Gal}(K/\VF)$ by $\op{Int} u(\sigma) := \psi
\sigma(\psi)^{-1}$.  By our embedding $K\to L$ we obtain an enumerated
coycle of $\op{Gal}(L/\VF)$.  Set $v(\sigma) = h u(\sigma)
\sigma(h)^{-1}$ and $\bar v(\sigma) = \bar h u(\sigma) \sigma(\bar
h)^{-1}$. The cocycle $\op{inv}(X_H,X_G,\bar X_H,\bar X_G)$ is then
\[
\sigma \mapsto (v(\sigma)^{-1},\bar v(\sigma))\in U(L).
\]
All this data is definable.

\subsection{Tate-Nakayama}

We continue with our fixed choice of a sufficiently divisible integer
$k$.  We have an element $\kappa\in \hat T$ that is part of our
endoscopic data.  The article \cite{CHL} shows how to treat this
element as an enumerated cocycle $\kappa_T$ with values in $X^*(T)$
for various Cartan subgroups $T$.  It uses this to give the
Tate-Nakayama pairing in terms of the Denef-Pas language.  (This
requires the cocycle space of $Z$ to include the space of pairs
$(E,\op{qFr})$ of an unramified extension of sufficiently large fixed
degree and generator $\op{qFr}$ of the Galois group $E/\VF$.)  In
particular, with the fixed choice of a highly divisible integer $k$ as
above, for $j\in \ring{Z}$, and for a definable parameter space of
enumerated cocycles $c(\sigma)$ with values in $T(L)$, there is a
definable set
\[
\langle c(\sigma),\kappa_T\rangle_l
\]
of all cocycles such that the Tate-Nakayama pairing of the coycle with
$\kappa_T$ has value $e^{2\pi l/k}$.  We define $D_{l,\rom1}(X_H,X_G)$ to
be
\[
\langle \lambda(T_{sc}),\kappa_{T_{sc}}\rangle_l
\]

Similarly, we may form an element $\kappa_U$ (the image of
$(\kappa_T,\kappa_{\bar T})$ in $X^*(U)$).  We let
\[
D_{l,\rom{3}}(X_H,X_G,\bar X_H,\bar X_G)
\]
 be the definable set
\[
\langle \op{inv}(X_H,X_G,\bar
X_H,\bar X_G),\kappa_{U}\rangle_l
\]
Combined with our earlier definitions, this completes the definition
of the Lie algebra Langlands-Shelstad transfer factor as a
constructible motivic function.


Each part of the definition is a direct translation of the definition
over nonarchimedean fields, adapted to the Lie algebra.  The choice
$a_\alpha = \alpha(X)$ causes the term $\Delta_{\rom2}$ in the
Langlands-Shelstad definition to equal $1$.  For the Lie algebra
transfer factor, we may take $\Delta_{2}=1$.

\section{Construction of motivic Shalika germs}

We give a proof of Theorem~\ref{thm:lie-shalika} asserting the
existence of motivic Shalika germs on the Lie algebra.

\begin{proof}
This section uses the following notation.  We make fixed choices
$G^{**}$, $\Sigma$, $\cF$, and so forth, as above.
We work with respect to a fixed cocycle space $Z$ that is used to
define a form $G$ of $G^{**}$ and inner form of $G^*$.

For $k\in \ring{N}$, let $\NF^k$ be the definable subset of $k$-tuples
of pairs $\Y_i=(N_i,\s_i)$, where $\Y_i$ is a Barbasch-Moy pair, and
$N_i$ is a nilpotent element $\fg_{\s_i,0}$, and such that the
$k$-tuple of first coordinates $(N_1,\ldots,N_k)$ are pairwise non
conjugate.  For any $\Y\in \NF^k$, we let $\s(\Y)=(\s_1,\ldots,\s_k)$
denote the tuple of second coordinates of $\Y$.  We write $\NF^k_\s$
for the subset of $\NF^k$ with second coordinates with $k$-tuple
$\s=\s(\Y)$.  For any $\Y=(N,\s)\in \NF^1$, let $1_\Y$ be the
characteristic function of $N+\fg_{\s,0+}$.

We form the motivic orbital integrals
\[
O(X,\Y) = \int_{O(X)} 1_{\Y} \,d\mu,
\]
for $X\in\fg$ that is nilpotent or regular semisimple,
$d\mu=d\mu^{\op{nil}}$ or $d\mu^\reg$ as appropriate, and $\Y\in
\NF^1$.  If $\Y\in \NF^k$, we let $O(X,\Y)$ be the $k$-tuple whose
$i$th coordinate is $O(X,\Y_i)$.

If $\Y=((N_1,\s_1),\ldots)\in \NF^k$, we write $\Theta(\Y)$ for the
square matrix with entries
\[
O(N_i,\Y_j),\quad\text{for}\quad i,j=1,\ldots,n.
\]
Let $\Theta^a(\Y)$ be the adjugate matrix of $\Theta(\Y)$, so that
\[
\Theta^a(\Y) \Theta(\Y) = \Theta(\Y) \Theta^a(\Y) = d_k(\Y)I_k,\quad 
\text{where}\quad d_k(\Y)=\det (\Theta(\Y)). 
\]
For $X\in \fg^\reg$ and $\Y\in \NF^k$, we define the Shalika germs
$\Gamma(X,\Y)$ as the $k$-tuple given by the matrix product
\[
\Gamma (X,\Y) = \Theta^a (\Y) O(X,\Y).
\]
It then follows directly from the
definition of adjugate that orbital integrals have a motivic Shalika
germ expansion
\[
d_k(\Y) O(X,\Y) = \Theta(\Y)\Gamma(X,\Y),
\]
That is, up to a determinant $d_k$ (which may be zero), the orbital
integral of $X$ of a collection of test functions indexed by $\Y$ can
be expanded in terms of the Shalika germs weighted by the nilpotent
orbital integrals of the test functions.


If we specialize the data to a nonarchimedean field $F$, take $z\in
Z(F)$, and choose $k$ be the number of nilpotent orbits of $G_z(F)$,
then it follows from Theorem~\ref{thm:bm} that there exists a
$k$-tuple $\s$ and $\Y\in \NF^k(F)$ for which $\det\Theta_F(\Y)$ is
nonzero. In fact, by suitable ordering of indices, the matrix is upper
triangular with nonzero diagonal entries.  We obtain the Shalika germ
expansion in its usual form.
\end{proof}

\section{a transfer principle for asymptotic relations}

Our aim is to transfer identities of Shalika germs from
one field to another.  Since germs express the asymptotics of
orbital integrals, we develop a transfer principle for
asymptotic identities.  We have the following transfer principle.

\begin{theorem} 
Let $S$ be a definable set.    
Let $g$ be a constructible function on $S$ and let
  $f:S\to \ring{Z}$ be a definable function.  For each $a\in\ring{Z}$,
let $S_a = f^{-1}(a)$.  Then there exists 
  $m\in\ring{N}$ such that for all $F,F'\in\Loc_m$ with the same
  residue field, we have the following: for all $a\in\ring{Z}$, 
  $g_F$ is zero on $S_a(F)$ if and only if $g_{F'}$ is zero
  on $S_a(F')$.
\end{theorem}

\begin{proof} We use the notation of the cell decomposition theorem
  \cite[Theorem 7.2.1]{CL}.  By quantifier elimination, we may assume
  that none of our formulas contain bound variables of the valued
  field sort.  Assume that $S\subset S'[1,0,0]$, for some definable set
  $S'$.  By cell decomposition, we may partition $S$ into finitely many
  cells $Z$ such that each cells comes with an definable isomorphism
  $\iota:Z \to Z'\subset S'[1,s,r]$ and a projection $\pi$ of the
  cell $Z'$ onto a base $C\subset S'[0,s,r]$.  Furthermore, there exists
  a constructible function $g_C$ on each base $C$ such that
\[
g_{|Z} = p^* g_C, \quad\text{where}\quad p = \pi\circ\iota.
\]
The morphism $p$ is an isomorphism followed by a
projection of a cell to its base.  As such, there exists $m$ such that
for all $F\in \op{Loc}_m$, the map $p_F$ is onto.

Similarly, we may pick a cell decomposition of $S$ adapted to the
definable function $f:S\to \ring{Z}$.  (We use a slightly stronger
property than what is stated in \cite{CL}.  Namely, we require $f_C$ to
be definable, rather than merely constructible.)  By \cite[Prop
7.3.2]{CL}, there is a common cell refinement that is adapted to both
$g$ and $f$.

Note that the base $C$ has one fewer valued field variable than $S$.
We iterate this construction, to eliminate all valued field variables.
After iteration, and choosing new constants $m$, $r$, and $s$, we
have the following situation.  $S$ can be partitioned into finitely
many definable sets $Z$, each equipped with a morphism $p:Z\to
C\subset h[0,s,r]$.  Furthermore, for each $C$ there is a definable
function $f_C:C\to\ring{Z}$ and constructible function $g_C$ on $C$
such that
\[
f_{|Z} = p^* f_C,\quad  g_{|Z} = p^* g_C.
\]
The morphisms $p_F:Z(F)\to C(F)$ are onto, for $F\in\op{Loc}_m$.

Let $F,F'\in\op{Loc}_m$ have the same residue field.  Since $g_C$ and
$f_C$ have no valued field variables, we may assume that $f_{C,F} =
f_{C,F'}$ and $g_{C,F} = g_{C,F'}$, when $F,F'\in\op{Loc}_m$ have the
same residue field.

The set $S_a$ is definable.  Assume that $g_{|S_a,F} = 0$.  Then on
any part $Z$,
\[
0 = g_{|S_a\cap Z,F} = p^*_F g_{C_a,F}, \quad\text{where}\quad C_a = f_^{-1}(a)
\]
Since $p_F$ is onto, $g_{C_a,F}$ is identically zero, and thus so is
$g_{C_a,F'}$ and $0 = p^*_{F'} g_{C_a,F'} = g_{|S_a\cap Z,F'}$.  Thus,
$g_{|S_a,F'}=0$.  This proves the theorem.
\end{proof}

\begin{cor}\label{cor:12} Let $g$ be a constructible function on $S\times\ring{Z}$.
  There exists $m$ such for all $F,F'\in\op{Loc}_m$ with the same
  residue field, we have the following.  If for some $a_F\in\ring{Z}$, the
  function $g(\cdot,a)_F$ is identically zero on $S(F)$ for all $a\ge
  a_F$, then the function $g(\cdot,a)_{F'}$ is also identically zero
  on $S(F')$ for all $a\ge a_F$.
\end{cor}

\begin{proof} Let $f:S\times \ring{Z}\to\ring{Z}$ be the projection
  onto the second factor.  The preimage of $a$ under $f$ is
  $S\times\{a\}$.  Apply the theorem to $g$ and $f$ for each $a\ge
  a_F$.
\end{proof}

We may apply the theorem and corollary to obtain a transfer principle
for asymptotic relations as follows.  Let $g_0$ be a constructible
function on $S$ and let $\chi_a$ be a definable family of support
functions on $S$ indexed by $a\in\ring{Z}$.  For example, $\chi_a$ might
be a family of characteristic functions of a shrinking family of
neighborhoods of a point $s_0$.  Then we obtain a constructible
function $g$ on $S\times\ring{Z}$ by $(s,a)\mapsto g_0(s)\chi_a(s)$.
The corollary gives a transfer principle for the vanishing of $g_0$ in
sufficiently small neighborhoods of the point $s_0$.  The size $a_F$ of the
neighborhood is allowed to vary with the field $F$.

\subsection{Uniformity of asymptotic relations}

We use the following lemma based on \cite[Th 4.4.4]{CGH}.

\begin{lem}  
  Let $\Lambda\times S$ be a definable set.  Let $g$ be a
  constructible function on $\Lambda\times S$.  Then there exists
  $m\in\ring{Z}$ and a constructible function $\locus{g}$ on $\Lambda$,
  such that for all $F\in\Loc_m$, the zero locus of $\locus{g}_{\,F}$
  equals the locus of
\[
\{v\in \Lambda(F)\mid \forall s\in S(F),\quad g_F(a,s)=0\}.
\]
\end{lem}

\begin{proof}  Theorem 4.4.4 of \cite{CGH} works with exponential
  constructible functions instead of constructible functions, but the
  identical proof applies when working with non-exponential functions.

  Also, that theorem is restricted to $S=h[n,0,0]$.  We show that
  this special case implies the more general statement of our lemma.
  First, the case $h[n,n',n'']$ reduces to $h[n+n'+n'',0,0]$ by
  replacing each $\ring{Z}$-variable with $\ord\, x$ for some new $\VF$
  parameter $x$, and replacing each residue variable with some $\ac\, x$
  for some new $\VF$ parameter $x$.

  If $S\subset h[n,n',n'']$ is arbitrary, then we replace the function
  $g$ with the function $g 1_S$, where $1_S$ is the indicator function
  on $S$ to go from $h[n,n',n'']$ to $S$.  We are now in the context
  covered by Theorem~4.4.4.
\end{proof}


We can actually show that the bound $a_F$ in the previous corollary
can be chosen to be independent of the field $F$.

\begin{thm}
Let $g$ be a constructible function on a definable set $S\times\ring{Z}$.  Suppose
that there exists $m\in\ring{Z}$ such that for all $F\in \op{Loc}_m$, there
exists $a_F$ for which $g(\cdot,a)_F$ is identically zero on $S(F)$,
for all $a\ge a_F$.   Then there exists a single $a_0$ such that we
can take $a_F=a_0$ for all $F\in \op{Loc}_m$.
\end{thm}

\begin{proof}
We apply the lemma with $\Lambda=\ring{Z}$ to obtain a constructible
function $\locus{g}$ on $\ring{Z}$ that describes the locus of identical
vanishing of $g$, which is a subset of $\ring{Z}$ depending on $F\in\Loc_m$.

A constructible function $\locus{g}$ lies in the tensor product
$Q(\ring{Z})\otimes_{P^0(\ring{Z})} P(\ring{Z})$, where $P(\ring{Z})$
is the ring of constructible Presburger functions on $\ring{Z}$ and
$Q(\ring{Z})$ is a quotient of the free abelian group with generators
$[Y]\in \ring{Z}[0,n',0]$.  We may assume by quantifier elimination
that $\locus{g}$ contains no variables of valued field sort.  The
function $\locus{g} = \sum q_i\otimes p_i$ may depend on residue field
variables through $q_i\in Q(\ring{Z})$.

By the definition of constructible Presburger functions $p_i$ on
$\ring{Z}$, we may partition $\ring{Z}$ into a finite (field
independent) disjoint union of Presburger sets such that on each of
these sets, $\locus{g}$ has the form
\[
\locus{g}_F(t) = \sum_{i=1}^\ell c_i t^{a_i} q^{b_i t},
\]
where each $c_i$ depends only on residue field variables, the integers
$a_i$ and $b_i$ do not depend on any variables in the Denef-Pas
language, and we can assume that $(a_i,b_i)$ are mutually different
for different $i$.

The key point is that such a function can have only finitely many
zeroes, and their number is bounded by a constant that depends only on
the number $\ell$ of terms in the sum.  This is \cite[Lemma 2.1.7]{CGH1},
and follows from $o$-minimality of the reals with the exponential.
Thus, the only way for the zero locus of $\locus{g}$ to contain all integers
greater than $a_F$ is for the coefficients $c_i$ to be zero on all
unbounded Presburger sets in the disjoint union for all fields $F$.

If we now take $a_0$ to be larger than the maximum of all the bounded
Presburger sets in the the disjoint union, then this integer works
uniformly for all fields $F$.
\end{proof}



We use the following result from \cite{CGH2}.  We conjectured
this result to hold based on the 
requirements of smooth endoscopic matching.

\begin{theorem}[Cluckers-Gordon-Halupczok]\label{thm:cgh}
Let $f:P\to B$ be a definable morphism between
  definable sets.  For $\theta$ any constructible function on $P$, write
  $\theta_{b}$ for the constructible function $P_b$ on the fiber over $b\in
  B$.  Let $\Theta$ be a list (i.e., finite ordered tuple) of constructible
  functions on $P$. For $b\in B$, let $\Theta_b$ be the corresponding list
  of constructible functions on the fiber $P_b$.  Then there exists a
  natural number $m$ with the following property.  For every $F,F' \in
  \Loc_{m}$ such that $F$ and and $F'$ have isomorphic residue
  fields, the following holds: if for each $b\in B_F$, the list of
  functions $\Theta_{b,F}$ is linearly dependent, then then also for each
  $b'\in B_{F'}$, the list of functions $\Theta_{b',F}$ is linearly
  dependent.
\end{theorem}

Let $S \subset S'\times\ring{Z}$ be any sets.  For $a_0\in\ring{Z}$,
we say that a list of functions $[f_1;\ldots;f_r]$ (each taking values
in $\ring{C}$) are {\it $a$-asymptotically linearly dependent} if for all $a\ge
a_0$, the list of functions $[f_{1\,|S_a};\ldots;f_{r\,|S_a}]$ is
linearly dependent, where $f_{|S_a}$ denotes the restriction of $f$ to
$S\cap (S'\times\{a\})$.

We need an asymptotic variant of the theorem.


\begin{theorem}\label{thm:cgh-asymp}
Let $f:P\to B$ be a definable morphism between
  definable sets.   Assume that $P\subset P_1\times\ring{Z}$, for some
  definable set $P_1$ and that $f:P\to B$ extends to
  $P_1\times\ring{Z}\to B$.  Let $\Theta$ be a list (i.e., finite ordered tuple) of constructible
  functions on $P$.   Then there exists a
  natural number $m$ with the following property.  For every
  $a_0\in\ring{Z}$ and every $F,F' \in
  \Loc_{m}$ such that $F$ and and $F'$ have isomorphic residue
  fields, the following holds: if for each $b\in B_F$, the list of
  functions $\Theta_{b,F}$ is $a$-asymptotically linearly dependent, then then also for each
  $b'\in B_{F'}$, the list of functions $\Theta_{b',F}$ is $a$-asymptotically linearly
  dependent.
\end{theorem}

\begin{proof} We adapt the proof in \cite{CGH2}.

We may replace $f_i$ with $f_i 1_P$ on $P_1\times\ring{Z}$ to reduce
to the case $P=P_1\times\ring{Z}$.  We may reduce asymptotic linear
dependence to the identical vanishing of a determinant.
Let 
\[
S = \{(b,a,X_1,\ldots,X_r)\mid (X_1,a),\ldots,(X_r,a)\in P\times_B
P\times_B\cdots\times_B P,~X_i\mapsto b\} \subset B\times\ring{Z}\times P^r.
\]
We have a constructible function $g$ on $B\times\ring{Z}\times P^r$
given by the indicator function of $S$ times
\[
(b,a,X_1,\ldots,X_r)\mapsto 
\det f_i((X_j,a)).
\]
By \cite{cgh}, there is a constructible function $\locus{g}$ on
$B\times\ring{Z}$ giving the locus of identical vanishing of $g$ on
$(B\times\ring{Z})\times P^r$.
Let $m\in\ring{Z}$ be an integer that is sufficiently large for
Corollary~\ref{cor:12} applied to $\locus{g}$ and large enough for the
$Iva$ locus in \cite{cgh}.

Let $F,F'\in\Loc_m$ have isomorphic residue fields.  Choose any
$a_0\in\ring{Z}$.  Assume that for all $b\in B(F)$, the list of
functions $\Theta_{b,F}$ is $a_0$-asymptotically linearly dependent.
Then $g_F$ vanishes identically for each element of $A:=\{(b,a)\mid
a\ge a_0\}$.  Then $\locus{g}_F$ is zero on $A$.  By
Corollary~\cite{cor:12}, $\locus{g}_{F'}$ is zero on $A$.  Following
the preceding steps backwards now for $F'$ instead of $F$, from
$\locus_{g}_F'$ back to asymptotic linear depedence, we conclude that
$\Theta_{b,F'}$ is $a_0$-asymptotically linearly dependent.
\end{proof}



\section{smooth endoscopic matching}


We will invoke Theorem~\ref{thm:cgh-asymp} a finite number of times for different
($r$-tuples of) constants $l$, $k$, $t$, $\s$.  These constants as
well as $r$-tuples (with $r$ bounded) of these constants range over a
finite set of possibilities.  For instance, $l$ is bounded above by
the number of nilpotent conjugacy classes in $\fh$.  Corollary~\ref{cor:nilbound}
gives a
constant $l_{\op{max}}$ such that for all nonarchimedean fields of
sufficiently large residue characteristic, the number of nilpotent
conjugacy classes is at most $l_{\op{max}}$.  So $0\le l\le
l_{\op{max}}$.  We can arrange so that the bound $l_{\op{max}}$
depends only on fixed choices such as the root data for the endoscopic
group, and not on the nonarchimedean field.  Similarly, each component
of $t$ is bounded by the number of parahoric subgroups up to
conjugacy, so that $t$ is constrained to run over finitely many
possibilities (for each fixed choice of root data). By invoking the
theorem a finite number of times, we may pick a single natural numbers
$m$ that works in all cases by taking the maximum of the natural
numbers obtained in the finitely many separate cases.  Let $m$ be a
natural number that works for all $k,l,\s,t$ (keeping the other
choices fixed).

Let $F'\in\op{Loc}_m$.  Choose a second field $F\in \op{Loc}_m$ that
has characteristic zero and a residue field isomorphic to that of
$F'$.

Our argument is highly depedent on making choices in the right
order.  First we pick root data for $G$ and $H$ with all the
associated fixed choices $\Sigma,\cF,\ldots$.  Then we bound various
discrete parameters 
$r,l,k,\s,t$ to lie in a finite set.  Then we pick $m$, $F$, and $F'$.
Then we pick the particular values of $r,l,k,\s,t$ among the finite
set of possibilities.

Let $X_H,X_G,\bar X_H,\bar X_G$ be parameters that appear in the
transfer factor.  As described above, we have invariant motivic 
measures $d\mu^\reg_G$
and $d\mu^\reg_H$ on the stable orbits of $X_G$ and $X_H$.

We define $\kappa$ orbital integrals by the following equation:
\begin{equation}\label{eqn:kappa}
O^\kappa(X_H,X_G,\bar X_H,\bar X_G,f) = \int_{x\in O^{st}(X_G)}
\Delta(X_H,x,\bar X_H,\bar X_G) f (x)\,d\mu^\reg_G,
\end{equation}
where $f$ is a constructible motivic function on $\fg$ and the
parameters $(X_H,X_G,\bar X_H,\bar X_G)$ run over the definable set
$V$ in Equation \ref{eqn:delta-domain}.  Similarly, we define stable
orbital integrals on $\fh$ by
\begin{equation}\label{eqn:stable}
SO(X_H,f^H) = \int_{x\in O(X_H)} f^H (x)\,d\mu^\reg_H,
\end{equation}
where $f^H$ is a constructible motivic function on $\fh$.


Waldspurger has proved that the germs of $O^\kappa$ are linear
combinations of the germs of $SO$ in Equation~\ref{eqn:stable} when
the field $F$ has characteristic zero~\cite{W}.

The basic naive strategy is to invoke the Theorem~\ref{thm:cgh-asymp} to transfer these
linear relations to a field $F'$ of positive characteristic.
Unfortunately, the naive strategy does not work, because the
coefficients of the linear relation potentially vanish identically
on part of the cocycle space $Z$, which would mean that we would
obtain no information about the endoscopic matching for some of the
twisted forms of a reductive group.  We note that it is not always
possible to isolate definably a single isomorphisms class of reductive
groups $G$.  See Section~\ref{sec:classification}.

The next (less naive) strategy is take a product $G =
G_1\times\cdots\times G_r$ over all of the groups up to isomorphism
that are parameters in the cocycle space $Z$.  On this product, the
corresponding cocycle space does in fact determine a single group up
to isomorphism.  The germs on this product are the products of germs
of the factors.  We can invoke the theorem to transfer linear
relations for this product of groups.  Unfortunately, this strategy
also fails, because all of the $\kappa$-Shalika germs might vanish
identically for one of the factors; and when this happens, we cannot
derive any information from the product about the other individual factors
$G_\rho$.

This nonetheless, leads to a strategy that works.  Again we take
$r$-fold products of factors, but we enhance our collection of factors
to include the stable Shalika germs on the quasi-split endoscopic
groups.  We know how to choose stable Shalika germs that are nonzero.
Indeed the stable Shalika germ of the regular nilpotent class in a
quasi-split inner form is
nonzero.  Thus, we are able to avoid the bad situation where one of
the factors vanishes identically.  This works.

\subsection{Application to endoscopic matching of Shalika germs}

We change notation slightly to let $Z^1$ denote the
cocycle space.  For each $r\in\ring{N}$, let $Z^r$ be the definable
set of $r$-tuples $(z_1,\ldots,z_r)$, where $z_i \in Z^1$ and such
that $z_i$ and $z_j$ are not cohomologous for $i\ne j$.  There exists
$m$ and a bound $r_{\op{max}}$ such that for all $F\in \op{Loc}_m$,
and all $r\ge r_{\op{max}}$, we have $Z^r(F) = \emptyset$.

The Lie algebras $\fh_z$, $\fg_z$ depend on a parameter $z\in Z^1$.
We have spaces
\begin{align*}
B^1 = B^1_{kl\s t} = &\{(z,X_G,\bar X_H,\bar X_G,\Y_G,\Y_H) \mid \\
     &\quad z\in Z, X_G\in \fg_z,
\bar X_H \in \fh_z, \bar X_G\in \fg_z, \Y_G\in \NF^k_{G,t}, \Y_H\in \NF^l_{H,\s}\}.
\end{align*}
For any $r$-tuples $k=(k_1,\ldots,k_r)$, $l=(l_1,\ldots,l_r)$,
$\s = (\s_1,\ldots,\s_r)$, $t=(t_1,\ldots,t_r)$, we set
\[
B^r = B^r_{kl\s t} = \{(b_1,\ldots,b_r) \mid b_i \in B^1_{k_il_is_it_i}\quad
  (z(b_1),\ldots,z(b_r))\in Z^r \}.
\]
(We apologize for the nesting of subscripts.  Note that $\s_i$ is
itself a tuple indexing parahorics.)  For $b\in B^r$, we write
$z(b_i)$, $X_G(b_i)$, and so forth for the components of $b_i$.  We
define for $1$-tuples $k$, $l$, $\s$, $t$:
\[
P^1_{kl\s t} = P^1 = \{(X_H,b) \mid b\in B^1_{kl\s t},
\quad z = z(b),\quad X_H \in \fh_z\}.
\]
Also, for $r$-tuples $k$, $l$, $\s$, $t$:
\[
P^r_{kl\s t} = P^r = \{(X_H,b) \mid b = (b_1,\ldots,b_r)\in B^r_{kl\s t},\quad
  X_H = (X_H^1,\ldots,X_H^r), \quad (X_H^i,b_i) \in P^1_{k_il_is_it_i}\}.
\]
We have a projection map $P^r\to B^r$ onto the second coordinate.
Our two definitions of $Z^1$, $B^1$ and $P^1$ are coherent when
$r=1$.

For any $r$-tuples $l=(l_1,\ldots,l_r)$ and $k=(k_1,\ldots,k_r)$, set
$J_\rho = J^+_\rho\sqcup J^-_\rho$, where $J^-_\rho =
\{1,\ldots,k_\rho\}\times\{-\}$ and $J^+_\rho =
\{1,\ldots,l_\rho\}\times\{+\}$.
\[
J = \{i = (i_1,\ldots,i_r)\mid i_\rho=(i'_\rho,\pm)\in J_\rho\}.
\]
Let
\[
J^+ = \{i = (i_1,\ldots,i_r)\mid i_\rho=(i'_\rho,+)\in J^+_\rho\}.
\]


For $i\in J_{kl}$, 
We define constructible functions.  For $(X_H,b)\in
P_{kl\s t}$, and $\rho=1,\ldots,r$,
set
\[
\psi^{\rho,i_\rho}(X_H^\rho,b_\rho) = 
\begin{cases}   
  O^\kappa(X_H^\rho,X_G(b_\rho),\ldots,1_{\Y_G^{i'_\rho}(b_\rho)}),
   & \text{if } i_\rho = (i'_\rho,-)\in J^-_\rho \\
  SO(X_H^\rho,1_{\Y_H^{i'_\rho}}),
   & \text{if } i_\rho = (i'_\rho,+)\in J^+_\rho.
\end{cases}
\]
 We also define constructible functions
on $P_{kl\s t}$ indexed by $i\in J$:
\[
\theta^i(X_H)_b = \prod_{i=1}^r \psi^{\rho,i_\rho}(X_H^\rho,b_\rho).
\]

We use the following elementary facts about tensor products of
finite dimensional vector spaces in our analysis of the products
defining $\theta^i$.

\begin{lem}\label{thm:tensor}
Let $V_1,\ldots,V_r$ be finite dimensional vector spaces of
dimensions $\dim(V_i) = n_i$.  
\begin{enumerate}
\item 
Let $S_\rho\subset V_\rho$ be a finite set of vectors for each
  $\rho = 1,\ldots,r$. 
If the product of the cardinalities of the
  sets $S_\rho$ is $n_1\cdots n_r$ and if the tensors $w_{i_1}\otimes
 \cdots\otimes w_{i_r}$, for $w_{i_\rho}\in S_\rho$, span 
$V_1\otimes\cdots\otimes V_r$, then
  for each $\rho$, the set $S_\rho$ is a basis of $V_\rho$.
\item For each $\rho=1,\ldots,r$, let $W_\rho$ be a subspace of
$V_\rho$.  Let $v_{1}\otimes \cdots \otimes v_{r} \in
  W_1\otimes\cdots\otimes W_r$, with $v_{\rho}\in V_\rho$.  For
  each $\rho$, if $v_{{\rho'}}\ne0$, for all $\rho'\ne\rho$, then
  $v_{\rho}\in W_\rho$.
\end{enumerate}
\end{lem}

\begin{proof}
\end{proof}

Let $r$ be the maximum for which $Z^r(F)\ne\emptyset$.  That is, it is
the number of nonisomorphic reductive groups in the family of
cocycles.  This maximum can be expressed as sentence in the Denef-Pas
language and can hence be transferred to $F'$ by an Ax-Kochen style
transfer principle.  Thus, $r$ is also the maximum for which
$Z^r(F')\ne\emptyset$.

Pick $z\in Z^r(F')$.  Let $k=(k_1,\ldots,k_r)$ be tuple such that
$k_\rho$ is the number of nilpotent orbits in $\fg_{z_\rho,F'}$.  By the
Barbasch-Moy classification of nilpotent orbits in terms of residue
field parameters~\cite{barbasch-moy}, $k_\rho$ is also equal to the
number of nilpotent classes in $\fg_{z_\rho,F}$.  Choose a $k_\rho$-tuple
$\s_\rho\in \cF_G^{k_\rho}$ such that $\NF^{k_\rho}_{\s_\rho,F'}\ne
\emptyset$.  Let $\s = (\s_1,\ldots,\s_r)$.

Let $l_\rho$ be the dimension of the space of stable Shalika germs on
$\fh_{z_\rho}$.  By the definition of stable distribution, this equals
the dimension of the space of stable orbital integrals supported on
the unipotent set.  For each $\rho$, let $\mu_i^{st}$, for $i\le l_\rho$,
be a basis of the stable distributions supported on the unipotent set
of $H_{z_\rho}$.  We have a Shalika germ expansion
\[
SO(X,f^H) = \sum_{i=1}^l S\Gamma_i(X) \mu_i^{st}(f^H),
\]
for a compactly supported function $f^H$ on $H(F)$ and some stable
germs $S\Gamma_i$.  The germ
expansion holds on $G$-regular semisimple elements in a suitable
neighborhood (depending on $f^H$).  For each $\rho$, we pick a
parameter $\Y^\rho_{H_{z_\rho},F}\in \NF^{l_\rho}_{H_{z_\rho},F}$ such
that the functions $1_{\Y_H^{\rho,j}}$, for $j=1,\ldots,l_\rho$, form a
dual basis to $\mu_i^{st}$.  (The functions span the space dual to
invariant distributions with nilpotent support, hence span the
quotient space dual to stably invariant distributions with nilpotent
support; hence a subset of the functions forms a dual basis.)  We can
do this by Theorem~\ref{thm:bm}.  Let $t_\rho = \s(\Y^\rho_{H_{z_\rho},F})\in
S^{l_\rho}_{H}$ and $t = (t_1,\ldots,t_r)$.  We have now fixed the
choices $r,k,l,\s,t$.  To lighten notation, we sometimes drop
$r,k,l,\s,t$.  We also return to our earlier notation of $z$ as a
parameter running over the definable set $Z^r$.

We may construct germs of functions as follows.  Let $1_a$, for
$a\in\ring{Z}$ be sequence of support functions, defined as the
indicator functions of small balls around $0$ in $\fg$ tending to $0$
as $a$ tends to infinity.  We do the same for $\fh$, ambiguously
denoted that support sequence $1_a$ as well.  We may truncate the
orbital integrals using these functions $1_a SO(X,f^H)$ and $1_a
O^\kappa(X_H,X_G,\bar X_H,\bar X_G,f)$.  Integrals now have an extra
integer parameter $a\in\ring{Z}$ that we may use to study
$a$-asymptotic linear dependence.  Similarly, (abusively) write $1_a\theta^i_b$
for the truncation of $\theta^i_b$.

We claim that for all $j \in J\setminus J^+$, there exists
$a_0\in\ring{Z}$, such that for each $b\in
B^r(F)$, the functions $1_{a_0,F}\theta^i_{b,F}$, for $i\in \{j\}\cup J^+$,
are $a_0$-asymptotically linearly dependent.  If the functions indexed by $J^+$ are
$a_0$-asymptotically linearly dependent, this is immediate.  If, on the other hand, the
functions indexed by $J^+$ are linearly dependent, then there exists
a parameter $\Y_H(b)$ from the base space $B^r(F)$ such that for each
$\rho$, the functions $1_{\Y_{H,b}^{\rho,j}}$ give a dual basis to the
stable distributions on the nilpotent set of $\fh_{z_\rho}$.  We use
Waldspurger's fundamental result \cite{W} that the $\kappa$-orbital
integral (over $F$) admits smooth matching, which implies that
$\theta^{j}_{b,F}$ is a linear combination of $\theta^i_{b,F}$, for
$i\in J^+$.  This proves the claim.
(XX fix, need the right dependence on $b$ and $a_0$ in the logic.)

We apply Theorem~\ref{thm:cgh-asymp} to the morphism $f:P^r_{kl\s t}\to
B^r_{kl\s t}$.  By the theorem, for each $j\in J\setminus J^+$, and for
all $b'\in B^r_{F'}$, the functions $\theta^i_{b',F'}$, for $i\in
\{j\}\cup J^+$ are linearly dependent.  Let $c^i_{j,b'}\in \ring{C}$
be the coefficients of a nontrivial linear combination (depending on
$b'$).  We write this relation as
\begin{equation}\label{eqn:du}
\sum_{i\in \{j\}\cup J^+} c^i_{j,b'}\theta^i_{b',F'}  = 0.
\end{equation}
For each $j$ and $b'$, the nontriviality of the relation
means
that $c^i_{j,b'}\ne 0$ for some $i$.

We claim that there exists $b'\in B^r_{F'}$, such that for all $j\in
J\setminus J^+$,
$c^j_{j,b'}\ne 0$.  Otherwise, for all $b'\in B^r(F')$, there exists
$j\in J\setminus J^+$ such that $c^j_{j,b'}=0$.  Equation~\ref{eqn:du}
becomes
\begin{equation}\label{eqn:du}
\sum_{i\in J^+} c^i_{j,b'}\theta^i_{b',F'}  = 0.
\end{equation}
This asserts that for all $b'\in B^r(F')$, the functions
$\theta^i_{b',F'}$ are linearly dependent.  Applying
Theorem~\ref{thm:cgh-asymp} again in the opposite direction, to go from
dependence on $F'$ to linear dependence on $F$, we find that for all
$b\in B^r(F)$, the functions $\theta^i_{b,F}$, for $i\in J^+$ are linearly
dependent.  The choice of $l,t$ gives linear independence, contrary to
the assumption of dependence.  This gives the claim.

Let $b'\in B^r_{F'}$ be the parameter as in the claim.  The
nonvanishing $c^j_{j,b'}\ne 0$, gives for each $j$ a formula for
$\theta^j_{b',F'}$ in terms of products of stable orbital integrals on
factors $\fh_{z_\rho}$.  In particular, fixing $\rho$, we let
$j = (i_1,\ldots,i_r)$, where 
\[
i_{\rho'} = \begin{cases} (j_{0,\rho'},+), &\text{if } \rho'\ne \rho\\
       (i'_0,-) & \text{if } \rho' = \rho.
  \end{cases}
\]
We let $(j_{0,\rho'},+)\in J^+$ be chosen so that it represents the
regular stable nilpotent orbit in the endoscopic group
$\fh_{z_\rho'}$, and we let $i'_0\le k_{\rho}$ be arbitrary.  It is
known by Langlands and Shelstad that when suitably normalized, the
Shalika germ of the stable regular nilpotent class is identically $1$.
In particular, it is nonzero.  Considering the linear relation
involving $\theta^j_{b',F'}$ as a function of $X_H^\rho$ alone, it
gives a nontrivial relation between functions
$X_H^\rho\mapsto\psi^{\rho,i'_0}(X_H^\rho)_{b'}$ (with nonzero
coefficient) and the stable nilpotent orbital integrals on
$\fh_{z_\rho}$. As we run over all $i'_0$ and $\rho=1,\ldots,r$, we
obtain the desired matching of all $\kappa$-Shalika germs on all
reductive groups in the cocycle space $Z^r$.  This proves the theorem.


(An analysis of the proof shows that we might modify the proof to take
$J^+_\rho$ to be a singleton set for each $\rho$ with $\Y_H^{i'_\rho}$
a regular nilpotent element.)

\section{Adaptation to (unipotent) Shalika germs}

We now adapt the results of the previous two sections to unipotent
classes in the group.

All of the proofs go through with the following changes.

Functions on the Lie algebra (Shalika germs, orbital integrals) are
replaced with functions on a neighborhood of the identity element in
the group.  We use notation $\gamma_G,\gamma_H,\bar \gamma_G,\bar
\gamma_H$, and so forth instead of $X_G,X_H,\bar X_G,\bar X_H$.

Nilpotent elements are replaced with unipotent elements.  In large
residue characteristic,
the exponential map is defined on the set of nilpotent elements and
gives a bijection between nilpotent classes in the Lie algebra and
unipotent elements in the group.  The exponential map is polynomial
and shows that the set of unipotent elements is a definable set.

The set of parahoric subalgebras are replaced with parahoric subgroups.
The Kostant section is replaced with the Steinberg section.  

The
choice of $a$-data becomes $a_\alpha = \alpha(\gamma)^{1/2} -
\alpha(\gamma)^{-1/2}$ (instead of $\alpha(X)$).  For this, we
restrict $\gamma$ to a neighborhood of $1$ on which square roots can
be extracted.

We may assume that the multiplicative characters that occur in the
transfer factor are trivial on topologically unipotent elements.  This
was observed in \cite{hales-simple}.  The point is that we may choose
the multiplicative characters to be tamely ramified, and any such
character is trivial on topologically unipotent elements.

\section{Endoscopic matching}

In \cite{LSxf}, Langlands and Shelstad define the notion of matching
(transfer) of orbital integrals from a reductive group over a local
field to an endoscopic group.  In particular, they introduce the
transfer factors that are needed for the matching.  In \cite{LSd},
they take a the further step of reducing the existence of matching to
a local statement at the identity in the centralizer of a semisimple
element.  We review this reduction below.  It is this reduction that
we will use in this section, in combination with the endoscopic
matching of Shalika germs from the previous section, to establish
endoscopic matching in sufficiently large characteristic.

In earlier sections, the primary focus was on constructible
functions.  In this section, the focus is on reductive groups over
local fields in sufficiently large residue characteristic.


In this section, we prove Theorem~\ref{thm:xfer}.
Before presenting the proof, we recall some results from \cite{LSd}.
That article assumes that $F$ is a local field of characteristic zero.
We will need to relax the restriction on the characteristic.  In fact,
the assumption is rarely used.  It is used to cite other articles that
also assume characteristic zero.  We need Harish-Chandra descent of
orbital integrals near a singular semisimple element, which holds in
positive characteristic by \cite{XX}.

The only essential use of the assumption appears in Case III on page
558 of the article, which uses the global argument to deduce a local
identity in characteristic $2$.  We assume that our fields have
sufficiently large residue characteristic, and in particular, we can
do without this case in characteristic $2$.

Let $G$ be a reductive group over a non-archimedean local field $F$,
and let $H$ be an endoscopic group of $G$.    We recall \cite[Sec.2.1]{LSd} that $(G,H)$
admits {\it endoscopic $\Delta$-matching} if for each $f\in
C_c^\infty(G(F))$ there exists $f^{\tilde H}\in C_c^\infty(\tilde
H(F),\tilde \lambda)$ such that $f$ and $f^{\tilde H}$
have\footnote{Langlands and Shelstad fix $\tilde{\bar \gamma_H}$ and
  $\bar\gamma_G$ in the transfer factor and then drop them from
  notation, so that only the first two variables of $\Delta$ are displayed.}
$\Delta$-matching orbital integrals:
\[
SO(\tilde \gamma_H,f^{\tilde H}) 
 = \sum_{\gamma_G} \Delta(\tilde\gamma_H,\gamma_G)O(\gamma_G,f)
\]
for all strongly $G$-regular $\tilde \gamma_H \in \tilde H(F)$.
The group $\tilde H$ is obtained as an admissible $z$-extension from
$H$ as explaned in \cite[Sec. 4.4]{LSxf}.

We recall that $(G,H)$ admits {\it local endoscopic $\Delta$-matching}
if for any $f\in C_l(G(F))$ we can find $f^H \in C_c^\infty(H(F))$
such that
\[
SO( \gamma_H,f^{ H}) 
 = \sum_{\gamma_G} \Delta_{loc}(\gamma_H,\gamma_G)O(\gamma_G,f)
\]
for all strongly $G$-regular elements $\gamma_H$ near $1$ in $H(F)$.
(We may allow the size of the neighborhood of $1$ to depend on $f^H$.
In fact, it is enough to match finitely many functions that span the
dual to the space of invariant distributions supported on the
unipotent set.

For each semisimple element $\epsilon_H \in H(F)$ that is the image of
some $\epsilon_G\in G(F)$, Langlands and Shelstad construct an
endoscopic pair $(G_{\epsilon_G},H_{\epsilon_H})$ with corresponding
transfer factor $\Delta_\epsilon$.  The group $G_{\epsilon_G}$ is the
connected centralizer of $\epsilon_G$.

We will need the following result.

\begin{theorem}[Langlands-Shelstad 2.3.A]  Suppose all pairs
  $(G_{\epsilon_G},H_{\epsilon_H})$ have local endoscopic
  $\Delta_\epsilon$-matching, then $(G,H)$ has endoscopic $\Delta$-matching.
\end{theorem}

Conversely, if endoscopic matching holds on the entire group, then
local endoscopic matching holds in particular at the identity:
$\epsilon_G = 1$ and $\epsilon_H =1$.

Note that the admissible $z$-extensions are needed to formulate the
statement of endoscopic $\Delta$-matching, but they do not appear in
the statement of local endoscopic matching.  Thus, the admissible
$z$-extensions will play no part in our proof.

We now turn to the proof of Theorem~\ref{thm:xfer}.

\begin{proof} Let $G$ be a definable reductive group over a cocycle
  space $Z$.

  Over a local field, there are only a finite number of endoscopic
  groups, up to conjugacy.  Also, there are only finitely many
  different centralizers that are obtained by descent, up to conjugacy
  \cite[Sec.2.2]{LSd}.  We can state this uniformly as the field
  varies: there are finitely many definable connected reductive
  centralizers each having finitely many definable reductive
  endoscopic groups such that for all fields of sufficiently large
  residual characteristic, all the centralizers and their endoscopic
  groups are obtained from these finitely many definable groups by
  specialization to the field in question up to conjugation of
  centralizers and equivalence of endoscopic data.  For each separate
  definable pair $(G_{\epsilon_G},H_{\epsilon_H})$, we will obtain a
  natural number $m$ that works for all fields $F\in \op{Loc}_m$.
  Then the maximum of all such $m$ will work for all descent data.

  By Theorem~\ref{XX}, for each endoscopic pair
  $(G_{\epsilon_G},H_{\epsilon_H})$, there exists $m$ such that we
  have local endoscopic $\Delta_\epsilon$ matching for all
  $F\in\op{Loc}_m$.  Note that the local endoscopic matching is
  a direct consequence of the endoscopic matching of Shalika germs.  This
  completes the proof.
\end{proof}


\section{classification of definable reductive groups}\label{sec:classification}

We rely on the classification of reductive groups from \cite{Gille},
\cite{Tits}, \cite{Sel}, \cite{Petrov}, \cite{Reeder}, closely following the
presentation in \cite{Gross}.  We refer to \cite{Gross} for necessary
details about our presentation here.  We recall that we do not have
access to a Frobenius element of the Galois group, which is used in
essential ways in the classification.  We make do the best we can with
$\op{qFr}$, a generator of $\Sigma^{unr}$.

We describe the extent to which nonisomorphic reductive groups $G$
over a given field $F$ can give the same set of fixed choices above.
For simplicity, assume that $G$ and $G'$ are absolutely almost simple
and simply connected.  Assume that the fixed choices can be made for
$G$ and $G'$ so that they have identical combinatorial data.  There
are obvious such cases:
\begin{enumerate}
\item $G$ and $G'$ are inner forms of $SL(n)$, and their invariants in
  the Brauer group are $a/m$ and $a'/m$ with $(a,m)=(a',m)=1$.
\item $G$ and $G'$ have isomorphic split outer forms, are not split,
  are quasi-split, and split over different ramified
  extensions of the same degree.  For example, we may have two
  quasi-split special unitary groups that split over different
  ramified quadratic extensions.
\item $G$ and $G'$ are not quasi-split, but are inner forms of
  quasi-split groups $G^*$ and $G'^{*}$ described in the previous
  situation (and that split over a ramified quadratic extension).  
  For example, we may have two nontrivial inner forms of
  quasi-split special unitary groups that split over different
  ramified quadratic extensions.
\end{enumerate}

\begin{lem} Let $G$ and $G'$ be almost simple simply connected groups
  for which the fixed choices can be made equal.  Then $G$ and $G'$
  fall into one of the three cases just enumerated.
\end{lem}

\begin{proof}
  Specifically, suppose that we are given $F$ and an enumerated Galois
  group $\op{Gal}(E/F)$ that splits $G$.  Suppose that it has a short
  exact sequence
\[
1 \to\op{Gal}(E/E^{unr})\to \op{Gal}(E/F) \to \op{Gal}(E^{unr}/F)\to 1,
\]
that is isomorphic under the enumeration with the fixed data
\[
1 \to\Sigma^t\to\Sigma\to\Sigma^{unr}\to 1.
\]
The enumeration fixes a distinguished generator $\op{qFr}$ of
$\Sigma^{unr}$, which need not correspond with the Frobenius generator
of $\op{Gal}(E^{unr}/F)$.

In what follows, we use the notation of the fixed choices in
Subsection~\ref{sec:fixed}.  We refer to \cite{Gross} for a detailed treatment
of this subject material.

First, consider the case $e=1$.  Then the group splits over an
unramified extension.  Identifying $\op{Gal}(E^{unr}/F)$ with
$\Sigma^{unr}$, the fixed data provides an action
$\phi:\op{Gal}(E^{unr}/F)\to \op{Aut}(\R)$ on the affine diagram.  When
$\R = \cal{A}_n$, the different images correspond to different Tits
indices.  We can specify the Tits indices but not the particular form
of the anisotropic kernel (unless we are given the Frobenius generator).
This is the first case.

In the other affine diagrams with $e=1$, by case-by-case inspection,
if the images of two homomorphisms $\phi$, $\phi'$ are equal, then
they are equal up to conjugation by $\op{Aut}(\R)$.  This homomorphism
modulo conjugation determines the cohomology class of the cocycle
defining the form of the group $G$.  Hence the underlying groups are
isomorphic, and no cases occur here.

Next, consider the case of quasi-split forms of the same split group
with $e>1$.  In this case, the group is determined by the action on
the Dynkin diagram modulo conjugation:
\[
\op{Gal}(E/F)\to \op{Aut}(\op{Dyn}).
\]
The image is cyclic with corresponding extension totally
ramified, except in the case of an $S_3$-action on the Dynkin diagram.

Finally, consider the case of non quasi-split inner forms with $e>1$.  In
this situation, the only affine diagrams ${}^e\R$ with automorphisms
are ${}^2\cal{A}_{2n+1}$ (with one inner form) and ${}^2\cal{D}_n$
(with one inner form).  The nontrivial inner form is determined by the
quasi-split inner form, and each of these is determined by the
ramified quadratic extension that splits it. This is the third case.
\end{proof}

\section{Open Problems}

\subsection{Igusa theory}

Langlands and Igusa have results about the asymptotic behavior of
integrals \cite{langlands:sl3} \cite{igusa}.  Under certain
assumptions, a family of $p$-adic integrals with parameter $\lambda$
in a nonarchimedean field $F$ can be expanded in a finite asymptotic
series
\begin{equation}\label{eqn:igusa}
\sum_{a,b,\theta} c_{a,b,\theta} \,
\theta(\lambda) \,|\lambda|^a \, (\op{ord}\lambda)^b\, 
\end{equation}
for some integers $a,b$, multiplicative characters $\theta:F^\times\to
\ring{C}^\times$, and for some constants $c_{a,b,\theta}$.  The
constants are given as principle-valued integrals.  The proof of the
asymptotic series relies on the Igusa zeta function.

In the article, we have presented some results about the asymptotic
behavior of integrals.  This suggests that it may be possible to adapt
the asymptotic series expansion (\ref{eqn:igusa}) to a motivic
context.  A general motivic theory of multiplicative characters is
not yet available.  Nonetheless, an inspection of the proofs in
Langlands suggest that it might be possible to obtain an
expansion similar to (\ref{eqn:igusa}) using far less than a general
theory of multiplicative characters.

\subsection{twisted and nonstandard endoscopy}

In this article, we have transfered the endoscopic matching of smooth
functions for standard endoscopy.  Kottwitz, Shelstad, Waldspurger,
and Ng\^o have developed a corresponding theory for twisted endoscopy
\cite{kottwitz-shelstad}, \cite{waldspurger}, \cite{ngo}.  We expect
that these results can be combined with our methods to obtain transfer
results for twisted endoscopic matching of smooth functions.

\subsection{smooth transfer and the trace formula}

Our paper relies on Waldspurger's proof that the fundamental lemma
implies smooth matching in characteristic zero.  Waldspurger's primary
tool is a global Lie algebra trace formula based on the adelic Poisson
summation formula on the Lie algebra.  Hrushovski-Kazhdan and
Chambert-Loire-Loeser have developed motivic Poisson summation
formulas \cite{hru-kazhdan}, \cite{loeser-cl}.  In fact, Hrushovski
and Kazhdan have applied the motivic Poisson summation formula to show
that if two test functions on two division algebras have matching
orbital integrals, then their Fourier transforms also have matching
orbital integrals~\cite[Theorem 1.1]{hru-kazhdan}.  Their methods and
result are closely related to Waldspurger's smooth matching result.
This suggests the problem of adapting Waldspurger's proof to a motivic
setting.

\subsection{definably indistinguishable reductive groups}

In Section~\ref{sec:classification}, we described the extent to which fixed choices
may be used distinguish various isomorphism classes of reductive
groups.  We may ask further whether {\it groups in the same family
  (1), (2), (3) of Section~\ref{sec:classification} are indistinguishable with respect
  to all (first-order) properties expressible in the Denef-Pas
  language?}

We may also ask {\it to what extent is the harmonic analysis on two groups
in the same family the same?}

We give an example.  Consider the Denef-Pas language with constants in
$\ring{Z}[[t]]$.  Consider the quasi-split definable simply connected group of
typ ${}^eA_n$, ${}^eD_n$, or ${}^eE_6$, with $e=2$ or $3$ as
appropriate, associated with the totally ramified extension $K/\VF$ of
degree $e$:
\[
K = \VF[t]/(x^e - t),\quad \ord (t) = 1.
\]
We make the remarkable observation that the given definable group has
nonisomorphic interpretations as reductive groups over a
nonarchimedean field $F$, simply because nonisomorphic field
extensions of $F$ are obtained by different choices of the uniformizer
$\varpi\in F$ interpreting $t$.  So in particular, for two differerent
quasi-split unitary groups over $F$, splitting over different ramified
quadratic extensions, the Shalika germs are given by identical
constructible formulas, differing over $F$ only by the specialization
of $t$ to different uniformizers $\varpi\in F$.  This suggests that
the harmonic analyses on two groups in the same family are closely
related.


\subsection{definability of nilpotent orbits}

We have carefully avoided the issue of the definability of nilpotent
orbits in this paper.  This has required us to use certain roundabout
constructions such as the set $NF^k_G$.

If we allow a free parameter $N$ that ranges over the nilpotent set,
then the nilpotent orbits are trivially definable by the formula
\[
O(u) = \{u' \mid \exists g\in G\mid u' g = g
u \}.
\]  
However, this leaves open the question of whether nilpotent orbits can
be defined in the Denef-Pas language without the use of parameters.
Diwadkar has obtained partial results on this problem \cite{Did}.  The
results of Barbasch-Moy and DeBacker on the classification of
nilpotent --together with definability for the Moy-Prasad filtration
\cite{cgh} -- reduce this problem to the definability of nilpotent
orbits (or more simply of distinguished nilpotent orbits) in reductive
groups over finite fields.  For some groups such as $SL(n)$, roots of
unity are required to specify the nilpotent orbits.  But in the
adjoint case, the situation is simpler. This leads to the following
question: Let $G=G_{\op{adj}}$ be an adjoint semisimple linear algebraic
group defined over a finite field.  When the characteristic is
sufficiently large, is each nilpotent orbit in the Lie algebra
definable in the first-order language of rings?


\bibliographystyle{alpha}
\bibliography{refs}
(Jan 27, 2015)

\end{document}





