%  Paper: On the fundamental lemma for standard endoscopy
%  Author: Thomas C. Hales
%  Date:   Jan 30, 1995
%  Format: Ams-Tex, amsppt
%  to be published in the Canadian Journal.

\documentclass{amsart}

%Packages
\usepackage{graphicx}
\usepackage{amsfonts}
\usepackage{enumitem}
\usepackage{amscd}
\usepackage{amssymb}
\usepackage{alltt}
\usepackage{amsrefs}

\newtheorem{thm}{}
\newenvironment{cthm}[1]
  {\renewcommand\thethm{\sc #1}\thm}
  {\endthm}


%\magnification=\magstep1   %MM steps depending on magnification are marked %MM
%\input amstex
%\documentstyle{amsppt}
%\voffset=0.5in    %MM use only if unmagnified
%\hoffset=.5in 
%\rightheadtext{Fundamental Lemma for Standard Endoscopy}
%\leftheadtext{Thomas C. Hales}
% Minor corrections June 7, 1994
% Minor revisions Jan 30, 1995 
% principle -> principal Feb 24, 1995

\newcommand\Fr{\text{$F\hskip-.1em r$}}
\newcommand\FrG{\text{$F\hskip-.1em r_G$}}
\newcommand\FrH{\text{$F\hskip-.1em r_H$}}
\newcommand\tr{\text{trace\,}}
\newcommand\s{{\text{sc}}}
\newcommand\adj{{\text{adj}}}
\newcommand\der{{\text{der}}}
\newcommand\Fq{{\Bbb F}_q}
\newcommand\Gal{\text{Gal}}
\newcommand\Ind{\text{Ind}}
\newcommand\bfa{\text{\bf a}}
%\let\sc=\it

\begin{document}

\title[Fundamental Lemma for Standard Endoscopy]{On the fundamental 
		 lemma for standard endoscopy:
		 \\ reduction to unit elements }

\author {Thomas C. Hales}

\address{ University of Michigan}



\begin{abstract}
The fundamental lemma for standard endoscopy 
follows from the matching of unit elements in Hecke algebras. 
A simple form of the stable trace formula,
based on the matching of unit elements, shows the
fundamental lemma to be equivalent to a collection of
character identities.  These character identities are
established by comparing them to a compact-character
expansion of orbital integrals.
\end{abstract}

\thanks{I would like to thank R. Kottwitz for guiding me repeatedly
in the right direction on this project.  I would also like to
thank G. Henniart for explaining the method of Section 5 to me
and L. Clozel for providing an argument at the archimedean places.}

\thanks{Canadian Journal of Mathematics 47.5 (1995): 974--994.}



\maketitle

%\document
%\vsize=.85\vsize
%\parskip=.4\baselineskip

\section{Introduction }%1

L. Clozel has deduced the {\it fundamental lemma} for stable
base change from the corresponding result for the unit elements
of Hecke algebras \cite{Cl2}.  J.-P. Labesse has offered a second proof
of this result \cite{La}. This paper adapts Clozel's argument to standard
endoscopy, thereby reducing this fundamental lemma for reductive
groups to the
unit elements of Hecke algebras.
Unlike stable base change, the matching of units is not
currently known.

One of the main purposes of this paper is to clarify the set of
local conditions that imply the fundamental lemma.  These
local conditions are formalized as {\it local data\/} in Section
4.1.  Local arguments reduce the fundamental lemma to 
groups $G$ with connected anisotropic centers.  For such $G$, 
local data are a collection of finite
character identities between a reductive group $G$ and an endoscopic
group $H$.  If local data exist, and if the fundamental lemma
is known for the Levi factors of $G$, then the fundamental lemma
holds for $G$.

Although the conditions we formulate are local, the only methods
currently known to establish the existence of local data are global.
In the final section we show how a simple stable
trace formula 
may be used to prove the existence of local data.  This argument
assumes the matching of unit elements of Hecke algebras at almost
all places of a global situation that we construct.

\section{Notation and Terminology }%2

Let $F$, $\bar F$, and $\varpi$  denote a $p$-adic
field of characteristic zero, 
a fixed algebraic closure $\bar F$, 
and a uniformizing element
in the ring of integers $O_F$ of $F$.  Background
on the next few definitions can be found in the survey article by
Borel \cite{B}.
If $G$ is a reductive
group, then  $\hat G$ is the connected component of the
complex dual group ${}^L\!G$.  The group $\hat G_\der$ is
its derived group and is not to be confused with the connected
complex dual of $G_\der$.  We write $Z(G)$ for
the center of $G$.  
For any torus $S$, let $X^*(S)$ and 
$X_*(S)$ be the character and cocharacter groups of $S$.
In fact, $X^*(S)$ is defined for any diagonalizable group $S$.
Let $T_d$ be the split subtorus of a maximally
split Cartan subgroup $T_G$ of $G$.
The complex dual to $T_d$ is conventionally denoted $Y$.
Dual to the inclusion $T_d \subset T_G$ is the
canonical projection $\hat T \to Y$.

Let $W'$ is the subgroup of the Weyl group of $T_G$ that is
invariant under the action of the Galois group.
The group $W'$ is denoted ${}_k\!W$ in \cite{B}.
The Hecke algebras in this paper are understood to be
the spherical Hecke algebras, composed of functions bi-invariant
by a fixed hyperspecial maximal compact subgroup.  Initially,
we will use the Hecke algebra of functions of compact support,
but we will also have occasion to use Hecke algebras of
$Z(G)^0(F)$-invariant functions that
are compactly supported modulo the center.  
The Satake transform $f\hat{\phantom{o}}$ of a compactly supported Hecke
function $f$ is 
a $W'$-invariant regular function
on $Y$.  Under the bijection between $Y/W'$ and $\text{Int}({}^LG^0)$-orbits
on $({}^LG^0\rtimes\Fr)_{\text{ss}}$, it is also viewed as a
function on a component of ${}^L\!G$.  (See \cite{M} and \cite{B}.)
Let $c(f,\lambda)$, for $\lambda\in X^*(Y)$, be the coefficient:
$f\hat{\phantom{o}}(t) = \sum_\lambda c(f,\lambda) \lambda(t)$.
Associated with $\lambda\in X^*(Y)$,
there is a compactly supported Hecke function $\phi_\lambda$,
determined by the requirement that $c(\phi_\lambda,\cdot)$
is the characteristic function of the $W'$-orbit of $\lambda$.
The functions $\phi_\lambda$
form a linear basis of the Hecke algebra of compactly supported
functions.  When the group $G$ is defined over $O_F$ and
$G(O_F)$ is hyperspecial, the
Hecke algebra is to be defined relative to the hyperspecial
subgroup $G(O_F)$.

Endoscopic data are attached to a triple $(G,\theta,\omega)$
consisting of a reductive group $G$ defined over $F$, 
an automorphism $\theta$  of $G$ over $F$, and a quasicharacter
$\omega$
of $G(F)$.  For background material on twisted endoscopy, we refer
the reader to the work of
Kottwitz and Shelstad \cite{KS1}.
{\it Standard endoscopy}
refers to data obtained when $\theta$ and $\omega$ are trivial.
This paper treats a slightly larger class, obtained by
requiring only $\theta$ to be trivial; call this
enlarged class {\it standard endoscopy with quasicharacters}.
The fundamental lemma in this paper refers exclusively
to the fundamental lemma for standard endoscopy with 
quasicharacters.  In fact, we also deal exclusively with
the fundamental lemma for strongly $G$-regular semisimple
elements.  As explained in greater detail below, we assume that
the endoscopic data is unramified.  

We let $\Fr$ denote the Frobenius element in the Galois group of the maximal
unramified extension of $F$.  When $G$ splits over an unramified extension,
the Frobenius element acts on the connected dual $\hat G$, and various
functorially related dual objects.  

A quasicharacter is {\it unramified} if its Langlands
parameter in $H^1(W_F,Z(\hat G))$,
the first continuous
cohomology group of the Weil group $W_F$ of $F$ with coefficients
in $Z(\hat G)$, 
is unramified (see \cite{B}).  An unramified cocycle in this group is
determined by its value at the Frobenius element, and
thus unramified quasicharacters correspond bijectively
to the set $Z(\hat G)\rtimes \Fr$ modulo conjugation by $Z(\hat G)$.

Two definitions of transfer factors have been given, one
an extension of the other.  The Langlands-Shelstad factor,
defined for standard endoscopy \cite{LS1}, extends
readily to the slightly larger class of standard endoscopy
with quasicharacters, but not to the fully twisted situation.
The general twisted transfer factor of Kottwitz and Shelstad
\cite{KS1} agrees with the Langlands-Shelstad definition
when it is restricted to standard endoscopy with quasicharacters.
Thus, we may use results from either paper, depending on
which better suits our purposes.

The paper \cite{H2} treats standard endoscopy, not
twisted endoscopy.  Nevertheless, just as the definition
of Langlands and Shelstad extends readily to encompass
quasicharacters, so also the results of \cite{H2} extend.
We will make use of the canonical normalization of
transfer factors, descent for Levi factors, results
on the unramified central character, and other
minor results from that paper.

The definition of unramified endoscopic group should be
amended as follows to include standard endoscopy with
quasicharacters.  Let $(H,{\mathcal H},s,\xi)$ be endoscopic
data for $G$ in the sense of \cite{KS1}.  We say that
$(H,{\mathcal H},s,\xi)$ are {\it unramified} endoscopic data
for $(G,\omega)$ if
\begin{enumerate}[label=(\arabic*)]
  \item $G$ is defined over $O_F$, $G(O_F)$ is hyperspecial,
and $G$ is unramified,

\item $H$ is defined over $O_F$, $H(O_F)$ is hyperspecial,
and $H$ is unramified,

\item ${\mathcal H}$ is the $L$-group of $H$,

\item The embedding $\xi$ descends to an unramified
field extension $E/F$, and

\item  $\omega$ is an unramified quasicharacter of
$G(F)$.
\end{enumerate}

The changes to the definition from the paper \cite{H2}
are minor.  The reference to \cite{LS1} has been changed
to \cite{KS1}, the quasicharacter $\omega$ has been added,
along with Condition 5.  
An unnecessary hypothesis,
 the finiteness of the extension $E/F$, has been
eliminated from Condition 4.
Finally, a redundant condition 
\cite{H2,6-Condition 4} has been eliminated.

The map of Hecke algebras (of compactly supported functions) 
associated with an embedding $\xi$
of $L$-groups will be denoted $f\mapsto b_\xi(f)$, or simply
$f\mapsto b(f)$ when the context is clear.
Let $\Phi(\gamma_G,f)$ denote the orbital integral of $f$
on the conjugacy class of $\gamma_G$ for the
quasicharacter $\omega$.  If $\gamma_G$ is strongly regular
with centralizer $T$, this orbital integral is defined
as 
$$\Phi(\gamma_G,f) = \int_{T(F)\backslash G(F)}
   f(g^{-1}\gamma_G g)\omega(g)\,dg,$$
where $dg$ is an invariant measure on $T(F)\backslash G(F)$.
Similary, let $\Phi^{\text{st}}(\gamma_H,b(f))$ be
the stable orbital integral of $b(f)$. It is 
defined as
$$\Phi^{\text{st}}(\gamma_H,b(f)) =
\int_{(T_H\backslash H)(F)} b(f)(h^{-1}\gamma_H h) dh,$$
where $T_H$ is the centralizer of $\gamma_H$, and
$dh$ is an invariant measure on $(T_H\backslash H)(F)$
obtained from an invariant form on $T_H\backslash H$.
The distinction between $(T_H\backslash H)(F)$ and
$T_H(F)\backslash H(F)$ is essential.
We take the measures on $G$ and $H$ to be compatibly normalized.
We let $\Lambda=0$
denote the identity of the fundamental lemma.  Namely, set
$$\Lambda(\gamma_H,f) = \sum_{\gamma_G}\Delta(\gamma_H,\gamma_G)\Phi(\gamma_G,f)
 - \Phi^{\text{st}}(\gamma_H,b(f)),$$
 where $\Delta$ is the transfer factor of Kottwitz and Shelstad with
 the canonical normalization given in \cite{H2,7}.  
 Let $H(F)_{G\text{-reg}}$ denote the set of strongly $G$-regular
 elements in $H(F)$ (see \cite{LS1}).  The fundamental
 lemma conjecturally asserts that $\Lambda(\gamma_H,f) = 0$,
 for all $\gamma_H\in H(F)_{G\text{-reg}}$ and all compactly supported
 Hecke functions $f$.  

Next, we elaborate on
what it means for the fundamental lemma to hold for $Z(G)^0(F)$-invariant
Hecke functions.
For each basis function $\phi_\lambda$ of the Hecke algebra, we
define a $Z(G)^0(F)$-extension $\phi'_\lambda$ by the the condition
$\phi'_\lambda(\gamma) = \phi_\lambda(z\gamma)$ if there
exists $z\in Z(G)^0(F)$ such that
$z\gamma\in \text{supp}(\phi_\lambda)$, and $\phi'_\lambda(\gamma) = 0$
otherwise.  Lemma 3.2 will prove that each unramified character
of $G(F)$
is constant on the support of $\phi_\lambda$.  Thus, the
ambiguity of the expression $z\gamma$ is by an element of $Z(G)^0(F)$
lying in the kernel of all unramified characters.  Such an
element belongs to $G(O_F) \cap Z(G)^0(F)$, so that $\phi'_\lambda$
is well-defined.   
The functions $\phi'_\lambda$ span the space of $Z(G)^0(F)$-invariant
Hecke functions that are compactly supported modulo the center. 
Equivalently, $\phi'_\lambda$ is given by
$$\sum_{z\in Z(G)^0(F)/Z(G)^0(O_F)} R_z\phi_\lambda,$$
where $R_z$ is defined by $R_zf(g) = f(zg)$.
Set $b'_\xi(\phi'_\lambda) = \sum b_\xi(R_z\phi_\lambda)$,
where the locally finite sum extends over $z\in Z(G)^0(F)/Z(G)^0(O_F)$.  
Use $b'$ to extend the notion
of the fundamental lemma.  
Write $\Lambda'(\gamma_H,f)$, when $f$ is
 a $Z(G)^0(F)$-invariant Hecke function, for the expression obtained
 from $\Lambda(\gamma_H,\cdot)$
 by replacing  $b$ with $b'$.


\section{Routine Facts and Routine Reductions }%3


\begin{cthm}{Lemma 3.1}  Every quasicharacter
on $G$ is constant on each stable conjugacy
class.
\end{cthm}

\begin{proof}  By a $z$-extension argument \cite{Ko1,3.1.2},
we may assume that the derived group of $G$ is simply connected.
If $\gamma$ and $\gamma^g$ are stably conjugate, then
$\gamma^{-1}\gamma^g$ lies in both $G_{\der}(\bar F)$ and
$G(F)$, and hence also in $G_{\s}(F)$.  Any quasicharacter
vanishes on $G_{\s}(F)$, and from this the result follows.\end{proof}


\begin{cthm}{Lemma 3.2}
Every unramified character is constant on the support
of each Hecke function $\phi_\lambda$.
\end{cthm}

\begin{proof}  Fix an unramified character $\theta$.
The function
$\gamma\mapsto\tilde\phi_\lambda (\gamma)$, defined to be $\phi_\lambda(\gamma)$
when $\theta(\gamma) = \theta(\varpi^\lambda)$ and zero otherwise,
belongs to the Hecke algebra and has the same orbital integrals
on the maximally split torus as $\phi_\lambda$.  
If two functions in the Hecke algebra have
equal orbital integrals on this torus,
then the functions are equal.  Hence $\phi_\lambda
=\tilde\phi_\lambda$.  Consequently, the unramified character
$\theta$ is constant on the support of $\phi_\lambda$.
\end{proof}

The rest of the section carries out some routine
reductions that simplify the exposition in later
sections.  For instance, nothing is lost by assuming
that $H$ is an elliptic endoscopic group, because
the fundamental lemma for a nonelliptic endoscopic
group is equivalent to a fundamental lemma for
an elliptic endoscopic group obtained by descent (see \cite{H2}, \cite{LS2}).

\begin{cthm}{Lemma 3.3}  If the fundamental lemma holds
for one choice of embedding $\xi$, then it holds for all
choices of embeddings.
\end{cthm}

\begin{proof}
Compare two embeddings of $L$-groups $\xi$ and $\xi'$.  Because
of the standing assumption that the endoscopic data is unramified,
both $\xi$ and $\xi'$ are unramified.
Write $\Delta^\xi(\gamma_H,\gamma_G)$ for
the transfer factor associated with the embedding $\xi$.
The ratio of the transfer factors depends only on the
term $\Delta_{III_2}$, defined in \cite{LS1}.  The ratio
has the form
$$\Delta^{\xi'}(\gamma_H,\gamma_G)/\Delta^{\xi}(\gamma_H,\gamma_G) =
   \langle a_{\xi'}/a_\xi,\gamma\rangle,$$
where $\gamma$, inside a Cartan subgroup $T=T_G$, is an image
of $\gamma_G$ in the quasisplit form of $G$, and $a_\xi$ and
$a_{\xi'}$ are cocycles, defined in \cite{LS1}, in the Weil cohomology
group $H^1(W_F,\hat T)$.  According to Langlands' theory of
abelian groups, this cohomology group pairs canonically
with $T(F)$.

The embeddings $\xi$, $\xi':{}^L\!H\to {}^L\!G$ have the form $\xi(1\rtimes \Fr) =
x_1m_0\rtimes \Fr$ and $\xi'(1\rtimes \Fr) = x_1'm_0\rtimes \Fr$,
for some elements $x_1$, $x_1'$, and  $m_0\in \hat G$.  Both $x_1m_0\rtimes\Fr$
and $x_1'm_0\rtimes \Fr$ fix a splitting in $\hat H$, and this
is possible only if $x'_1/x_1$ belongs to $Z(\hat H)$.  It follows by
consulting the definition of $a_\xi$ and $a_{\xi'}$ in \cite{LS1},
discussed further in \cite{H2},
that $(x_1'/x_1)a_\xi(w) = a_{\xi'}(w)$, for all elements $w$ in the
Weil group $W_F$ lying over the Frobenius element $\Fr$.  If
$\theta$ is the unramified character on $H(F)$ whose parameter
in $Z(\hat H)\rtimes \Fr$ is $(x_1'/x_1)\rtimes \Fr$, then we
conclude that the ratio of transfer factors 
$\langle a_{\xi'}/a_\xi,\gamma\rangle$ simplifies to $\theta(\gamma)$.

Now assume that the fundamental lemma holds for the
embedding $\xi$.  Then
$$\sum_{\gamma_G} \Delta^{\xi'}(\gamma_H,\gamma_G)
  \Phi(\gamma_G,f) = \theta(\gamma_H)
  \sum_{\gamma_G}\Delta^\xi(\gamma_H,\gamma_G)
  \Phi(\gamma_G,f) = \theta(\gamma_H)\Phi^{\text{st}}(\gamma_H,b_\xi(f)).$$
Lemma 3.3 will then follow from the identity
\begin{equation}\tag{*}
 \theta(\gamma_H)\Phi^{\text{st}}(\gamma_H,b_\xi(f)) =
  \Phi^{\text{st}}(\gamma_H,b_{\xi'}(f)),
\end{equation}
  for all $\gamma_H$ in $H(F)_{G\text{-reg}}$.
There are three separate cases to consider in the
proof of this identity.

Case 1.  Suppose that the endoscopic group $H$
is an elliptic torus of $G$.  We may identify $\hat H$ with 
the complex dual of $T_G$ in $\hat G$.
An ordered pair $(t,z)$ will
represent the element $tz$ of $\hat G$ 
according to the decomposition $\hat G =\hat G_\der Z(\hat G)^0$.
With a slight shift in notation,
the embedding
$\xi:{}^L\!H\to{}^L\!G$ takes the form $(t,z)\rtimes\FrH\mapsto (tx_0m_0,zx)\rtimes \FrG$,
where $m_0$ lies in the normalizer of $\hat H$ 
and is independent of the embedding.
The subscripts $H$ and $G$ have been added to the Frobenius element to
distinguish the $L$-actions coming from $H$ and $G$.  
Set $\hat T_\der=\hat G_\der\cap \hat T_G$.
Since $\hat T_\der$ with its $H$-induced Frobenius action
is dual to an unramified elliptic torus, 
$tx_0m_0$ is $\FrG$-conjugate to an element $\rho\in\hat T_\der$
that is independent of $t$ and $x_0$.  In other words, 
the map $\hat T_\der \to
\hat T_\der$ given by $t\mapsto t^{-1}m_0\FrG(t)m_0^{-1}
= t^{-1}\FrH(t)$ is surjective.
We obtain the conjugate element $(\rho,zx)\rtimes\FrG\in \hat T\rtimes\FrG$.

Recall that $Y$ is the complex dual to the split torus $T_d$.
If $\lambda\in X^*(Y)$, then it pulls back to a character,
also denoted $\lambda$, in $X^*(\hat T)$
satisfying $\FrG(\lambda) = \lambda$.  
By definition, the Satake transform of $b_\xi(\phi_\lambda)$,
viewed as a function on $\hat H\rtimes \FrH$,
evaluated at $(t,z)\rtimes\FrH$ is $\sum_{w\in W'} w\cdot\lambda(\rho,zx)$,
where, as usual,  $W'$ is the subgroup of the Weyl group fixed by
the Frobenius element $\FrG$. 
The 
group $W'$ acts trivially on $Z(\hat G)^0$, so that this expression simplifies
to $c_\lambda\,\lambda(1,zx)$, where $c_\lambda = \sum_{w\in W'} w\cdot \lambda(\rho,1)$.
The ambiguity by $\hat T_\der\cap Z(\hat G)^0$ in the decomposition 
$\hat T=\hat T_\der Z(\hat G)^0$
may be used to show that $c_\lambda=0$, unless the restriction of $\lambda$ to 
$\hat T_\der\cap Z(\hat G)^0$ is trivial.  Hence, for nonvanishing terms, 
the character $\lambda$
descends to a character on $Z(\hat G)^0/(Z(\hat G)^0\cap \hat T_\der)$,
which is the complex dual to $Z(G)^0$.
The
actions of $\FrG$ and $\FrH$ coincide on $Z(\hat G)^0$.  Thus, the invariance
$\lambda(\FrG(z))=\lambda(z)$ gives $\lambda(\FrH(z))=\lambda(z)$, for $z\in Z(\hat G)^0$.  The complex torus
$Y_H$, defined as the
 dual to the maximally split subtorus of $H$,
is a quotient of the complex dual to $Z(G)^0$, 
because $H$ is elliptic in $G$.
This shows that $\lambda$ descends to a character $\bar \lambda$ in $X^*(Y_H)$.
The Satake transform of $b_\xi(\phi_\lambda)$ becomes 
$\bar z\mapsto c_\lambda\,\bar\lambda(\bar z)\bar\lambda(\bar x)$,
where $\bar z$ and $\bar x$ are the images of $(1,z)$ and $(1,x)$ in $Y_H$.
The Satake transform on a torus $H$ is essentially trivial; we find that
$b_\xi(\phi_\lambda)$ is the characteristic function of 
the double coset $H(O_F)\varpi^{\bar \lambda}H(O_F)$
times the constant $c_\lambda\bar \lambda(\bar x)$.  
For $\gamma=\varpi^{\bar \lambda}$,
the identity (*) becomes
$$c_\lambda\,\theta(\varpi^{\bar\lambda}) \bar\lambda(\bar x) = c_\lambda\,\bar\lambda(\bar x'),$$
where $\bar x'$ corresponds to $\xi'$.  Recall that $\theta$ is the unramified
character defined by comparing the embeddings $\xi$ and $\xi'$; they differ
by an element $(1,x'/x)\in Z(\hat G)^0$.  Thus,
$\theta(\varpi^{\bar\lambda})=\bar\lambda(\bar x'/\bar x)$.
The identity (*) is now evident.
\smallskip

Case 2.  Suppose that the element $\gamma_H$ lies
in the maximally split Cartan subgroup $T_H$ of $H$ 
and that $H\ne T_H$.  In this case $T_H$ is not elliptic,
and a routine descent argument (for instance, \cite{H2,9}) reduces
this case to the previous case.
\smallskip

Case 3.  Suppose that the element $\gamma_H$ does
not lie in the maximally split Cartan subgroup.  
Fix a function $f$ of the Hecke algebra, and write
$b_\xi(f)$ as a finite linear combination $\sum_{\lambda}c_{\xi,\lambda}
\phi^H_\lambda$, where $\phi^H_\lambda$ is the basis function on
the group $H$
corresponding to $\lambda\in X^*(Y_H)$,
analogous to the function $\phi_\lambda$ on $G$.  Do the same
for $b_{\xi'}(f)$.  By Lemma 3.2 an unramified
character $\theta$ is constant on the support of $\phi^H_\lambda$.


Recall that $\theta$ is also constant on each
stable conjugacy class (Lemma 3.1).
Thus, (*) may be rewritten, when $\gamma_H$ belongs to the maximally
split torus (Case 2), as the collection of identities 
$\theta(\varpi^\lambda) c_{\xi,\lambda}  = c_{\xi',\lambda}$, for
all $\lambda$.  Select $\lambda_0$ such that $\theta(\gamma_H)
 = \theta(\varpi^{\lambda_0})$.
Then (*) holds generally,
because
\begin{align*}
\theta(\gamma_H)\Phi^{\text{st}}(\gamma_H,b_\xi(f)) &= 
\sum_{\lambda}\theta(\gamma_H)c_{\xi,\lambda}
	\Phi^{\text{st}}(\gamma_H,\phi^H_\lambda)
 = \sum_\lambda
\theta(\varpi^{\lambda_0}) 
	c_{\xi,\lambda}\Phi^{\text{st}}(\gamma_H,\phi^H_\lambda)\\
 &= \sum c_{\xi',\lambda}\Phi^{\text{st}}(\gamma_H,\phi^H_\lambda) =
  \Phi^{\text{st}}(\gamma_H,b_{\xi'}(f)).
\end{align*}
\end{proof}
%

Let $\theta$ be an unramified character of $G(F)$ with parameter $t\rtimes \Fr$
in $Z(\hat G)\rtimes \Fr$.  The canonical inclusion of $Z(\hat G)\to Z(\hat H)$
leads to a character $\theta_H$ on $H(F)$ with the same parameter, now
in $Z(\hat H)\rtimes \Fr$.

\begin{cthm}{Lemma 3.4} With $\theta_H$ constructed as above, $\theta_H$ is
constant on the support of $b(\phi_\lambda)$.  Moreover,
$\theta(\text{supp}(\phi_\lambda)) = \theta_H(\text{supp}(b(\phi_\lambda)))$.
\end{cthm}

\begin{proof}
Certainly $b(\phi_\lambda)$ is determined by its values on the elements
$\varpi^{\lambda'}$, for $\lambda'\in X^*(Y_H)$, and $\theta_H$ is
constant on every double coset of $H(O_F)$.  Thus, we consider 
$\theta_H(\varpi^{\lambda'})$, for $\varpi^{\lambda'}\in\text{supp}(b(\phi_\lambda))$.
By a descent argument we reduce to the case that $H$ is an elliptic torus
of $G$.
The argument of Lemma 3.3 (Case 1) shows that $b(\phi_\lambda)$ is
supported on the double coset of $\varpi^{\bar\lambda}$.
The character $\bar\lambda\in X^*(Y_H)$
gives a character on $\widehat{Z(G)^0}$ and hence a character
$\lambda'$ on $Z(\hat G)$, by dualizing the inclusion
$T_d^H\subset Z(G)^0$, where $T_d^H$ is the maximally split
torus in the elliptic Cartan subgroup $H$.  One then easily finds that
$\theta_H(\varpi^{\bar\lambda}) = \langle t,\lambda'\rangle 
= \theta(\varpi^\lambda)$.
\end{proof}

\begin{cthm}{Lemma 3.5}  The fundamental lemma is true for a reductive
group if it is true for the $Z(G)^0(F)$-invariant Hecke functions
on the group.
\end{cthm}

\begin{proof}  
By Lemma 3.3, we may pick whatever embedding is the most
convenient.  As the proof of Lemma 3.6 will explain in
greater detail, there exists an embedding for which
the transfer factor is invariant by the connected center:
$\Delta(z\gamma_H,z\gamma_G) = \Delta(\gamma_H,\gamma_G)$, for
$z\in Z(G)^0(F)$.
We refer the reader to the definition
of $b'(\phi'_\lambda)$ in Section 2.
We have
\begin{equation}\tag{*}
\Lambda'(\gamma_H,\phi'_\lambda) = \sum_z \Lambda(\gamma_H,R_z\phi_\lambda) =
\sum_z \Lambda(z\gamma_H,\phi_\lambda).
\end{equation}
The second equality makes use of the compatibility of the
fundamental lemma with translations by $Z(G)^0(F)$, as proved in
\cite{H2,11}.  (With a different choice of embedding, a character
of $Z(G)^0(F)$ would appear in the right-hand term.)
The hypothesis of the lemma means that $\Lambda'(\gamma_H,\phi'_\lambda)=0$,
for all $\lambda$.  By translating $\gamma_H$ by a central element
of $G$, 
we may assume that
$\gamma_G$ lies in the same coset of ${}^0\!G$ as the support of $\phi_\lambda$
(see Lemma 3.2),
where ${}^0\!G$ is the intersection of the kernels 
of all unramified characters on $G$.  
Then the stable
conjugacy class of $z\gamma_G$ does not meet the support of $\phi_\lambda$
unless $z$ belongs to $Z(G)^0(O_F)$.

By Lemma 3.4, the stable conjugacy class of $z\gamma_H$ 
does not meet the support of $b(\phi_\lambda)$
unless $z$ belongs to $Z(G)^0(O_F)$.  Consequently,
(*) becomes $\Lambda(\gamma_H,\phi_\lambda)=0$, and the fundamental
lemma holds.
\end{proof}


\begin{cthm}{Lemma 3.6}  If the fundamental lemma holds
whenever $G$ is an adjoint group, then it holds in
general.  \end{cthm}

\begin{proof}
By Lemma 3.3, we may pick whatever embedding is the most
convenient.  For our purposes, it is best to pick an embedding
for which the image in ${}^L\!G$ of the Frobenius element $\Fr$
lies in $\hat G_\der\rtimes\Fr$.  Such embeddings
exist (see for example \cite{H2,6.1}).  By adjusting the
choice of the element $s$, which is
a given of the unramified endoscopic data, 
by a central element in $\hat G$,
we may assume that $s$ lies in the derived group of $\hat G$.
With these choices the endoscopic data $(H,{\mathcal H},s,\xi)$ easily
lead to endoscopic data $(\bar H,{}^L\!\bar H,s,\bar\xi)$
of the semisimple group $\bar G$ dual to $\hat G_\der$.

As an initial step toward the proof, let us deduce the fundamental
lemma associated with the data  $(H,{\mathcal H},s,\xi)$
from the fundamental lemma associated with the data
$(\bar H,{}^L\!\bar H,s,\bar\xi)$.
There is nothing difficult here, but a number of small facts
must be checked.  The sequence $1\to \hat G_\der\to\hat G
\to \hat G/\hat G_\der\to 1$ is dual to the sequence
$1\to Z(G)^0 \to G \to G/Z(G)^0\to 1$.  In fact, in the
sequence $1\to A\to G\to \bar G\to 1$ dual to 
$1\to  \hat G_\der\to\hat G\to\hat G/\hat G_\der\to 1$,
we know that 
$A$ is central and connected, and $\bar G$ is semisimple, and these properties
characterize the sequence $1\to Z(G)^0 \to G \to G/Z(G)^0\to 1$.
Thus $\bar G$ is isomorphic to the quotient $G/Z(G)^0$.
Similarly, $\bar H$ is isomorphic to $H/Z(G)^0$, under the canonical
embedding of $Z(G)^0$ into $H$.  Thus, we may project a pair $(\gamma_H,\gamma_G)$
in $H\times G$ to a pair $(\bar \gamma_H,\bar \gamma_G)$ in $\bar H\times
\bar G$.  Because of the choice of $\xi$ made above, the cocycle
$a_\xi$ restricts to a character that is trivial on $Z(G)^0$.
(See \cite{H2,11}).  It is then a mere formality to check that
$\Delta^\xi_{H,G}(\gamma_H,\gamma_G) = \Delta^{\bar \xi}
_{\bar H,\bar G}(\bar \gamma_H,\bar \gamma_G)$.  Subscripts
to $\Delta$ have been added to distinguish transfer factors on
different groups.

The image of a stable conjugacy class in $G$ is a stable conjugacy
class in $\bar G$, and  likewise for $H$.  This simple fact follows
from the observation that the image of $G(F)$ in $\bar G(F)$ is
the kernel of a collection of quasicharacters, and that quasicharacters
are constant on stable conjugacy classes (Lemma 3.1).  By pulling
Hecke functions on $\bar G$ and $\bar H$ back to $G$ and $H$, the
functions are only compactly supported modulo the center. 
But Lemma 3.5 states
that this does not matter.  In this way the fundamental lemma
is displaced to $\bar G$ and $\bar H$.

Now we assume that $G$ is semisimple and reduce to the case where $G$
is adjoint.  The endoscopic data $(H,{}^L\!H,s,\xi)$ easily lift
to data for the simply connected cover $\hat G_{\s}$
It is essential to allow
the the quasicharacters $\omega$ to be nontrivial
at this point, otherwise the lift would not always exist.  The character
$\theta$ on the center of $G$, defined by 
$\theta(z) =\Delta(z\gamma_H,z\gamma_G)/\Delta(\gamma_H,\gamma_G)$,
is trivial.  (The existence of such  a character
$\theta$ is proved in \cite{LS2}).  It
is trivial because it is the restriction, 
of an unramified character on the maximally split torus,
to the compact (finite) group of $F$-rational points in $Z(G)$
(see \cite{H2,11}).
There is a canonical injection from the Hecke algebra on $G$
to that on $G_{\text{adj}}$, since the center of $G$
is contained in $G(O_F)$.  There
is a corresponding injection on the Hecke algebra of $H$.  The
remaining details are similar to those already given for the reduction
to the semisimple case.\end{proof}



Not only may we assume that $G$ is adjoint, we may also assume
that it is simple over the algebraic closure of $F$.  Orbital
integrals, endoscopy, transfer factors, Hecke algebras, and
the map $b$ of Hecke algebras are all compatible with products
and are all compatible with the restriction of scalars.

It is convenient
to reduce to standard endoscopy, for which the unramified quasicharacter
$\omega$ is trivial.  The next lemma describes the
group to be used for this.
The reductive groups $G$ constructed in the next lemma will be called the
{\it basic cases}.

\begin{cthm}{Lemma 3.7}  For any simple 
unramified adjoint group $G_{\text{adj}}$,
there is an unramified reductive group $G$
(whose adjoint group is $G_{\text{adj}}$) with the following properties:
\begin{enumerate}[label=(\arabic*)]
\item  The center of $G$ is connected and anisotropic;
\item The image of $G(F)$ in the adjoint group is equal to the
kernel of $\omega$;
\end{enumerate}
\end{cthm}

\begin{proof}  If we construct an unramified 
group $G_1$ whose center is connected,
whose derived group is $G_\s$, and whose image in $G_\adj(F)$ is the
kernel of $\omega$, then we may define $G$ to be the quotient of $G_1$
by the split component of the center.  

An embedding $G_{\s}\subset G_1$
is determined by a homomorphism $\alpha$ from
$X^*(Z(G_1))$ {\it onto}  $X^*(Z(G_{\s}))$.  The center of $G_1$ is
connected if $X^*(Z(G_1))$ is torsion free.  
There is an exact sequence
$$1\to \text{Im}(G_1(F)\to G_\adj(F))
	\backslash G_\adj(F) \to H^1(F,Z(G_1))\to H^1(F,D),$$
where $D$ is the torus $G_1/G_\s$; see \cite{Ko3,1.5}.  The
group $X_*(D)$ may be identified with the lattice dual to the
kernel of $\alpha$.  By Tate-Nakayama duality, the quotient of
$G_\adj(F)$ by $G_1(F)$ is identified with the subgroup
of $H^{-1}(F,X_*(Z(G_1)))$ of elements whose image in 
$H^{-1}(F,X_*(D))$ is trivial.
We list the
homomorphisms $\alpha:X^*(Z(G_1))\to X^*(Z(G_{\s}))$ and the action of
the Frobenius on $X^*(Z(G_1))$, and leave the verification of the
properties of the lemma to the reader.

When $\omega$ is trivial, we take $G_1$ to be any unramified $z$-extension
of $G_{\text{adj}}$.  Every unramified quasicharacter is trivial on
adjoint groups of types ${}^2A_{2k}$, ${}^3D_4$, ${}^2E_6$, $E_8$, $F_4$, and
$G_2$. 

\bigskip
{Case 1}. (split, but not of type $D_{2\ell}$)
$\omega$ has order $\ell$ dividing $n$, 
$X^*(Z(G_{\text{sc}})) = {\Bbb Z}/n{\Bbb Z}$,
and $\Fr$ acts trivially on ${\Bbb Z}/n{\Bbb Z}$.
$$1\leftarrow {\Bbb Z}/n{\Bbb Z} \overset\alpha\to\leftarrow
 {\Bbb Z}^{\ell+1}/(n,1,\dots,1){\Bbb Z}.$$
The homomorphism $\alpha$ is projection onto the first factor, and
$\Fr(y,x_1,\dots,x_\ell)
 = (y,x_2,\dots,x_\ell,x_1)$ on $X^*(Z(G_1))$.

\bigskip
{Case 2}. (${}^2A_{2k-1}$,
  ${}^2D_{2\ell+1}$) $\omega$ has order 2, $X^*(Z(G_{\text{sc}})) = {\Bbb Z}/2k{\Bbb Z}
$,
and $\Fr$ acts by $\Fr(x) = -x$ on $X^*(Z(G_{\text{sc}}))$.
$$1\leftarrow {\Bbb Z}/2k{\Bbb Z} \overset\alpha\to\leftarrow {\Bbb Z}.$$
Let $\Fr$ act by
$\Fr(x) = -x$ on ${\Bbb Z}$.

\bigskip
{Case 3}. ($D_{2\ell}$, ${}^2D_{2\ell}$) $\omega$ has order 2,
$X^*(Z(G_{\text{sc}})) = {\Bbb Z}/2{\Bbb Z}\oplus {\Bbb Z}/2{\Bbb Z}$,
and $\Fr$ acts by order 1 or 2.  If $\Fr$ acts trivially, then
we proceed as in Case 2 with $k=1$.  Now assume that
$\Fr(x,y) = (y,x)$ on $X^*(Z(G_{\text{sc}}))$.  
We induce the data from Case 1.  
Let $X^*(Z(G_1)) = {\Bbb Z}^6/((2,1,1,0,0,0){\Bbb Z} +
(0,0,0,2,1,1){\Bbb Z})$, $\alpha(y,x_1,x_2,y',x_1',x_2') = (y,y')$,
$\Fr(y,x_1,x_2,y',x_1',x_2') = (y',x_1',x_2',y,x_2,x_1)$, and so
forth.

Cases 1, 2, and 3 cover the only possibilities that arise when
$\omega$ is nontrivial.
\end{proof}



\begin{cthm}{Lemma 3.8}  If the fundamental lemma holds for
the basic cases $G$,
then it holds in general.
\end{cthm}

\begin{proof}  The fundamental lemma holds for elementary
reasons for functions on $G_{\text{adj}}$ whose support
does not meet the kernel of the quasicharacter $\omega$.  
To see this, consider
a strongly regular semisimple element $\gamma$ that is not
in this kernel.  By the $\omega$-invariance of the transfer
factor \cite{KS1}, making the change of variables $g\mapsto\gamma g$
in the integrand $f(g^{-1}\gamma g)\omega(g)
    = f((\gamma g)^{-1}\gamma (\gamma g))\omega(g)$
of the orbital integral,
we find that the orbital integral
is $\omega(\gamma)$ times itself and is hence zero.
The transfer $b(f)$ of a function $f$ that does not meet
the kernel of $\omega$ is zero.  
To see this using Lemma 3.4, we have the trivial character $\omega_H$
on $H(F)$, and, if the support of $b(\phi_\lambda)$ is nonempty,
then 
$1 = \omega_H(\text{supp}(b(\phi_\lambda))) = \omega(\text{supp}(\phi_\lambda))$.
Hence, the fundamental lemma is
true of such functions.  Since the kernel of $\omega$ is bi-invariant
by the hyperspecial maximal compact subgroup, we may assume now
that the support of $f$ lies in the kernel of $\omega$.

Waldspurger has explained the rest of this reduction \cite{W,3.1.2}.  He
gives the argument for $GL(n)$, but the argument is general.
The orbital integral on the adjoint group is equal 
to a $\kappa$-orbital integral of a 
Hecke function on $G$.  This lift to $G$ is compatible with endoscopy.
\end{proof}


\section{Local Data}%4

We are now ready to undertake the nontrivial part of the local argument
needed for the fundamental lemma. 
Local data, discussed in this section, are the local character
identities,
 pertaining to the fundamental lemma, that can be obtained
from the trace formula.  Throughout this section, we assume
that $G$ is a basic case.
 Let $R(G)$ denote the set of 
 irreducible admissible representations of
 $G(F)$ with an Iwahori fixed vector.  Let $R(H)$ denote the set 
 of irreducible admissible representations of $H(F)$ with an
 Iwahori fixed vector.

 \begin{cthm}{Definition 4.1} {\it Local data\/}  for $(G,H)$ consist of
 the data (a), (b), and (c) 
subject to Conditions 1 and 2 below.
\begin{enumerate}[label=(\alph*)]
 \item An indexing set $I$ (possibly infinite) 
 \item A collection of complex constants $a_i^G(\pi)$ for $i\in I$
 and $\pi\in R(G)$
 \item A collection of complex constants $a_i^H(\pi')$ for
 $i\in I$ and $\pi'\in R(H)$
\end{enumerate}
\begin{enumerate}[label=(\arabic*)]
 \item
 For $i$ fixed, the constants $a_i^G(\pi)$ and
 $a_i^H(\pi')$ are zero for all but finitely many $\pi$ and $\pi'$.
 \item For every function $f$ in the Hecke algebra of $G$,
 the following are equivalent:
\begin{enumerate}[label=(\Alph*)]
 \item for all $i\in I$, we have $\sum_\pi a^G_i(\pi)\tr\,\pi(f) =
	\sum_{\pi'} a_i^H(\pi')\tr\,\pi'(b(f))$, and
\item for all $\gamma_H\in H(F)_{G\text{-reg}}$, we have $\Lambda(\gamma_H,f)=0$.
\end{enumerate}
\end{enumerate}
\end{cthm}

The essential part of the definition is Condition 2.  Roughly, local
data indicate how to translate the fundamental lemma
into a collection of character identities.
Nothing would change if $R(G)$ and $R(H)$ were
taken to be spherical representations, since $f$ and
$b(f)$ belong to Hecke algebras.



The justification of local data 
comes from the following theorem, which
will be proved in the rest of this section.
Langlands and Shelstad have shown how to obtain
an endoscopic group of a Levi factor by descent from
an endoscopic group of $G$ (see \cite{LS2}).

\begin{cthm}{Theorem 4.2}  Let $G$ be a basic case.
Suppose that there exist local data
for $(G,H)$.  Suppose that the fundamental lemma holds for all proper
Levi factors of $G$ for the endoscopic groups obtained
by descent from $H$.  Then the fundamental lemma holds
for the endoscopic group $H$ of $G$.
\end{cthm}

Let local data be given.  It consists of an indexing
set $I$, and functions $a_i^G$ and $a_i^H$, for $i\in I$.  In
light of the equivalence expressed by Condition 2, the fundamental
lemma holds if the identities
$\sum_\pi a^G_i(\pi)\tr\,\pi(f) =
   \sum_{\pi'} a_i^H(\pi')\tr\,\pi'(b(f))$
hold for all $i\in I$.  
To show this, we fix our attention
on a single identity, for some $i\in I$, and drop $i$ from the notation.
By elementary
properties of spherical functions, for each $\pi'\in R(H)$, 
there exist
 $\pi\in R(G)$ and a parameter $s\in Y$ such that
$$ \text{trace}\,\pi'(b(f)) = \text{trace}\,\pi(f)= f\hat{\phantom{o}}(s).$$
This allows us to rewrite the desired identity as 
$0=A(f)$, where $A(f)$ is a finite sum of the
form $A(f) = \sum a(s) f\hat{\phantom{o}}(s)$, for
appropriate functions $a(s)$ on $Y$.  If we show that $A$ is
zero (Lemma 4.4), then the theorem is proved.

The compact trace, denoted $\text{trace}_c\,\pi(f)$, is defined
in \cite{Cl1}; in brief, it is equal to $\text{trace}\,\pi(1_cf)$, where
$1_c$ is the characteristic function of the compact elements in $G(F)$.
Similarly, we form $\text{trace}_c\,\pi'$, for
$\pi'\in R(H)$.  


\begin{cthm}{Proposition 4.3}  
Under the hypotheses of Theorem 4.2, the linear functional
$f\mapsto A(f)$ on the Hecke algebra is a finite linear combination of
the linear functionals $\Lambda(\gamma_H,\cdot)$, for 
$\gamma_H\in H(F)_{G\text{-reg}}$.
There is also an expression for $A(f)$
as a finite linear combination of linear functionals of the form
$$f\mapsto \text{trace}_c\,\pi(f)\ \ \text{and}\ \ 
  f\mapsto \text{trace}_c\,\pi'(b(f)),\ \ \text{for} \ \
  \pi\in R(G)\ \ \text{and}\ \ \pi'\in R(H).$$
\end{cthm}

\begin{proof}  
By the hypothesis on Levi factors in Theorem 4.2, we
may assume that $\Lambda(\gamma_H,f)=0$, if $\gamma_H$ is not
elliptic.  Thus, the expansion to be produced in Proposition 4.3 will
involve functionals $\Lambda(\gamma_H,\cdot)$, for $\gamma_H$ elliptic.
By the Howe conjecture, proved by Clozel, applied to
both $G$ and $H$, we find for any elliptic
element $\gamma_H\in H(F)_{G\text{-reg}}$ that $\Lambda(\gamma_H,f)$ has an
expansion in terms of compact traces of the sort given
in the proposition.  (Details of this are given in \cite{H1,1}.)  

Turn to the first statement of Proposition 4.3.  Again, 
by the Howe conjecture, the space
of distributions $f\mapsto \Lambda(\gamma_H,f)$ on the Hecke
algebra of $G$ is finite dimensional.  Thus, the Condition 2.B
in the definition of local data may be replaced with the
condition
\begin{equation}\tag{$B'$}
\Lambda(\gamma_j,f) = 0,\quad \text{for } j=1,\dots,k,
\end{equation}
for an appropriate finite collection $\{\gamma_j\}$ of strongly $G$-regular
semisimple elements in $H$.  Condition 2 now implies that, if $\Lambda(\gamma_j,f)=0$,
for $j=1,\ldots,k$, then $A(f)=0$.  This means that the functional
$A$ is a linear combination of the functionals $\Lambda(\gamma_j,\cdot)$.
\end{proof}

\begin{cthm}{Lemma 4.4}  The functional $A$ vanishes identically
on the Hecke algebra.
\end{cthm}

\begin{proof} The parameters $s$,
finite in number, for which $a(s)\ne 0$
are tempered.  This temperedness argument is given by Clozel \cite{Cl2,5.5}.  In the
present context the argument is even easier because we avoid the complications
of base change.  The argument relies on Proposition 4.3 expressing the functional
$A$ as a finite linear combination of the distributions $\Lambda(\gamma_j,\cdot)$ and the temperedness of orbital integrals.

In the compact-trace expansion of $A(f)$ given in Proposition 4.3,
we may assume that each of the representations $\pi$ and
$\pi'$ comes from a nonunitary point in the spectrum.  To see this,
we consider various cases.
Begin with $H$. Since the orbital integrals on $H$ are stable,
we may assume that $\pi'$ 
is obtained by pulling back a representation on the adjoint group.
A tempered representation on the adjoint group with an Iwahori fixed
vector is a full induced unitary principal series representation \cite{Ke}.
By examining the principal series character formula, we see that all 
principal series representations have the same compact trace; in
particular, $\pi'$ has the same compact trace as a nonunitary point in the
spectrum.

Next, consider the representation $\pi$ in the special case that $G_\adj=PGL(n)$
and $\omega$ has order $n$.  
The fundamental lemma has been established in this
case by Kazhdan \cite{Ka}. Thus, $\Lambda(\gamma_j,f)=0$, for all $\gamma_j$,
so that, by the definition of local data, $A(f)$ is also zero.

Finally,  consider any representation $\pi$ 
in any case not yet treated.
Keys has analyzed the reducibility of unitary principal series representations.
By analyzing the parameters of \cite{Ke} case by case, we see that it
is always possible to deform the inducing parameter away from a unitary point
in the spectrum, except in the one case already treated above ($G_\adj=PGL(n)$, $\omega$
of order $n$).  Keys actually treats only the semisimple simply connected case,
but the other cases are an easy consequence of this, the most reducible case.
In fact, in other cases, reducibility is understood by looking at which
constituents have vectors fixed by the various hyperspecial subgroups.

To work one example in more detail, we consider the group $Sp(2n)$
and review some of the results of Keys.
We may
think of the unitary parameter $s$ as lying in the diagonal subgroup
$\{(s_1,s_2,\dots,s_n,1,s_n^{-1},\dots,s_1^{-1})\}$ of
the complex dual group $SO(2n+1,{\Bbb C})$. 
Even if there is reducibility, there will be at most
two constituents.
A unitary
parameter $s$ that gives reducibility satisfies, for instance, $s_n=-1$.
Unitarity implies that
$|s_i|=1$, for all $i$.  But these representations remain
reducible when the unitarity constraint $|s_i| = 1$ is dropped.
A calculation with intertwining operators similar to that
given in \cite{H1,2} shows that the compact trace of each constituent remains 
unchanged as $s$ varies under the constraint $s_n=-1$.
Therefore, there is a nonunitary parameter
$s$ satisfying the constraint except in the rank one situation, where
$s_n=-1$ determines $s$. 
But when the rank is one,
we fall within the case previously considered ($PGL(n)$, $\omega$
of order $n$,
$n=2$).

In every other case, we observe that the intertwining
operators ${\mathcal A}(w,\lambda)$ forming the commuting algebra are
formed by $R$-group elements $w\in W$
that are realized in proper Levi subgroups. In particular,
the constituents
are not elliptic by the results of Arthur \cite{A2}.  
Thus, there is a Levi subgroup
to which we may apply the arguments of \cite{H1,2}.

The final step follows the argument provided by a referee to Clozel's
base change paper \cite{Cl2,p.257}.  
We will rewrite the
identity
$$A(\phi_\lambda) = \sum c_i \text{trace}_c\,\pi(\phi_\lambda)
    + \sum c'_i \text{trace}_c \,\pi'(b(\phi_\lambda)),$$
produced
    by Proposition 4.3, in a more suggestive form.
For $\lambda\in X^*(Y)$, we have
$$A(\phi_\lambda) = \sum_s a(s) \phi_\lambda\hat{\phantom o}(s)
 = \sum_s \sum_{W'} a(s) \lambda(w\cdot s) 
 = \sum_i \lambda(s_i),$$ for some finite collection
of tempered parameters $s_i$ and complex constants $a_i$.

Next we consider a term $\text{trace}_c\,\pi(\phi_\lambda)$.
By a theorem of Clozel and Waldspurger, the compact trace
is a linear combination of forms 
$(\hat\chi_N\phi_\lambda^{(P)})\hat{\phantom o}(z)$.  
The superscript $(P)$ indicates the function obtained from 
integration over the unipotent radical of a standard parabolic
subgroup $P=MN$. 
The function $\hat\chi_N$ factors as a product of three
maps
$$M(F) \to \bfa_M \to \bfa_{M_0} \to {\Bbb R},$$
where $\bfa_M$ is the real Lie algebra of the split center
of $M$, and $M_0\subset M$ is a Levi subgroup of 
a minimal parabolic subgroup $P_0\subset P$.
The first map 
is the Harish-Chandra map $H_M: M\to \bfa_M$. The second
map is a natural
identification of $\bfa_M$ with a subspace of
$\bfa_{M_0}$ (defined by Arthur \cite{A1}).
The third map is
the characteristic function $\hat\tau_P^G$ of
the {\it obtuse\/} Weyl chamber \cite{A1,p.936}, \cite{Cl2,2.1}.
In particular, for $\mu\in X^*(Y)$ and $m\in M(F)$, we have
$\hat\chi_N(m \varpi^\mu m^{-1}) = \hat\tau_P^G(\mu)$,
where we have identified
$X^*(Y)$ with a lattice in $\bfa_{M_0}$.
Then
$$(\hat\chi_N\phi_\lambda^{(P)})\hat{\phantom o} (z) = 
   \sum_{w\in W'} \hat\tau_P^G({}^w\!\lambda)\, {}^w\!\lambda(z),$$
where ${}^w\!\lambda = w\cdot\lambda$.
There are finitely many hyperplanes $X_1,\ldots,X_r$ through
the origin of $X^*(Y)\otimes {\Bbb R}$ such that
$\hat\tau_P^G({}^w\!\lambda) = \hat\tau_P^G({}^w\!\lambda')$
for all $P$ and all $w\in W'$, 
whenever $\lambda$ and $\lambda'$ belong
to the same component of $X^*(Y)\otimes {\Bbb R} \setminus
(X_1\cup\cdots\cup X_r)$.  (For example, take all
singular hyperplanes and all hyperplanes in the $W'$-orbit
of the walls of the obtuse Weyl chambers.)  Fix one
such component $C$.
Then 
$$(\hat\chi_N\phi_\lambda^{(P)})\hat{\phantom o}(z) = 
\sum_{w\in W''}\lambda(w\cdot z)$$ for $\lambda$ in $C$,
for some subset $W''\subset W'$ that depends on
$C$, but not on $\lambda\in C$.


%The sums are finite, the parameters $s_i\in Y$ are unitary, and the
%parameters $z_i\in Y$ are nonunitary.  To write the identity in this form,
%first consider $A(\phi_\lambda)$, or $\sum a(s) \phi_\lambda\hat{\phantom{o}}(s)$.
%Since $\phi_\lambda\hat{\phantom{o}}(s) = \sum_{W'} {}^w\!\lambda(s)$,
%with the summation extending as usual over the Frobenius-fixed subgroup of
%the Weyl group,
%it is clear that $A(\phi_\lambda)$ has the form of the left-hand side
%of (*).  
%$\pi$ comes from a nonunitary point in the spectrum, so
%$z$ is now nonunitary. 

The terms $\text{trace}_c\,\pi'(b(f))$ are treated
similarly.  The map $\xi$ sends nonunitary parameters to nonunitary
parameters, and through $\xi$ this compact trace is expressed as a linear
combination of terms $\lambda(z)$, again for lattice points $\lambda$
in a suitable open cone of $X^*(Y)$.
By passing to a smaller open cone $C'\subset C$, if necessary, to accomodate the
terms $\text{trace}_c\,\pi'(b(f))$, 
the identity then takes the form
    \begin{equation}\tag{*}
    \sum_i a_i \lambda(s_i) = \sum_j a_j'\lambda(z_j),
    \end{equation}
for $\lambda\in C'$.
The sums are finite, the parameters $s_i\in Y$ are unitary, and
the parameters $z_i\in Y$ are nonunitary.
The characters $s_i$, $z_i$ of the lattice 
may be assumed to be linearly independent,
so
 the only such identities (*) are with both sides zero.
Integral combinations
of elements in $C'$ span $X^*(Y)$.
Thus, if both sides vanish on the cone, then both sides vanish
for all $\lambda\in X^*(Y)$.
\end{proof}

\section{Local Transfer }%5

To stabilize the simple trace formula of Deligne and Kazhdan,
as established in \cite{He}, we transfer the $\kappa$-orbital integrals
of a suitable linear combination of matrix coefficients of
supercuspidal representations to the endoscopic
group.  For this result, we work with fields of sufficiently
large residual characteristic.  We may assume that our groups
are unramified, are defined over the ring of integers $O_F$,
and contain an anisotropic Cartan subgroup.
(For the global groups we construct, these conditions will hold locally
at infinitely many places.)

The groups $G$ and $H$ give reductive groups $G_0$ and $H_0$
over the residue field $\Fq$ of $F$.  Select an elliptic Cartan
subgroup $T_0$ in $H_0$ and transfer it to $G_0$.  We may assume
that $T_0$ comes from an unramified Cartan subgroup $T/O_F$ and
that $T(F)\subset H(O_F)$. We identify $T$ with a Cartan subgroup
$T(F)\subset G(O_F)$ by an isomorphism $\psi:T\to G$ defined over
$O_F$.

\begin{cthm}{Lemma 5.1}  In this context, there exists a linear
combination of matrix coefficients of supercuspidal representations
of $G$ whose orbital integrals are supported on the $G(F)$-orbit
of the set of strongly regular semisimple elements of $T(F)$.
The same conclusion holds on $H$, with the additional property
that the orbital integrals of $\gamma$ and $\gamma^n\in T(F)\subset H(F)$ 
are equal if $n\in N_G(T,F)$.
(By $\gamma^n$ we mean $\psi^{-1}(\psi(\gamma)^n)$.)
\end{cthm}


\begin{proof}  Consider the Deligne-Lusztig characters
$R_{T_0,\theta}$ of $G_0(\Fq)$, where $\theta$ is in general
position. (See \cite{C} for a definition and the standard
facts about these generalized characters.)
%% In particular, 9.3.2 and Chapter 4.
 $\pm R_{T_0,\theta}$ is an irreducible cuspidal
character of $G_0(\Fq)$.  
 The characters $\theta$ that are
in general position correspond to 
rational regular elements in
a torus dual to $T_0$, and so the number of such characters
has leading term $q^r$, where $r=\dim\,T_0$.  The Deligne-Lusztig
characters are supported on elements whose semisimple part is conjugate
to $T_0(\Fq)$.  The characters $R_{T_0,\theta}$ and
$R_{T_0,\theta'}$ are linearly independent if $\theta$ and $\theta'$
belong to different orbits under the Weyl group.
The number of singular
elements in $T_0(\Fq)$ is bounded by a constant times $q^{r-1}$.
So for $q$ sufficiently large, an obvious counting argument
shows that there exists a  nonzero linear
combination $f_0$ of irreducible cuspidal
Deligne-Lusztig characters that is supported on the set of
strongly regular semisimple elements conjugate to $T_0(\Fq)$.

If $\sigma$ is an irreducible cuspidal representation of $G_0(\Fq)$,
then the representation of $G(F)$ obtained by the inflation of $\sigma$
to $G(O_F)$ and compact induction to $G(F)$ is supercuspidal.
(See \cite{G,5.2}.)  Extend $f_0$ to $G(O_F)$ and then 
to a function $f\in C_c^\infty(G)$
that is supported on $G(O_F)$.  It follows from the results of
\cite{G,5.2} that $f$ is a finite linear combination of matrix coefficients
of supercuspidal representations of $G$.

The orbital integrals of $f$ are supported on the set of elements 
$\gamma$ conjugate
to elements $g^{-1}\gamma g$
of $G(O_F)$ that are congruent to regular semisimple
element of $T_0(\Fq)$. The powers $(g^{-1}\gamma g)^{p^{m\ell}}$ tend
to $g^{-1}\gamma_s g$, a regular
semisimple element in the $G(O_F)$-orbit of
$T(F)$, where
$\gamma_s$ is the absolutely semisimple part of $\gamma$
(see, for example, \cite{H2,3}).  As $\gamma_s$
and $\gamma$ commute and $\gamma_s$ is regular, we see that $\gamma$
itself lies in the $G(F)$-orbit of the set of regular semisimple
elements of $T(F)$.  Thus, the orbital integrals of $f$ are zero except
on conjugates of regular semisimple elements of $T(F)$.

The proof of the second claim of the lemma is similar.  The
irreducible cuspidal Deligne-Lusztig
characters on $H_0$ from $T_0$ span a vector space whose dimension
is
asymptotic to $q^r/w_0$, where $w_0$ is the cardinality of
$W(T_0,H_0)=N_{H_0}(T_0,\Fq)/T_0(\Fq)$, whereas the 
space of invariant functions on $H_0(\Fq)$
that are invariant by the Weyl group of $(T_0,G_0)$ and are
supported on regular semisimple elements in the orbit of $T_0(\Fq)$
has dimension
$\sim q^r/w_1$, where $w_1$ is the
cardinality of $N_{G_0}(T_0,\Fq)/T_0(\Fq)$.  
Since these are both subspaces of the space (of dimension 
$\sim q^r/w_0$) of functions invariant under $W(T_0,H_0)$, they
have nontrivial intersection for $q$ sufficiently large.
Thus, the desired
linear combinations of Deligne-Lusztig characters exist, when
$q$ is sufficiently large.\end{proof}

We match functions with regular support on $G$ and $H$ by
the following characterization of orbital integrals.

\begin{cthm}{Lemma 5.2}  Let $C_c^\infty(T^{'G})$ be the set of locally
constant
compactly supported functions on the $G(F)$-orbit of the
strongly regular semisimple elements of $T(F)$.  Set
$\Phi(\gamma) = \Phi(\gamma,f)$, for $f\in C_c^\infty(T^{'G})$.
Then $\Phi(\gamma)$, for $\gamma\in T(F)$,  satisfies
\begin{enumerate}[label=(\roman*)]
\item  $\Phi(\gamma^n) = \Phi(\gamma)$, for $n\in N_G(T,F)$,

\item $\Phi(\gamma)$ is a locally constant compactly supported
	function on $T(F)$,

\item $\Phi(\gamma)$ is supported on the strongly regular semisimple
	elements of $T(F)$.
\end{enumerate}
Conversely, a function $\Phi$ on $T(F)$
satisfying (i), (ii), and (iii) is realized
as the
orbital integrals of some function in $C_c^\infty(T^{'G})$.
\end{cthm}

\begin{proof}  This is a special case of \cite{Vi}.\end{proof}

\bigskip
To obtain the simple trace formula for a global group, at some place
we take the
linear combination of matrix coefficients
$f$ on $G(F)$ constructed by Lemma 5.1.  Its $\kappa$-orbital
integrals coincide with its ordinary orbital integrals.
By the characterization of Lemma 5.2, 
there is a matching function $f^H$ on an
endoscopic group $H$.  Similarly, at another place, we may
select the function $f^H$ constructed by the second part of
Lemma 5.1 and find a function on $G$ with matching orbital
integrals by the characterization of Lemma 5.2. 
The hypotheses in \cite{He} for a simple trace formula
are then satisfied for $G$ and $H$.


\section{Global Arguments }%6

This section uses the
matching of the unit elements in the Hecke algebra, 
a global argument, and an inductive hypothesis to
produce local data.  We assume that $G$ is a basic case
and that $H$ is an elliptic endoscopic group of $G$.
 We say that the matching of units holds if 
$\Lambda(\gamma_H,f)=0$, for all strongly $G$-regular
$\gamma_H$, when $f$ is the unit element of the
Hecke algebra.  A reductive group, defined over a number field,
will be associated with each $G$ and endoscopic group $H$.
We are now ready for the main theorem of the paper.

\begin{cthm}{Theorem 6.1}  Suppose that $G$ is a basic case with
elliptic endoscopic group $H$.  
Suppose that $\Lambda(\gamma_H,f)$ is zero
when $\gamma_H$ is not elliptic.  
Suppose that the matching of units holds
at almost all unramified places of the global group and corresponding
endoscopic group associated with
$G$ and $H$.  Then the fundamental lemma holds for $(G,H)$.
\end{cthm}

\begin{proof} 
To simplify notation in the proof, we now shift notation and add a subscript
$w$ to data over the local field $F$.  Thus we have the
local field $F_w$, functions $f_w$, the group $G_w$, an endoscopic
group $H_w$, and so forth.  The terms without subscripts
will be global objects.  Thus, $f$ will be a function on the
adelic points of a global group $G$, a function 
soon to be defined as a product over its
local components, and $H$ will be an endoscopic
group of $G$.

For each reductive group $G_w$ and corresponding elliptic
endoscopic group $H_w$, we
select quasisplit groups $G$ and $H$ over a global field
$F$ that specialize at a given
place $w$ to $G_w$ and $H_w$.  We may choose $F$ in
such a way that $F_v$ is complex for every archimedean place $v$.
The groups $H_w$ and $G_w$ are unramified and the embedding $\xi_w$ of
$L$-groups may be chosen to factor
through a finite unramified extension
$E_w$ of $F_w$ of some degree $k'$  (see \cite{H2}).  Adjusting
$F$, $G$, and $H$ if necessary, we may assume that there is a 
cyclic extension $E/F$ of degree $k'$ that splits $H$ and $G$,
that $E_w$ is a field, and that
the natural map $\phi:\Gal(E_w/F_w)\to \Gal(E/F)$ is an
isomorphism.  The maps $\xi_w$ and $\phi$ combine to give
a global embedding of $L$-groups $\xi:{}^LH\to{}^LG$ that
factors through $W_F\to \Gal(E/F)\simeq \Gal(E_w/F_w)$.  
At every archimedean place $v$,
this embedding of $L$-groups reduces to product of the inclusion
map ${}^LH^0\subset {}^LG^0$ and the identity map $\xi_v:W_{F_v}\to W_{F_v}$.
This means that $\xi_v$ is of {\it unitary type\/} in the sense
of Shelstad \cite{Sh3}.

By the Tchebotarev density theorem, there are then infinitely many places
$v$ at which $E_v$ is a field and $\Gal(E_v/F_v)$ is isomorphic
to $\Gal(E/F)$.  This means that $G$ and $H$ have the weak approximation
property:  $G(F)$ is dense in $G(F_S)$ and $H(F)$ is dense in $H(F_S)$
for the completion $F_S$ at any finite set of places $S$ (see
\cite{KR}).


Fix a regular elliptic element $\gamma_H$ in $H_w(F_w)$.
Select a strongly regular semisimple element
$\gamma\in H(F)$ approximating $\gamma_H$ at $w$.  More specifically,
we demand that $\Lambda(\gamma_w,f) = \Lambda(\gamma_H,f)$,
for all $f$ in the Hecke algebra of $G_w$.
Such elements exist by weak approximation
and the Howe conjecture.  We may also assume, by weak approximation,
that $\gamma$ belongs to an anisotropic unramified Cartan subgroup
at some place $w_1\ne w$, and that $\gamma_v$, for $v$ every archimedean
place, lies in a given open set $U$ (to be specified below).
Let $T$ be the centralizer of $\gamma$.  The Cartan subgroup $T$
is anisotropic and unramified at $w_1$, and so by the Tchebotarev
density theorem, it is anisotropic and unramified at infinitely
many places.

Following Kottwitz \cite{Ko2}, we say that a torus 
of $H$ {\it transfers\/}
to $G$ if there is an admissible
embedding of the torus into $G$, defined over $F$.
A Cartan subgroup in $H$ transfers to $G$ locally everywhere because
$G$ is quasisplit. 
This fact, combined with a criterion of Kottwitz \cite{Ko2,9.5} and
the results of \cite{S,1.9}, implies that 
a Cartan subgroup in $H$ transfers to
$G$ if it is elliptic.  In particular, $T$ transfers to $G$.

Identify $T$ with a Cartan subgroup in $G$ and take the
preimage $T_\s$ of $T$ in $G_\s$.
Consider a character $\kappa$ on the image of $H^1(F_v,T_{\s,v})$ in
$H^1(F_v,T_v)$ at some nonarchimedean place $v$.  In general, the
character $\kappa$ and $T_v$ do not uniquely determine an
endoscopic group $H_v$.  
By Tate-Nakayama, the character $\kappa$
determines a character $s$ on the elements of
$X_*(T_{\s,v})$ of norm zero,  and a choice
is involved in lifting $s$ to a character of $X_*(T_{\s,v})$.  But
when $T_v$ is elliptic, all elements of $X_*(T_{\s,v})$
have norm zero,
and no choice is involved.  Thus, the various $\kappa$
separate the endoscopic groups associated with an elliptic torus.  This
means that there exist functions supported on the regular
elements in the stable orbit of an elliptic Cartan subgroup,
whose $\kappa$-orbital integrals vanish except when $\kappa$
is associated with a single prescribed elliptic endoscopic
group $H_v$.  Choose such a function at a place $v_0$.
Similarly, at another place $v$ at which $T_v$ is
anisotropic and unramified, select matching functions $f_v$ and $f_v^H$
with $f_v$ supported on the $G(F)$-orbit of $T_v(F)$ and $f_v^H$
supported on the stable orbit of $T_v\subset H_v$.  We may select $f_v^H$
in such a way that
the unstable orbital integrals of $f_v^H$ vanish.  Then the only
endoscopic group that is relevant for the stabilization of $H$
is $H$ itself.

Let $S$ be a finite set of nonarchimedean places that includes $w$,
all the special places mentioned above,  and
all the places at which $G$, $H$, or $T$ is ramified.  There
are only finitely many endoscopic groups $H=H_0,H_1,\ldots,H_r$
of $G$ that are
quasisplit forms of $H$, that are equivalent to $H$ at $v_0$,
 and that are unramified outside $S$.
For $i>0$, pick a place
$v_i\not\in S$ at which $H$ and $H_i$ are inequivalent.  Since
endoscopic groups are 
quasisplit, the Tchebotarev density theorem gives
infinitely many choices for $v_i$.  Select a function $f_{v_i}$
supported on the orbit of the unramified torus $T_{v_i}\subset G_{v_i}$.
Then the only endoscopic groups $H$ that are relevant to the 
stabilization of $G$ are associated with $T_{v_i}$ at $v_i$.  Since
$T_{v_i}$ is unramified at $v_i$, every global endoscopic group
associated with $T_{v_i}$ at $v_i$ is also unramified at $v_i$.
The endoscopic group $H=H_0$
is then the only one relevant to the stabilization
of the trace formula, provided the functions
$f_{v_i}$ are used at $\{v_i\}$ and the unit element of the
Hecke algebra is used at all nonarchimedean places except $S\cup \{v_i\}$
(see \cite{Ko3,7.5}).  
\bigskip

Now we consider the complex reductive
group $G_\infty$ at the archimedean places.
Fix a maximal compact subgroup $K$ of $G_\infty$.  
Let $B$ be
a Borel subgroup with Langlands decomposition $B=MAN$.  This
is the only cuspidal parabolic subgroup of the complex
group $G_\infty$.
Let $W$ be the Weyl group of $MA$ in $G$. 
Let $H_\infty$ be a complex endoscopic group of $G_\infty$.  
Similarly, fix a maximal compact subgroup
$K_H$ in $H_\infty$.  The global embedding constructed above allows
us to assume
that the
embedding $\xi_\infty:{}^LH_\infty \to {}^LG_\infty$ is of unitary type.

We recall some facts from the work of Shelstad (\cite{Sh1},\cite{Sh2},
\cite{Sh4}, and especially \cite{Sh3}).
Fix a tempered parameter $\phi'$ for $H_\infty$.
Since $H_\infty$ is quasisplit, $\phi'$
is relevant in the sense of \cite{B}.
We may select $\phi'$ so that the $L$-packet is a
singleton corresponding to an irreducible principal series
representation.  The lift of the character to $G_\infty$ is a well-defined
invariant distribution on $G_\infty$ 
\cite{Sh3,4.0.1}.  The lift is,
up to a sign, the character of a principal series representation
of $G_\infty$.

Fix a regular character $\delta_0\in \hat M$ of $M$ and consider the
principal series representation $\pi_{\delta_0,\lambda}
	=\Ind(\delta\otimes\lambda)$, obtained by
unitary induction, for $\lambda\in\bfa^*$,
$\bfa_0=Lie(A)$, 
and $\bfa = \bfa_0\otimes {\Bbb C}$. 
We may select $\delta_0$ in such a way that
for all $\lambda\in i\bfa_0^*$, the representation $\pi_{\delta_0,\lambda}$
comes from a parameter $\phi'$ (depending on $\lambda$) for
$H_\infty$ in the manner described in the previous paragraph.

Let $C_c^\infty(G_\infty,K)$ denote the space of compactly
supported $C^\infty$ functions that are right and left $K$-finite.
Consider the subspace $C_c^\infty(G_\infty,\delta_0)$ of 
$C_c^\infty(G_\infty,K)$
satisfying the condition
$$\langle \tr \pi_{\delta,\lambda},f\rangle = 0,$$
for all $\lambda\in \bfa^*$ and all 
$\delta\in \hat M\setminus \{W\cdot \delta_0\}$.  By the identities
of \cite{Vo,6.6.7} and the irreducibility of $\pi_{\delta_0,\lambda}$,
we find that
$$\langle \tr \pi,f\rangle = 0,$$
for all irreducible admissible $\pi$ inequivalent
to $\pi_{w\delta_0,\lambda}$,
for $w\in W$ and $\lambda\in \bfa^*$.  

The invariant Paley-Wiener theorem theorem states that the
vector space of functions
	$$F_f(\lambda) = \langle \tr \pi_{\delta_0,\lambda},f\rangle,$$
for $f\in C_c^\infty(G,\delta_0)$, consists of all functions in the Paley-Wiener
space (on the complex vector space $\bfa^*$) 
and that
	$$F_f(w\lambda) = \langle\tr \pi_{w^{-1}\delta_0,\lambda},f\rangle$$
for $w\in W$ \cite{CD}.
Fix a function $f$ for which $F_f(\lambda)$ is not identically
zero.  By the character
formula for principal series representations \cite{Kn,10.18}, there
exists an open set $U$ contained in the set of regular semisimple
elements on which the orbital integrals $f$ of $\gamma\in U$ are nonzero.
Fix a function $f^H \in C_c^\infty(H_\infty)$ that is $K_H$-finite whose
orbital integrals match $f$.  The construction of $f^H$ in \cite{CD,A.4}
shows that the invariant distribution attached to a parameter $\phi_H$
for $H_\infty$ vanishes on $f^H$ except when the parameter
$\xi_\infty\circ\phi_H$ for $G_\infty$ gives the $L$-packet of
$\pi_{\delta_0,\lambda}$ for some $\lambda\in\bfa^*$ (not necessarily
in $i\bfa_0^*$).

Suppose that we have an equality of absolutely convergent sums
\begin{equation}\tag {6.2}
\sum a(\pi) \tr \pi(f) = \sum b(\pi') \tr \pi'(f^H)
\end{equation}
of characters of irreducible unitary representations of $G_\infty$
and $H_\infty$ that holds whenever $f\in C_c^\infty(G_\infty)$ 
and $f^H\in C_c^\infty(H_\infty)$ have matching
orbital integrals.  Inserting the functions $f$ and $f^H$ of the
previous paragraph, the sum for $H_\infty$ reduces to a sum
over irreducible tempered principal series
representations.  (Temperedness follows from the characterization of
 \cite{Kn,8.53,16.6}.)  By the definition of the lift
of a tempered distribution, each term $\tr\pi'(f^H)$ may
be replaced by a term $\tr\pi(f)$ for some tempered representation $\pi$
of $G_\infty$.  By our restriction on the function $f$, the 
character identity between $G_\infty$ and $H_\infty$ 
takes the form of an
absolutely convergent sum 
	\begin{equation}\tag {6.3}
        \sum_{\lambda\in\bfa^*} a(\lambda) F_f(\lambda) = 0,
        \end{equation}
for all $f\in C_c^\infty(G,\delta_0)$.

When $\pi_{\delta_0,\lambda}$ is unitary, we must have
$\lambda\in i\bfa_0^*$, so $a(\lambda)F_f(\lambda)$ vanishes off
$i\bfa_0^*$ (see \cite{Kn,16.6}).
Fix $f\in C_c^\infty(G,\delta_0)$.
We claim that $a(\lambda) F_f(\lambda) = 0$, for all $\lambda$.
Otherwise, there exists a nonzero constant $c = |a(\lambda_0)F_f(\lambda_0)|$ for some
$\lambda_0$.  The sum (6.3)
may be broken into the term $a(\lambda_0)F_f(\lambda_0)$,
a sum over a finite set $S_0\subset i\bfa_0^*$, and 
a sum over the remaining terms.  By choosing $S_0$ large enough,
we may assume that
$$\sum_{\lambda\in \bfa^*\setminus S_0} |a(\lambda)F_f(\lambda)|< {c}.$$
Pick a Paley-Wiener function $h$ on $\bfa^*$
such that $h(\lambda_0)=1$,
$h(\lambda)=0$, for $\lambda\in S_0$, and $|h(\lambda)|\le1$,
for all $\lambda\in i\bfa_0^*$.  
$h(\lambda)F_f(\lambda)$ is a Paley-Wiener function,
so there exists $f_1\in C_c^\infty(G_\infty,K)$ such that
$$F_{f_1}(\lambda)=
	h(\lambda)F_f(\lambda) 
	= \langle\tr \pi_{\delta_0,\lambda},f_1\rangle,$$
and whose trace vanishes on the other components of the admissible
dual of $G_\infty$.  
Apply equation (6.3) to $f_1$ to conclude that
$\sum a(\lambda)h(\lambda)F_f(\lambda)=0$,
with absolute convergence.  We then obtain the contradiction
$$ c = |a(\lambda_0)h(\lambda_0)F_f(\lambda_0)|
	= \left|\sum_{\bfa^*\setminus\{\lambda_0\}} 
	a(\lambda)h(\lambda)F_f(\lambda)\right| < c.$$


\bigskip
	
The simple form of the
trace formula gives a formula for the trace of the operator $R(f)$,
when $f$ is supercuspidal and $R$ is the right-regular representation
of $G(\Bbb A)$ on $L^2(G(F)\backslash G(\Bbb A))$.  (In our context,
$Z(G)(F)\backslash Z(G)(\Bbb A)$ is compact.) % Thm 5.5 of Platonov's book
When $f$ is supercuspidal, the image of $R(f)$ lies in the space
of cusp forms and $R(f)$ is of trace class.  

Kottwitz has stabilized the elliptic part of the trace formula.
We will only use the elliptic regular part, stabilized
by Langlands, obtained
by requiring the support of the function $f$ on the adelic points
of $G$ to be supported on the regular elliptic 
set at some place.  To compare the trace formulas on $G$ and $H$
we use the main identity from Kottwitz \cite{Ko3}, for both $G$ and
$H$.  The stabilization in \cite{Ko3} assumes that the
derived group of $G$ is simply connected.  But, as Kottwitz points out,
this assumption may be avoided; the treatments in \cite{L} and
\cite{KS2} do not make this assumption.
Kottwitz writes $T_e^{**}(f)$ for the elliptic term of the
trace formula, for a function $f$.  The superscript ${**}$
indicates that the sum extends only over the $(G,H)$-regular
terms of the trace formula.  By our support conditions on the
functions $f$ and $f^H$, the omitted terms do not belong
to the support of $f$ anyway.  Similarly, the expression $ST_e^{**}$
stands for the stable elliptic term of the trace formula.
The main identity of Kottwitz, applied to both
$G$ and $H$, becomes
\begin{align*}
T_e^{**}(f) &= i(G,H) ST_e^{**}(f^{H}),\ \text{and}\\
T_e^{**}(f^{H}) &= i(H,H) ST_e^{**}(f^{H}),
\end{align*}
where 
$i(\cdot,\cdot)$ are nonzero constants.
We take $f$ and $f^H$ to be products of compactly supported smooth
functions at all the places.
The functions $f$ and $f^{H}$ 
must have matching orbital integrals
locally everywhere
for these identities to hold.  
Combining the identities, we find a nonzero constant
$c$  for which $T^{**}_e(f) = c T_e^{**}(f^{H})$.

The existence of local data at the place $w$ is now
established by Clozel's arguments.  
We assume that $f$ and $f^{H}$ have matching
orbital integrals everywhere, except possibly at $w$, and
that the $w$-components of $f$ and $f^{H}$ are  $f_w$ and
$b(f_w)$ in the Hecke algebra.  Let $f_e^H$ be the function
obtained from $f^H$ by replacing $f_w^H$ with the characteristic
function of a compact set that meets all elliptic
conjugacy classes in $H_w$. The support of
$f_e^H$ meets only finitely many $H(\Bbb A)$-conjugacy
classes
in $H(\Bbb A)$ that come from global elements in $H(F)$
\cite{Ko3,8.2}.
 Shrink
the support of the function $f_v^H$ at some place $v$
so the only 
$H({\Bbb A})$-conjugacy classes in $H(\Bbb A)$
meeting the support of $f_e^H$ come from $\gamma$.
The transfer of $T$ to $G$ gives a
corresponding global element $\gamma\in T(F)\subset G(F)$.
Every $G(\Bbb A)$-conjugacy class in $G(\Bbb A)$ that comes
from a global element other than $\gamma\in G(F)$ and
that is elliptic
at $w$ has vanishing $\kappa$-orbital integrals at some place other
than $w$.
By the choices made above,
we may arrange 
that the $\kappa$-orbital integrals of $f$ on $\gamma$ 
are nonzero at all
nonarchimedean places except possibly $w$.

Suppose first that the Hecke functions $f_w$ and $b(f_w)$ have matching orbital
integrals at $w$.  
Viewed as an identity in $f_\infty$
and $f_\infty^H$, the spectral side of the identity 
$T_e^{**}(f) = cT_e^{**}(f^H)$  takes the form of Equation 6.2.
Set $F(\lambda) = F_{f_\infty}(\lambda)$.
The argument of 6.3 shows that $a(\lambda)F(\lambda)=0$ for all $\lambda$.
Each term $a(\lambda)F(\lambda)$, viewed as a function
$a(\lambda,f_v)F(\lambda)$ of $f_v$ in the Hecke algebra of $G_v$,
is linear.
By Harish-Chandra's finiteness theorem, applied to both $G$ and $H$,
each identity $a(\lambda,f_v)F(\lambda)=0$ is a finite sum of
the form of Condition
4.1.2.A \cite{BJ}.
This is the implication
(Condition 2.B implies Condition 2.A) in the definition of 
local data.  (It is necessary to vary the elliptic
element $\gamma_H$, to obtain
a collection of character identities for each $\gamma_H$.)

Conversely, if the character identities $a(\lambda,f_v)F(\lambda)=0$
hold for $f_v$ and all $\lambda$, then 
we have an equality on the spectral side of the trace formula.
The identity $T^{**}_e(f) - cT^{**}_e(f^{H})=0$ then holds.
The $\kappa$-orbital integrals of $\gamma$ are
nonzero away from $w$.  Since, up to stable conjugacy,
the support of $f$ contains only
one global element $\gamma$ that is elliptic at $w$, and
since the fundamental lemma is assumed on nonelliptic elements,
this identity simply becomes $\Lambda(\gamma,f_v)=0$.  This is
the implication (Condition 2.A implies Condition 2.B) in the definition of
local data. Note that the constant in the normalization of the
transfer factor at $v$ is fixed by the condition that $\Lambda(\gamma_H,f_v)=0$
when $\gamma_H$ is not elliptic.
\end{proof}


\section{References }
\widestnumber\key{DMOS}

\ref\key A1
\by J. Arthur
\paper A trace formula for reductive groups I
\jour Duke Math. J.
\vol 45
\yr 1978
\paperinfo no. 4
\pages 911--953
\endref

\ref\key A2
\by J. Arthur
\paper On elliptic tempered characters
\jour Acta Math
\vol 171
\yr 1993
\paperinfo no. 1
\pages 73--138
\endref

\ref\key B
\by A. Borel
\paper Automorphic $L$-functions
\inbook Automorphic Forms, Representations and $L$-functions
\bookinfo Proceedings of Symposia in Pure Mathematics, Volume 33, Part 2
\publ  American Mathematical Society
\publaddr Providence, Rhode Island
\yr 1979
\endref

\ref\key BJ
\by A. Borel and H. Jacquet
\paper Automorphic forms and automorphic representations
\inbook Automorphic Forms, Representations and $L$-functions
\bookinfo Proceedings of Symposia in Pure Mathematics, Volume 33, Part 1
\publ  American Mathematical Society
\publaddr Providence, Rhode Island
\yr 1979
\endref

\ref\key C
\by R. Carter
\book Finite Groups of Lie Type: Conjugacy Classes and Complex
	Characters
\publ Wiley Interscience
\publaddr New York
\yr 1986
\endref


\ref\key Cl1
\by L. Clozel
\paper Orbital integrals on $p$-adic groups: A proof of the Howe conjecture
\jour Ann. of Math.
\vol 129
\issue 2
\pages 237--251
\yr 1989
\endref


\ref\key Cl2
\by L. Clozel
\paper The fundamental lemma for stable base change
\jour Duke Math. J.
\yr 1990
\vol 61
\issue 1
\pages 255--302
\endref

\ref\key CD
\by L. Clozel and P. Delorme
\paper Le Th\'eor\`eme de Paley-Wiener invariant pour les groupes
	de Lie r\'eductifs
\jour Inv. Math.
\yr 1984
\vol 77
\pages 427--453
\endref

\ref\key G
\by P. G\'erardin
\book Construction de s\'eries discr\`etes $p$-adiques
\bookinfo Lect. Notes in Math. 462
\publ Springer
\yr 1975
\endref

\ref\key H1
\by T. C. Hales
\paper Unipotent Representations and Unipotent Classes in $SL(n)$
\jour Amer. J. Math.
\vol 115
\paperinfo no. 6
\pages 1347--1383
\yr 1993
\endref


\ref\key H2
\by T. C. Hales
\paper A Simple Definition of Transfer Factors for Unramified Groups 
\jour Contemporary Math
\vol 145
\yr 1993
\pages 109--134
\endref

\ref\key He
\by G. Henniart
\paper La conjecture de Langlands locale pour $GL(3)$
\jour Mem. Soc. Math. France (N.S.)
\yr 1984
\paperinfo No. 11-12
\endref

\ref\key Ka
\by D. Kazhdan
\paper On Lifting
\inbook Lie Group Representations II
\bookinfo Lecture Notes in Math 1041
\yr 1984
\endref

\ref\key Ke
\by D. Keys
\paper Reducibility of Unramified Unitary Principal Series
Representations of $p$-adic Groups and Class-1 Representations
\jour Math. Ann.
\pages 397--402
\vol 259--260
\yr 1982
\endref

\ref\key Kn
\by A. W. Knapp
\book Representation Theory of Semisimple Groups
\yr 1986
\publ Princeton 
\publaddr New Jersey
\endref

\ref\key Ko1
\by R. Kottwitz
\paper Rational Conjugacy Classes in Reductive Groups
\jour Duke Math. J.
\yr 1982
\vol 49
\issue 4
\endref

\ref\key Ko2
\by R. Kottwitz
\paper Stable trace formula: cuspidal tempered terms
\jour Duke Math. J.
\yr 1984
\vol 51, No. 3
\pages 611--650
\endref

\ref\key Ko3
\by R. Kottwitz
\paper Stable Trace Formula: Elliptic Singular Terms
\jour Math. Ann.
\yr 1986
\pages 365--399
\vol 275
\endref

\ref\key KR
\by R. Kottwitz and J. Rogawski
\paper The distributions in the invariant trace formula
are supported on characters 
\paperinfo preprint
\endref

\ref\key KS1
\by R. Kottwitz and D. Shelstad
\paper Twisted Endoscopy I: Definitions, Norm Mappings
  and Transfer Factors 
\paperinfo preprint
\endref

\ref\key KS2
\by R. Kottwitz and D. Shelstad
\paper Twisted Endoscopy II: Basic Global Theory
\paperinfo preprint
\endref

\ref\key L
\by R. P. Langlands
\book Les d\'ebuts d'une formule des traces stable
\publ Publ. Math. Univ. Paris VII
\yr 1983
\endref

\ref\key La
\by J.-P. Labesse
\paper Fonctions \'el\'ementaires\, et\, lemme fondamental\, pour
	le changement de base stable
\jour Duke Math J.
\vol 61
\yr 1990
\paperinfo no. 2
\pages 519--530
\endref

\ref\key LS1
\by R. Langlands and D. Shelstad
\paper On the definition of transfer factors
\jour Math. Ann.
\vol 278
\yr 1987
\pages 219--271
\endref

\ref\key LS2
\by R. Langlands and D. Shelstad
\paper Descent for transfer factors
\inbook The Grothendieck festschrift
\bookinfo Progress in Math.
\publ Birkh\"auser
\yr 1990
\endref

\ref\key M
\by I. G. MacDonald
\book Spherical Functions on a group of $p$-adic type
\publ Ramanujan Institute
\yr 1971
\endref

\ref\key S
\by J.-J. Sansuc
\paper Groupe de Brauer et arithm\'etique des groupes alg\'ebriques lin\'eaires
sur un corps de nombres
\vol 327
\yr 1981
\pages 12--80
\jour J. Reine Angew. Math.
\endref

\ref\key Sh1
\by D. Shelstad
\paper Embeddings of $L$-groups
\jour Can. J. Math
\vol 33,3
\yr 1981
\pages 513--558
\endref

\ref\key Sh2
\by D. Shelstad
\paper Characters and Inner forms of a quasi-split group over ${\Bbb R}$
\jour Comp. Math.
\vol 39
\paperinfo Fasc. 1
\yr 1979
\pages 11--45
\endref

\ref\key Sh3
\by D. Shelstad\ 
\paper $L$-indistinguishability\ \  for\ \  Real Groups
\jour Math. Ann.
\vol 259
\yr 1982
\pages 385--430
\endref

\ref\key Sh4
\by D. Shelstad\ 
\paper Orbital Integrals,\ \  Endoscopic Groups, and $L$-in\-dis\-tin\-guish\-a\-bil\-ity
	for Real Groups
\inbook Journ\'ees Automorphes
\publ Pub. Math. Univ. Paris VII
\bookinfo 15
\yr 1983
\pages 135--219
\endref

\ref\key Vi
\by M.-F. Vign\'eras
\paper Caract\'erisation des int\'egrales orbitales\ \  sur un groupe
	r\'eductif $p$-adique
\jour J. Fac. Sci. Univ. Tokyo, Sect 1A Math.
\yr 1981
\vol 28
\pages 945--961
\endref

\ref\key Vo
\by D. Vogan
\book Representations of Real Reductive Groups
\yr 1981
\publ Birkh\"auser
\publaddr Boston
\endref

\ref\key W
\by J.-L. Waldspurger
\paper Sur les int\'egrales orbitales tordues pour les groupes lin\'eaires:
       un lemme fondamental
\jour Canad. J. Math.
\vol 43
\yr 1991
\issue 4
\pages 852--896
\endref

\enddocument
% end of file
