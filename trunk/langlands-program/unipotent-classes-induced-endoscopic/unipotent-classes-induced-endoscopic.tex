% Thomas C. Hales
% Title: Unipotent classes induced from endoscopic groups
% MSRI preprint 08220-87
% September 1987.
% It includes a few revisions from September 1988.
% The abstract is from May 1989.
% Converted to latex March 2015.

\documentclass{amsart}

%Packages
\usepackage{graphicx}
\usepackage{amsfonts}
\usepackage{enumitem}
\usepackage{amscd}
\usepackage{amssymb}
\usepackage{alltt}
\usepackage{amsrefs}

\newtheorem{thm}{}
\newenvironment{cthm}[1]
  {\renewcommand\thethm{\sc #1}\thm}
  {\endthm}


%\magnification=\magstep1
%\pagewidth{5.75 truein}
%\pageheight{8.4 truein}
%\hcorrection{.25 truein}
%\vcorrection{.25 truein}
%\baselineskip=18 pt          %changed also below
%\nologo
%\NoBlackBoxes
%\printoptions
%\redefine\qed{{\unskip\nobreak\hfil\penalty50\hskip2em\vadjust{}\nobreak\hfil
%    $\square$\parfillskip=0pt\finalhyphendemerits=0\par}}

%Document macros
%\newcommand\dotlabel{%
%	\leavevmode
%	\hbox to 0pt{%
%		\hskip -0.5 in
%		$\bullet$%
%		\hfil}%
%	}%
%\define\ljleftlabel#1{%
%	\leavevmode
%	\hbox to 0pt{%
%		\hskip -0.5 in
%		#1%
%		\hfil}%
%	}%
%\define\Section#1{%
%	\vskip 0.75 in
%	\noindent
%	\subheading{Explicit results for {#1}}
%	\bigskip
%	}%


%Dave's macros for this file
\newcommand\Am		{A_{m-1}}
\newcommand\Amone[1]{A_{m_{#1}-1}}
\newcommand\axi	{a(\xi,\eta)}
\newcommand\bB{{\mathcal B}}
\newcommand\Bug	{\bB_u^G}
\newcommand\C		{{\Bbb C}}
\newcommand\deltai	{\underset{2i}\to \delta}
\newcommand\eps	{\varepsilon}
\newcommand\Elam	{E_\lam}
\newcommand\Emu	{E_\mu}
\newcommand\Gimm	{G_{i+m_1+m_2}}
\newcommand\HBug	{H^*(\Bug,\Q)}
\newcommand\Hom	{\operatorname{Hom}}
\newcommand\I		{\operatorname{I}}
\newcommand\Ilams	{\Ind(\lam^1,\dots,\lam^r)}
\newcommand\Ilammu	{\Ind(\lam^1,\dots,\mu^s)}
\newcommand\Ind	{\operatorname{Ind}}
\newcommand\IndHG	{\Ind_H^G(\cO_u)}
\newcommand\IndHH	{\Ind_{H_1}^{H_2}(\cO_u)}
\newcommand\IndCx[2]{\Ind_{#1}^{#2} C_x}
\newcommand\IndCy[2]{\Ind_{#1}^{#2} C_y}
\newcommand\Lam	{\Lambda}
\newcommand\Lami	{\Lam_i}
\newcommand\lam	{\lambda}
\newcommand\lami	{\lam_i}
\newcommand\lams	{(\lam_1,\dots,\lam_\ell)}
\newcommand\lamsmus{(\lam^1,\dots,\lam^r,\mu^1,\dots,\mu^s)}
\newcommand\lamsprime{(\lam_1',\dots,\lam_\ell')}
\newcommand\lamtil	{\tilde\lambda}
\newcommand\mus	{(\mu_1,\dots,\mu_\ell)}
\newcommand\mutil	{\tilde\mu}
\newcommand\NLam	{N\Lambda_r}
\newcommand\cO		{{\mathcal O}}
\newcommand\Ou		{(\cO_u,\psi)}
\newcommand\Q		{{\Bbb Q}}
\newcommand\rank	{\operatorname{rank}}
\newcommand\rhou	{\rho_u}
\newcommand\rhoxi	{\rho(\xi,\eta)}
\newcommand\sumlam[1]{\sum_{#1} (\tilde\lam_i^2 - \tilde\lambda_i)}
\newcommand\sumxi	{\sum_i (\tilde\xi_i^2 - \tilde\xi_i)}
\newcommand\Thetalam{\Theta_\lam}
\newcommand\Thetamu{\Theta_\mu}
\newcommand\UHo	{UH^0}
\newcommand\UGo	{UG^0}
\newcommand\x		{\times}
\newcommand\Z		{{\Bbb Z}}


%\topmatter
\begin{document}


\begin{abstract}
The Springer correspondence may be used to define a map from
certain unipotent classes in the endoscopic groups of a given
reductive group $G$ to unipotent classes of $G$.  Conjectures of
Langlands and Shelstad relating orbital integrals on $G$ to
orbital integrals on endoscopic groups leads one to suspect the
existence of such a map, and one of the first steps in proving
these conjectures in general is to show that such a map exists
satisfying certain properties. It is verified in this paper that
the map obtained by the Springer correspondence satisfies the
formal properties necessary for this map to be correct map for
the study of orbital integrals and its germs.  The map is
calculated explicitly for the exceptional groups.
\end{abstract}


\title {Unipotent Classes Induced from Endoscopic Groups}
\author {Thomas C. Hales}
\thanks{Research at MSRI sponsored in part by NSF Grant DMS-812079-05.}
\thanks{MSRI 08220-27 preprint series}
\date{September 1987}
\address{Mathematical Sciences Research Institute}

\maketitle

%\input hales1a.tex
%\input hales1b.tex
%\input hales1c.tex

%\enddocument
%\bye

Let $G$ be a connected reductive algebraic group defined over the
complex
number field $\C$.  Lusztig and Spaltenstein \cite{LS},
generalizing a result
of Richardson, have shown 
how to associate a unipotent conjugacy class $\Ind_H^G(\cO)$ in
$G$ with a
unipotent conjugacy class of a Levi subgroup $H$ of $G$.  The
Springer 
representation may be used to extend induction of unipotent
classes to the
more general setting where $H$ is allowed to be any endoscopic
group of 
$G$~\cite{Lu}.  It is no longer true that the
induction map is no longer defined for all unipotent classes.

Such an induction map on unipotent classes is needed for the
study of the
Shalika germ expansions of a $\kappa$-orbital integral on $G$. 
We show that 
the unipotent induction map defined through the Springer
construction satisfies
some of the properties required for it to be relevant to the
study of Shalika
germs.  In particular, we show that the map preserves the
dimension of the
centralizers, is transitive in the appropriate sense, and
preserves the 
partial order on unipotent classes.  
The definition of the induction map in section 2 has previously
been used by Lusztig and others.  Propositions 3 and 4 are direct
reformulations of known theorems.  The exposition of proposition
3 follows \cite{LS}.  Section 5 contains results of Shoji. 
Proposition 6 is new.  For the exceptional groups the proof
relies heavily on the charts of Spaltenstein giving the partial
order on unipotent classes.  The tables in the appendix giving
the induction of unipotent classes are taken directly from
\cite{AL} and \cite{A} for the groups $E_6$, $E_7$, and $E_8$. 
The table for the group $G_2$ is elementary and the table for
$F_4$ is new, but is easily deduced from \cite{A}, using the
involution introduced in section 8.

Since the classification of stable unipotent classes of adjoint
groups is 
independent of the
characteristic of the ground field provided the characteristic is
sufficiently
large~\cite{C}, we
obtain an induction map on stable unipotent classes in an
endoscopic group
$H$ over an arbitrary perfect field $k$ in sufficiently large
characteristic 
by restricting
to stable classes meeting $H(k)$ which the induction map sends to
a stable 
class meeting $G(k)$.

%\leftskip = 1 truein
%\vskip 2 truein

\centerline{\bf Outline}

\bigskip

\begin{enumerate}[label=\arabic*.]
\item Review of Springer's Construction
\item Definition of Induction
\item Dimension of Centralizers
\item Transitivity of Induction
\item Explicit Results for Classical
Groups
\item Preservation of Partial Order on
Classes
\item Extension of theory to Groups over fields
other than $\C$%
\item Explicit results for Exceptional
Groups
\end{enumerate}

%\leftskip = 0 truein
%\rightskip = 0 truein
%\newpage

\section{Review of Springer's Construction} %1

\bigskip

For any connected reductive linear algebraic group $G$ over $\C$,
let 
$\bB = \bB^G$ denote the variety of Borel subgroups of $G$.  For
any $G$ let
$\bB_x = \bB_x^G$ denote the variety of Borel subgroups of $G$
containing $x$.
We have an action of the centralizer $Z_G(x)$ on $\bB_x^G$.  We
are concerned
with $\bB_x^G$ only when $x = u$ is a unipotent element.

The dimension of $\Bug$ is $d_u = \frac 12(\dim Z_G(u) - \rank
G)$.  
Springer~\cites{Spr1,Spr2} has defined a natural action of
the Weyl 
group $W$ of $G$ on the 
cohomology $\HBug$ of $\Bug$.  There is also a group action of
$Z_G(u)$ on
$\HBug$ which descends to an action of the quotient 
$A(u) = Z_G(u) / Z_G(u)^0$.  The action of $A(u)$ commutes with
the action
of $W$ and affords a representation of $A(u) \prod W$ on
$H^{2d_u}(\Bug,\Q)$.  One obtains an injective map from
absolutely irreducible representations $\rho$ of $W$ to
equivalence classes of pairs $(\Ou,\psi)$ where $\psi$ is an
irreducible representation of $A(u)$.  The map is characterized
by the condition that $\rho$ is realized in the $\psi$-isotypic
subspace $H^{2d_u}(\Bug,\Q)_\psi$.  The map from irreducible
representations to such pairs is not surjective, but the image
includes all pairs of the form $(\Ou,1)$ where $1$ denotes the
trivial representation of $A(u)$.  Restricting to the set of
pairs $(\Ou,1)$ we obtain an inverse map from unipotent classes
$\Ou \mapsto (\Ou,1) \mapsto \rho$ to the set of irreducible
representations of $W$.  We call this map the Springer map. 
(Springer's original definition differs from this by the sign
representation).




%\newpage

\section{Definition of Induction} %2

\bigskip
\newcommand\sphat{} %% XX

As in the previous section, $G$ is a reductive algebraic group
over the
complex field $\C$.  This
assumption will remain in force until section 7.  The letter $U$
preceding
the name of a reductive group will denote the set of unipotent
conjugacy
classes in that group.  For a finite group $W$ we let $\hat{W}\sphat$
denote the
set of irreducible representations of $W$ over $\C$ up to
equivalence, or simply the set of characters.  Roughly speaking,
induction 
is given by the following steps:
     $$
     \begin{CD}
     \UHo                                              @.    
\subseteq UH   \\
     @V\text{Springer correspondence}VV                     @.\\
     \hat{W}_H\sphat                                              \\
     @V\text{truncated induction}VV                         @.\\
     \hat{W}_G\sphat                                              \\
     @V\text{Inverse of Springer correspondence}VV          @.\\
     UG                                                @.
     \end{CD}
     $$
The subset $\UHo$ of $UH$ on which induction is defined will be
described
below.  We now give details of the construction.  Let $\cO_u$ be a
unipotent
class in an endoscopic group $H$ of $G$.  The Springer
correspondence
associates with $(\cO_u,1)$ an irreductible representation $\rho$
of the
Weyl group $W_H$ of $G$.  We denote this representation by
$\rhou$ (or by
$\rho(u,\psi)$ if the representation $\psi$ is non-trivial). 
Recall that by
definition \cite{L} the dual group ${}^LH$ of $H$ is the
centralizer of
a semisimple element $s$ in the dual group ${}^LG$ of $G$.  For a
group $G$
over $\C$, the $L$-groups ${}^LH$ and ${}^LG$ are connected and
we may identify
the Weyl group $W_{{}^LH}$ of ${}^LH$ with a reflection subgroup
of the Weyl 
group
$W_{{}^LG}$ of ${}^LG$.  We also identify $W_H$ and $W_G$ with
$W_{{}^LH}$
and $W_{{}^LG}$ respectively so that $W_H$ is considered as a
reflection subgroup of $W_G$.  We pass from an irreducible
representation of
$W_H$ to an irreducible representation of $W_G$ by truncated
induction 
defined by I.G.~MacDonald \cite{M1}.  We briefly recall this
process.  The
Weyl group $W$ of $G$ acts as a reflection group on a real vector
space of
dimension $r = \rank(G)$.  This extends to an action of $W$ on
the graded ring $S^*(V)$ of real valued polynomial functions on
$V$ and on
$R^* = S^*(V)/I$ where $I$ is the ideal of $W$-invariant
polynomials
vanishing at 0.  $R^*(V)$ inherits a grading from $S^*(V)$ and
the 
representation on $R^*$ is isomorphic to the regular
representation of $W$.
Let $W'$ be a reflection subgroup of $W$.  We write $V = V'
\oplus V^{W'}$
as a direct sum of $W'$-modules where $V^{W'}$ is the space of
$W'$-invariant
vectors on $V$.

Let $\rho'$ be an irreducible representation of $W'$.  Let $e$ be
the smallest 
integer
for which $\rho'$ occurs in $R^e(V')$.  It follows from the fact
that $W'$ is
a reflection subgroup that $\rho'$ occurs with multiplicity one
in $R^e(V')$.
Let $U$ be the subspace of $R^e(V')$
which transforms by $\rho'$.  Identify $R^e(V')$  and $U$ with
$W'$-submodules of $R^e(V)$ via the direct sum decomposition 
$V = V' \oplus V^{W'}$.  Then $WU \subseteq R^e(V)$ is an
irreducible
$W$-module denoted by $\I_{W'}^W (\rho')$.  Moreover, $\I_{W'}^W
(\rho')$
occurs with multiplicity zero in $R^d(V)$ for $d<e$ and
multiplicity one
for $d=e$.

If $\I_{W'}^W (\rho')$ is the image of some conjugacy class
$(\cO,1)$ in $G$
by the Springer correspondence then we say that $\cO$ in $UH$ lies
in the 
domain of induction and that $\cO$ is obtained from $C$ by 
{\it endoscopic induction\/} or simply {\it induction}.  There is
no
ambiguity in this terminology when $H$ is a Levi subgroup, for
Lusztig
and Spaltenstein~\cite{LS} have proved this process coincides in
that case
with ordinary induction of unipotent classes.  We write $\UHo$
(or $UH^G$
when the dependence on the original group $G$ needs to be
indicated) for the
set of unipotent conjugacy classes in $H$ that lie in the domain
of induction
to $G$.  If $\cO_u$ lies in $\UHo$, we write $\IndHG$ for the
unipotent class
in $G$ obtained by this process.

%\newpage

\section{Dimension of Centralizers} %3

\bigskip

If $\cO_u \in UG$, set $d_u = d_u^G = \frac 12(\dim Z_G(u) - \rank
G)$.

\begin{cthm}{Proposition} Let $\cO_u \in \UHo$ and $\cO_{u'} =
\IndHG$, then
     $d_u^H = d_{u'}^G$.
     \end{cthm}

\begin{proof} We mentioned in Section~2 that $e_u = e_{u'}$, where
$e_u$ is 
the integer
associated to the irreducible representation $\rhou$ in the
previous section.
(We write $e_{(u,\psi)}$ if the representation $\psi$ of $A(u)$
is non-trivial.)
The proposition then follows if we show $d_u = e_u$.  We recall
the proof
of this fact from \cite{LS} and \cite{BM1}.

Fix a torus $T$ and Borel subgroup $B$ containing $T$.  For every
character
$\chi$ of $T$ (which we extend to a character of $B$) we let
$L_\chi$ be the 
associated line bundle on $B \backslash G \simeq \bB$ and let
$e_1(L_\chi)$ be 
its first Chern class in $H^2(\bB,\Z)$.  We extend $\chi \in
\Hom(X_*,\Z)$ to a 
function from $V_\Q = X_* \otimes \Q$ to $\Q$. Shifting notation
slightly,
$\chi \in S^1(V_\Q)$ where $S^i(V_\Q)$ now denotes the
$\Q$-valued
polynomials of degree $i$ on $V_\Q$.
The Chern class map extends uniquely to a ring homomorphism
$S^*(V_\Q) \to H^*(\bB,\Q)$ which factors through the ideal $I$. 
We obtain an
isomorphism $R^*(V_\Q) \simeq H^*(\bB,\Q)$ in which $R^i(V_\Q)$ 
is sent to
$H^{2i}(\bB,\Q)$ for all $i$ (in particular $H^i(\bB,\Q) = 0$ for
$i$ odd).  
The map from $R^*(V_\Q)$ to $H^*(\bB,\Q)$ is $W$-equivariant.  It
is shown 
in~\cite{BM} that the multiplicity of $\rho_{(u,\psi)}$ in
$H^{2i}(\bB,\Q)$ is zero if $i<d_u$ or if $i=d_u$ and $\psi \ne 1$
and further
that
the multiplicity is 1 if $i=d_u$ and $\psi = 1$.  In light of the
isomorphism
$R^*(V_\Q) \simeq H^*(\bB,\Q)$ we have $e_u = d_u$ (but
$e_{(u,\psi)} > d_u$ for
$\psi$ non-trivial).
\end{proof}

%\newpage

\section{Transitivity of Induction} %4

\bigskip

Let $H_1$ and $H_2$ be endoscopic groups of $G$.  Suppose that 
${}^LH_1 \subseteq {}^LH_2 \subseteq {}^LG$ in such a way that
$H_1$ is also
an endoscopic group of $H_2$.  

\begin{cthm}{Proposition} Suppose that $\cO_u \in UH_1^{H_2}$ and
that either
     $\cO_u \in UH_1^G$ or $\IndHH \in UH_2^G$.  Then $\cO_u \in
UH_1^G$, 
     $\IndHH \in UH_2^G$ and $\Ind_{H_1}^G(\cO_u) =
\Ind_{H_2}^G(\IndHH)$.
     \end{cthm}

\begin{cthm}{Remark} Without further restrictions it is possible for a
     class $\cO_u$ to lie in $UH_1^G$, but not $UH_1^{H_2}$.
     \end{cthm}

\begin{proof} Suppose $\IndHH \in UH_2^G$. Let 
$\IndHH = \cO_{u'}, \quad \Ind_{H_2}^G(\cO_{u'}) = \cO_{u''}$,
and let $W = W_G$, $W' = W_H$, $W'' = W_{H_2}$. Then by
definition
     \begin{align*}
     \rho_{u''} &= \I_{W''}^W \rho_{u'} \\
     &= \I_{W''}^W \I_{W'}^{W''} \rhou \\
     &= \I_{W'}^W \rhou
     \end{align*}
by the transitivity of truncated induction~\cite{C}.  Thus
$\Ind_{H_1}^G(\cO_u) = \Ind_{H_2}^G \IndHH$.  The case $\cO_u \in
UH_1^G$
is proved similarly.
\end{proof}

%\newpage

\section{Explicit Results for Classical Groups} %5
\newcommand\sptilde{} %% XX
\bigskip

We begin with a
few operations on partitions.  We follow the convention that partitions
are increasing: $\lam = \lams, \quad \lam_1 \le \dots \le \lam_\ell$.  We say
that a partition is positive if $\lam_1 > 0$.  We make the definitions:
	%$$
	\begin{alignat*}{2}
	&\eps(\lam) 
		&&: \text{ partition with parts } \lam_i + i-1, i=1,\dots,\ell. \\
	&E(\lam)
		&&: \text{ partition with parts } \frac{\lam_i}2 , 
		i=1,\dots,\ell \text{ and } \lam_i \text{ even.} \\
	&\Theta(\lam)
		&&: \text{ partition with parts } \frac{\lam_i -1}2,
		i=1,\dots,\ell \text{ and } \lam_i \text{ odd.} \\
	&\eps^{-1}(\lam)
		&& \text{ (defined only on strictly increasing partitions }
		\lam) \\
	&
		&&: \text{ partitions with parts } \lam_i - i+1,
		i=1,\dots,\ell. \\
	&(\lam)_+
		&&: \text{ partition with parts } \lam_i, 
		i=1,\dots,\ell \text{ and } \lam_i > 0. \\
	&\lam \cup \mu
		&&: \text{ partition with parts } \lam_i, \mu_j,
		\text{ for all } i,j. \\
	&\lamtil
		&&: \text{ dual partition.} \\
	&\lam+\mu
		&&: \ \overset {\text{def}} % \to 
                = (\lamtil \cup \mutil)\widetilde{\phantom{X}}\sptilde.\\
	&O_d(\lam)
		&&: \text{ partition with parts } \lambda_{\ell+1-i},
		\text{ for all } i \text{ odd.} \\
	&E_v(\lam)
		&&: \text{ partition with parts } \lambda_{\ell+1-i},
		\text{ for all } i \text{ even.}
	\end{alignat*}
	%$$

We let $|\lam| = \sum \lam_i \in \Z_{\ge 0}$ denote the length of the 
partition.  Also write $E^\eps$ for
$\eps^{-1} \cdot E \cdot \eps$ and $\Theta^\eps$ for 
$\eps^{-1} \cdot \Theta \cdot \eps$.
We write $\succeq$ for the natural partial order on partitions.  If
$\lam = \lams, \quad \mu = (\mu_1,\dots,\mu_m)$ then $\lam \succeq \mu$ if
$\lam_{\ell-i} + \lam_{\ell-i+1} + \dots + \lam_\ell \ge \mu_{m-i} +
	\dots + \mu_m$ for all $0 \le i \le \min(\ell,m) -1$.  
As a matter of convenience, if all of the parts of a partition are between
0 and 9 we omit the commas between the parts.

\begin{cthm}{Examples} $\lam = 1245; \eps(\lam) = 1368, E(\lam) = 12, 
	\Theta(\lam) = 02, \eps^{-1}(\lam) = 1122, (\lam)_+ = 1245, 
	\lamtil = 12234, O_d(\lam) = 14, E_v(\lam) = 25, E^\eps(\lam) = 33,
	\Theta^\eps(\lam) = 00$.
	\end{cthm}

We introduce the following notation.  Let $\lam,\mu,\nu,\xi,\eta$ denote
partitions.  For groups of type $A_{n-1}$ we set

%\leftskip = 40 pt
%\parindent = -20 pt

$\cO_\lam =$ unipotent class described by a positive partition $\lam$, 
	$|\lam| = n$.
	
$d(\lam) = \frac 12 (\dim C_G(u) - \rank G)$ for $u\in \cO_\lam$.

$\rho(\lam) =$ irreducible representation of $W_{A_{n-1}}$ described by a
	positive partition $\lam$.

$e(\lam) =$ smallest $a$ such that $\rho(\lam)$ occurs in $S^a(V)$
	(= homogeneous polynomials of degree $a$ on the real vector space $V$
	on which $W_{A_{n-1}}$ acts naturally as a reflection group).

%\leftskip = 0 in
%\parindent = 20pt

For groups of type $B_n, C_n, D_n$ we define $\cO_\lam$ and $d(\lam)$ 
simlarly.

For groups of type $B_n, C_n, D_n$ we let $\rho(\xi,\eta)$ denote the
representation of the Weyl group described by a pair of positive partitions
$\xi,\eta$ with $|\xi| + |\eta| = n$.  These representations are irreducible
except in the case $D_n$ and $\xi = \eta$, in which case the Weyl group 
representation breaks up into two inequivalent irreducible representations 
which we denote by $\rho(\xi,\xi)^+$ and $\rho(\xi,\xi)^-$.  The nonnegative
integer $e(\xi,\eta)$ associated to $\rhoxi$ is defined as above.

This section reviews some explicit results on unipotent classes and Weyl group
representations~\cites{C,Sp}.  The Springer correspondence was first
made explicit for classical groups by Shoji~\cite{Sh1}.  In describing the
induction map, we will often write $\Ind (\lam)$ instead of 
$\Ind_H^G(\cO_\lam)$ for a partition $\lam$.  Also, statements such as
$\lam \in UA_{n-1}$ should be interpreted as $\cO_\lam \in UA_{n-1}$.

\bigskip
\subsection{Explicit results for $A_{n-1}$}~\newline

\noindent
{\bf A.1} \qquad Unipotent conjugacy classes $\cO_\lam$ are parametrized by 
elementary divisors.  Elementary divisors are are described by positive
partitions $\lam$ of $n$.
	$$
	d(\lam) = \frac 12 \sumlam{i} \qquad
	(\lamtil \text{ dual partition}).
	$$

\noindent
{\bf A.2} \qquad The partial ordering unipotent classes by inclusion in the 
closure is $\cO_\lam \succeq \cO_{\lam'}$ if and only if $\lam \succeq \lam'$.

\noindent
{\bf A.3} \qquad Irreducible representations $\rho(\lam)$ of $W_{A_{n-1}}$ are parametrized
by positive partitions $\lam$ of $n$.
	$$
	e(\lam) = \frac 12 \sumlam{i}
	$$

\noindent
{\bf A.4} \qquad The Springer correspondence maps the class $\cO_\lam$ to the 
representation $\rho(\lam)$.  Notice that $e(\lam) = d(\lam)$.

\noindent
{\bf A.5} \qquad Induction from 
$H = \Amone 1 \times \dots \times \Amone r \times T^r$ to $G$.  $H$ is
both an endoscopic and Levi subgroup.  If $\cO_\lam i \in UA_{m_i}$ then
$\Ilams = \lam^1 + \dots + \lam^r$.  
Note that $e(\lam' + \dots + \lam^r) = e(\lam')+\dots+ e(\lam^r)$,
so that $d(\Ilams) = d(\lam^1)+\dots+d(\lam^r)$.  Induction is defined on all
classes in $UH$.

\noindent
{\bf A.6} \qquad If $\lam \succeq \lam'$ and $\mu \succeq \mu'$ then 
$\lam + \mu \succeq \lam' + \mu'$ so that induction is compatible with
the partial order.

\noindent
{\bf A.7} \qquad All unipotent classes and all irreducible representations 
are special.

\bigskip
\subsection{Explicit results for $B_n$}~\newline

\noindent
{\bf B.1} \qquad Unipotent conjugacy classes are described by positive partitions $\lam$ of
$2n+1$ such that every even part occurs an even number of times.
	$$
	d(\lam) = \frac 14 \sumlam{} - \frac 14 (R-1) \qquad 
	R = \#\{i \mid \lami \text{ is odd} \}.
	$$

\noindent
{\bf B.2} \qquad Partial ordering by inclusion is $\cO_\lam \succeq \cO_{\lam'}$ 
if and only if $\lam \succeq \lam'$.

\noindent
{\bf B.3} \qquad Irreducible representations of $W_{B_n}$ are parametrized 
by pairs $(\xi,\eta)$ of positive partitions with $|\xi| + |\eta| = n$.
	$$
	e(\xi,\eta) = \sumxi + \sum_i \tilde\eta_i^2
	$$

\noindent
{\bf B.4} \qquad The Springer correspondence is given by 
$\lam \to (\xi,\eta)$ where
$\xi = (\Theta^\eps(\lam))_+$, $\eta = (E^\eps(\lam))_+$.
The image of the Springer correspondence is described as follows. By adding
zeros as parts as necessary we may assume that $\xi$ has one more part than
$\eta$.  Then $(\xi,\eta)$ lies in the image of the Springer correspondence
if and only if
	$$
	\xi_1 \le \eta_1 \le \xi_2+2 \le \eta_2 + 2 \le \xi_3 + 4 \le \dots
	$$

\noindent
{\bf B.5} \qquad Induction from 
$H = T^r \times \Amone{1} \times \dots \times \Amone{r} \times B_{n_1}
	\times \dots \times B_{n_s}$, $(\sum m_i + \sum n_i = n)$ to $B_n$.
$H$ is an endoscopic group if $s \le 2$.  $H$ is a Levi subgroup if $s\le 1$.
Let $\lam^i \in UA_{m_i-1}$, $\mu^i \in UB_{n_i}$.  Then 
$\Ind(\lam^1,\dots,\lam^r,\mu^1,\dots,\mu^s)$ equals the unipotent class whose
associated partitions $(\xi,\eta)$ are
	\begin{align*}
	\xi &= (O_d\lam^1 + \dots + O_d\lam^r + \Theta^\eps \mu^1
		+ \dots + \Theta^\eps \mu^r)_+ \\
	\eta &= (E_v \lam^1 + \dots + E_v \lam^r + E^\eps \mu^1
		+ \dots + E^\eps \mu^r)_+
	\end{align*}
if such a class exists.  If no such class exists then induction is not
defined for $(\lam^1,\dots,\mu^s)$.

\noindent
{\bf B.6} \qquad We have $d(\lam) = e(\xi,\eta)$ if $\cO_\lam$ corresponds 
to $\rhoxi$ by the
Springer correspondence.  Also with $\xi$ and $\eta$ as defined by induction
	%$$
	\begin{multline*}
	e(\xi,\eta) = e(O_d\lam^1, E_v \lam^1) +\dots+ e(O_d\lam^r,E_v \lam^r) \\
	+ e(\Theta^\eps \mu^1, E^\eps \mu^1) +\dots+
		e(\Theta^\eps \mu^s, E^\eps \mu^s)
	\end{multline*}
	%$$
so that
	$$
	d(\Ilammu) = d(\lam^1) +\dots+ d(\lam^r) +\dots+ d(\mu^s).
	$$

\noindent
{\bf B.7} \qquad Let $[x]$ denote the greatest integer less than or equal to
$x$.

Let $\lam = \lams$ be a positive partition such that every even part occurs
an even number of times.  We assume that $\ell$ is odd.  Define an index
$\delta(\lam,2i)$ for $0 < 2i < \ell$ by
	$$
\delta = \begin{cases} 2 &\text{if $\lam_{2i} = \lam_{2i+1}$ 
is even and the largest odd $\lam_j$} \\
&\qquad \text{for $j<2i$ occurs for an odd $j$.} \\
\left[ \textstyle{\frac{\lam_{2i} + 1}{2}} \right] + 
\left[ \textstyle{\frac{-\lam_{2i+1} + 1}{2}} \right] &\text{otherwise.}
\end{cases}
	$$

\noindent
{\bf B.8} \qquad Domain of Induction.  We carry out the induction process in
two stages.  First we show how to pass from a class in the Levi subgroup
$A_{m_i-1}$ to a partition in $B_{m_i}$ by ordinary induction.  This induction
is defined for all unipotent classes and allows us to assume that there are
no factors of type $A$.  Then we determine which classes
$(\mu^1 ,\dots, \mu^s) \in UB_{n_1} \times\dots\times UB_{n_s}$ lie in the
domain of induction.

Let $\lam = \lams$ be a partition of $m$.  By adding a zero if necessary we
may assume that $\lam$ has an odd number of parts.  Then
$\Ind_{{\Am}}^{B_m} (\lam) = \mus$ where for $2i + 2 < \ell$
	$$
(\mu_{2i+1},\mu_{2i+2}) = \begin{cases} 
(2\lam_{2i+1}, 2\lam_{2i+2}) &\text{if $\lam_{2i+1} = \lam_{2i+2}$} \\
(2\lam_{2i+1} + 1, 2\lam_{2i+2} - 1) &\text{if $\lam_{2i+1} < \lam_{2i+2}$}
\end{cases}
	$$
and $\mu_\ell = 2\lam_\ell + 1$.  From this description it follows that
$\lam \succeq \lam'$ if and only if 
$\Ind_{{\Am}}^{B_m} (\lam) \succeq \Ind_{{\Am}}^{B_m} (\lam')$.

Now let $(\mu^1 ,\dots, \mu^s)$ be an $s$-tuple of partitions.  By adding
zeros as necessary we may assume that $\mu^1 ,\dots, \mu^s$ all have the
same odd number of parts (say $\ell$).  Then $(\mu^1 ,\dots, \mu^s)$ lies in 
the domain of induction if for all $i < \ell/2$
	\begin{equation}\tag{*}
	\delta(\mu^1,2i) +\dots+ \delta(\mu^s,2i) \le 2
\end{equation}
A unipotent conjugacy class $\cO_\mu \in UB_n$ is {\it special\/} if
$\delta(\mu,2i) < 2$ for all $2i < \ell$.  This is equivalent to the usual
definition of special~\cite{C}. We see in particular that if
$s=2$ and $\mu^1,\dots,\mu^s$ are all special then $\lamsmus$ lies in the
domain of induction.  

%\newpage

\bigskip
\subsection{Explicit results for $D_n$}~\newline

\bigskip

\noindent
{\bf D.1} \qquad Unipotent conjugacy classes are described by partitions $\lam$ of $2n$ such
that every even part occurs an even number of times, except that there are
two classes for partitions whose parts are all even.
	$$
	d(\lam) = \frac 14 \sumlam{} - \frac 14 R \qquad
	R = \#\{ i \mid \lami \text{ odd } \}.
	$$

\noindent
{\bf D.2} \qquad The partial ordering by inclusion is 
$\cO_\lam \succeq \cO_{\lam'}$ if and only if $\lam \succeq \lam'$.

\noindent
{\bf D.3} \qquad Irreducible representations of $W_{D_n}$ are described by pairs of partitions
$(\xi,\eta)$ such that $|\xi| + |\eta| = n$.  The representation associated
to $(\xi,\eta)$ is equivalent to $(\eta,\xi)$, but no others.  The
representations are irreducible except for $(\xi,\xi)$ which is a direct
sum of two inequivalent irreducible representations.

\noindent
{\bf D.4} \qquad The Springer correspondence is given by $\lam \to (\xi,\eta)$
where
	$$
	\xi = (E^\eps\lam)_+ \qquad
	\eta = (\Theta^\eps\lam)_+
	$$
The image is described as follows.  Arrange that
$\xi$ and $\eta$ have the same number of parts.  Then $(\xi,\eta)$ lies in the
image if and only if
	\begin{gather*}
	\xi_1 \le \eta_1 \le \xi_2+2 \le \eta_2+2 
		\le \xi_2+4 \le \dots \text{ or} \\
	\eta_1 \le \xi_1 \le \eta_2+2 \le \xi_2+2 \le \eta_2+4 \le \dots
	\end{gather*}

\noindent
{\bf D.5} \qquad Induction to $D_n$ from 
$H = T^r \times \Amone{1} \times\dots\times \Amone{r} \times
	D_{d_1} \times\dots\times D_{d_t}$, $(\sum m_i + \sum d_i = n)$.
$H$ is an endoscopic group if $t \le 2$.  $H$ is a Levi subgroup if 
$t \le 1$.  The unipotent class described by the $r+t$-tuple of partitions
$(\lam^1,\dots,\lam^r,\mu^1,\dots,\mu^t)$ is sent to the Weyl group
representation of $D_n$ described by the pair of partitions.
	\begin{align*}
	\xi &= O_d\lam^1 + O_d\lam^2 +\dots+ O_d\lam^r +
		E^\eps \mu^1 +\dots+ E^\eps \mu^t \\
	\eta &= E_v \lam^1 +\dots+ E_v \lam^r + \Theta^\eps \mu^1
		+\dots+ \Theta^\eps \mu^t
	\end{align*}

\noindent
{\bf D.6} \qquad Again $e$ is additive:  
$e(\xi,\eta) = e(O_d\lam^1,E_v \lam^1)_+ \dots 
	e(E^\eps \mu^t,\Theta^\eps \mu^t)$. Also
$e(\xi,\eta) = d(\lam)$ provided $\cO_\lam$ corresponds to $\rhoxi$. 
It follows that the dimension of the centralizer is preserved as predicted
by \S 3.

\noindent
{\bf D.7} \qquad Let $\lam = \lams$ be a positive partition such that every
even part occurs an even number of times.  We assume that $\ell$ is even.
Define an index $\delta(\lam,2i)$ for $0 < 2i < \ell$ by
	$$
\delta = \begin{cases} 2 &\text{if $\lam_{2i} = \lam_{2i+1}$ 
is even and the largest odd $\lam_j$} \\
&\qquad \text{for $j<2i$ occurs for $j$ odd.} \\
\left[ \textstyle{\frac{\lam_{2i+1} + 1}{2}} \right] + 
\left[ \textstyle{\frac{-\lam_{2i+1} + 1}{2}} \right] &\text{otherwise.}
\end{cases}
	$$
We also define $\delta(\lam,2i+1)$ for $0 < 2i+1 < \ell$ by
		$$
\delta = \begin{cases} 0 &\text{if $\lam_{2i+1} = \lam_{2i+2}$ 
is even and the largest odd $\lam_j$} \\
&\qquad \text{for $j<2i+1$ occurs for $j$ even.} \\
\left[ \textstyle{\frac{\lam_{2i+1} - 1}{2}} \right] + 
\left[ \textstyle{\frac{-\lam_{2i+2} - 1}{2}} \right] &\text{otherwise.}
\end{cases}
	$$

\noindent
{\bf D.8} \qquad Domain of induction.  As with $B_n$ we carry out the 
induction in two stages.  Let $\lam = \lams$ be a partition of $m$.  By
adding zeros as necessary we may assume that $\lam$ has an even number of
parts.  Then $\Ind_{{\Am}}^{D_m} (\lam) = \mus$ where for $2i + 2 \le \ell$
	$$
(\mu_{2i+1},\mu_{2i+2}) = \begin{cases} 
(2\lam_{2i+1}, 2\lam_{2i+2}) &\text{if $\lam_{2i+1} = \lam_{2i+2}$} \\
(2\lam_{2i+1} + 1, 2\lam_{2i+2} - 1) &\text{if $\lam_{2i+1} < \lam_{2i+2}$}
\end{cases}
	$$
From this description it follows that $\lam \succeq \lam'$ if and only if 
$\Ind_{{\Am}}^{D_m} (\lam) \succeq \Ind_{{\Am}}^{D_m} (\lam')$.
	
\noindent
{\bf D.9} \qquad Now let $(\mu^1 ,\dots, \mu^s)$ be an $s$-tuple of 
partitions where $\mu^i \in UD_{d_i}$.  By adding
zeros as necessary we may assume that $\mu^1 ,\dots, \mu^s$ all have the
same even number of parts (say $\ell$).  Then $(\mu^1 ,\dots, \mu^s)$ lies 
in the domain of induction if for all $2i < \ell$
	$$
	\delta(\mu^1,2i) + \delta(\mu^2,2i) +\dots+ \delta(\mu^s,2i) \le 2
	$$
A unipotent conjugacy class $\cO_\mu \in UD_m$ is {\it special\/} if
$\delta(\mu,2i) < 2$ for all $i$. If $t=2$ (i.e. the group is endoscopic) 
and $\mu^1$ and $\mu^2$ are special then the class 
$(\lam^1 ,\dots, \lam^r,\mu^1,\mu^2)$ lies in the domain of induction.  

\bigskip
\subsection{Explicit results for $C_n$}~\newline

\noindent
{\bf C.1} \qquad Unipotent conjugacy classes are described by partitions 
$\lam$ of $2n$ such that every odd part occurs an even number of times.
	$$
	d(\lam) = \frac 14 \sum \lamtil_i^2 +
	\frac 14 R - \frac {\rank G}2 = \frac 14 \sumlam{} +
	\frac 14 R \quad R = \# \{ i \mid \lam_i \text{ odd} \}.
	$$

\noindent
{\bf C.2} \qquad The partial ordering on unipotent classes coincides with the 
natural partial order on partitions, that is $\lam \succeq \lam'$ if and 
only if $\cO_\lam \succeq \cO_{\lam'}$.

\noindent
{\bf C.3} \qquad Irreducible representations of the Weyl group $W_{C_n}$ are 
described by pairs of partitions $(\xi,\eta)$ such that $|\xi| + |\eta| = n$.
	$$
	e(\xi,\eta) = \sumxi + \sum_i \tilde\eta_i^2.
	$$

\noindent
{\bf C.4} \qquad The Springer correspondence is described as follows.  
Arrange that $\lam$ has an even number of parts by calling 0 the first part 
if necessary.  Then $\lam \to (\xi,\eta)$ where
	$$
	\xi = (\Theta^\eps\lam)_+ \qquad \eta = (E^\eps\lam)_+.
	$$
The map is injective.  The image is described as follows.  Arrange that $\xi$
has exactly one more part than $\eta$ by adding 0 as parts are needed.  Then
$(\xi,\eta)$  lies in the image if and only if we have
	$$
	\xi_1 \le \eta_1+1 \le \xi_2+2 \le \eta_2+3 \le \dots
	$$

\noindent
{\bf C.5} \qquad Induction from $H$ to $C_n$ where 
	$$
	H = T^r \times \Amone{1} \times\dots\times \Amone{r} \times
	C_{c_1} \times\dots\times C_{c_s} \times D_{d_1} \times\dots\times
	D_{d_t}
	$$
$n= m_1 +\dots+ m_r+c_1 +\dots+ c_s+d_1 +\dots+ d_t$.
$H$ is an endoscopic subgroup if $s \le 1, t\le 1$.  $H$ is a Levi subgroup
if $s \le 1, t=0$.  The unipotent class of $H$ described by the 
$r+s+t$-tuple of partitions 
$(\lam^1 ,\dots, \lam^r,\mu^1 ,\dots, \mu^s, \nu^1 ,\dots, \nu^t)$ is sent
to the representation of the Weyl group of $C_n$ described by the pair of
partitions
	\begin{align*}
	\xi &= O_d\lam^1 + O_d\lam^2 +\dots+ O_d\lam^r +
		\Theta^\eps \mu^1 +\dots+ \Theta^\eps \mu^s + E^\eps\nu^1 
		+\dots+ E^\eps\nu^t \\
	\eta &= E_v \lam^1 +\dots+ E_v \lam^r +
		E^\eps \mu^1 +\dots+ E^\eps \mu^s + \Theta^\eps\nu^1
		+\dots+ \Theta^\eps\nu^t
	\end{align*}

\noindent
{\bf C.6} \qquad Again $e$ is additive
	$$
	e(\xi,\eta) = e(O_d\lam^1,E_v \lam^1) + e(O_d\lam^2,E_v \lam^2) +\dots+
	e(E^\eps\nu^t, \Theta^\eps\nu^t).
	$$
and since $e(\xi,\eta) = d(\lam)$, if $\cO_\lam$ corresponds to $\rhoxi$ the 
dimension of centralizers correspond as predicted by \S 3.


\noindent
{\bf C.7} \qquad Let $\lam = \lams$ be a positive partition such that every
even part occurs an even number of times.  We assume that $\ell$ is even.
Define an index $\delta(\lam,i)$ for $0 < i < \ell$ by
	$$
\delta = \begin{cases} 
1 &\text{if $i=2k$, $\lam_{2k} = \lam_{2k+1}$, $\lam_{2k}$
is odd, and the largest} \\
&\qquad \text{even $\lam_j$ for $j < 2k$ occurs for $j$ odd.} \\
1 &\text{if $i=2k+1$, $\lam_{2k+1} = \lam_{2k+2}$, $\lam_{2k+1}$
is odd, and the} \\ 
&\qquad \text{largest even $\lam_j$ for $j < 2k+1$ occurs 
for $j$ even.} \\
\left[ \textstyle{\frac{\lam_i}{2}} \right] + 
\left[ \textstyle{\frac{-\lam_{i+1}}{2}} \right] &\text{otherwise.}
\end{cases}
	$$

When it becomes necessary to distinguish this index and the index previously
defined in the discussion of $D_n$, we will write $\delta_C(\lam,i)$ and
$\delta_D(\lam,i)$ respectively.

\noindent
{\bf C.8} \qquad Domain of induction.  As before, we carry out the 
induction in two stages.  Let $\lams$ be a partition of $m$.  
Then $\Ind_{{\Am}}^{C_m} (\lam) = (2\lam_1 ,\dots, 2\lam_\ell)$.  It is
clear that $\lam \succeq \lam'$ if and only if 
$\Ind_{{\Am}}^{C_m} (\lam) \succeq \Ind_{{\Am}}^{C_m} (\lam')$. 
	
\noindent
{\bf C.9} \qquad Finally, let 
$(\mu^1 ,\dots, \mu^s,\nu^1 ,\dots, \nu^t)$ be an $s+t$-tuple of 
partitions corresponding to a unipotent class in 
$C_{c_1} \times\dots\times C_{c_s} \times D_{d_1} \times\dots\times D_{d_t}$.
By adding
parts as necessary we may assume that $\mu^1 ,\dots, \nu^t$ contain the
same even number of parts (say $\ell$).  Then $(\mu^1 ,\dots, \nu^t)$ lies 
in the domain of induction if and only if 
	$$
	\sum_{j=1}^s \delta_C (\mu^j,i) +
	\sum_{j=1}^t \delta_D (\nu^j,i) \le 1 \text{ for } i < \ell.
	$$
A class $\cO_\mu \in UC_m$ is {\it special\/} if
$\delta(\nu,2i) \le 0$ for all $2i < \ell$.  It follows that if
$\cO_\nu \in UD_d$ and $\cO_\mu \in UC_c$ are special then 
$\cO_\nu \times \cO_\mu \in U(D_d \times C_c)^0$.

%\newpage

\section{Preservation of Partial Order on Classes} %6

\bigskip

There is a partial order on unipotent conjugacy classes given by
$\cO_x \succeq \cO_y$ if and only if $\cO_y \subseteq
\overline{\cO_x}$ the 
closure of $\cO_x$. 

\begin{cthm}{Proposition} If $\cO_x$ and $\cO_y$ are classes in $\UHo$
with
     $\cO_x \succeq \cO_y$ then $\Ind \cO_x \succeq \Ind \cO_y$.
     \end{cthm}

\begin{proof} (For classical groups only. We postpone the proof
for exceptional
     groups to $\S 8$ where the induction map is given
explicitly.)  For
     $A_n$ the endoscopic groups are Levi subgroups $M$ and the
proposition
     is clear.  For if $U_P$ is the unipotent radical subgroup of
a parabolic
     subgroup containing $M, \overline{\cO_xU_P} \supseteq
\cO_yU_P$ so the 
     class dense in $\cO_yU_P$ is contained in the closure of the
class dense in
     $\cO_xU_P$, that is $\Ind \cO_x \succeq \Ind \cO_y$.

We also remark that it is enough to prove the result for simple
groups, so
we restrict our attention to $B_n, C_n, D_n$.  The endoscopic
groups are
(remember we are working over $\C$):
     \begin{alignat*}{3}
     &\underline{G_n} &\qquad &\underline{H} && \\
     &B_n &\qquad &M_i \times B_{n-i} &\qquad 
          &M_i \text{ is a Levi subgroup of $B_i$} \\
     &C_n &\qquad &M_i \times D_{n-i} &\qquad 
          &M_i \text{ is a Levi subgroup of $C_i$} \\
     &D_n &\qquad &M_i \times D_{n-i} &\qquad 
          &M_i \text{ is a Levi subgroup of $D_i$.}
     \end{alignat*}
By ordinary Levi induction applied to the classes inside $M_i$ it
is enough
to consider three cases:
     \begin{alignat*} {2}
     &\underline{G_n} &\qquad &\underline{H_n}  \\
     &B_n &\qquad &B_i \times B_{n-i} \\
     &C_n &\qquad &C_i \times D_{n-i} \\
     &D_n &\qquad &D_i \times D_{n-i} \\
     \end{alignat*}
We write in these cases 
     $$
H_n = G_i \times G_{n-i}' \text{ where } G_n' =
\begin{cases} G_n &\text{if $G = B, D$} \\ D_n &\text{if $G=C$.}
\end{cases}
     $$

\begin{cthm}{Lemma}
     \begin{enumerate}[label=(\alph*)]
     \item Fix classes $\cO_x$ and $\cO_y$ in $UG_n$.  For
$\cO$ fixed 
          $\in U\Am$.  Set:
               \begin{gather*}
               \cO_x^\prime = \Ind_{G_n \times \Am}^{G_{n+m}}
(\cO_x \times \cO) \\
               \cO_y^\prime = \Ind_{G_n \times \Am}^{G_{n+m}}
(\cO_y \times \cO)
               \end{gather*}
          Then $\cO_x \succeq \cO_y$ if and only if $\cO_x' \succeq
\cO_y'$.
     \item Fix classes $\cO_x$ and $\cO_y$ in $UA_{n-1}$. 
Then
          $\cO_x \succeq \cO_y$ if and only if
          $\Ind_{A_{n-1}}^{G_n} (\cO_x) \succeq
\Ind_{A_{n-1}}^{G_n} (\cO_y)$.
     \end{enumerate}
     \end{cthm}

Before proving the lemma, we point out how the lemma implies the
proposition
for classical groups by a diagram chase.  Consider the following
diagram:

\bigskip
     $$
     \begin{CD}
     G_i \times G^\prime_{n-i} \times \Amone{1} \times \Amone{2} 
@>0>>
          G_n \times \Amone{1} \times \Amone{2} \\
     @.   @VV6V \\
     @.   G_n \times A_{m_1 + m_2-1} \\
     @VV1V  @VV5V \\
     G_{i+m_1} \times G^\prime_{n-i+m_2}  @>>>  G_{n+m_1+m_2} \\
     @AA2A     @AA4A \\
     A_{i+m_1-1} \times A_{n-i+m_2-1}   @>3>>     A_{n+m_1+m_2-1}
     \end{CD}
     $$
\bigskip

An arrow from $A$ to $B$ in the diagram above indicates that $A$
is an
endoscopic group of $B$.  To simplify notation we write $\Ind_j$
for
$\Ind_A^B$ and label the arrow from $A$ to $B$ in the diagram
with a $j$.

We fix $\cO_2 \succeq \cO_1$ in $U(G_i \times G_{n-i}')$.  By part
(a) of the 
lemma the proposition will follow if we can find an 
$\cO' \in U(G_i \x G'_{n-i} \x A_m)$ such that 
$\Ind_7 (\cO_2 \x \cO) \succeq \Ind_7 (\cO_1 \x \cO)$ where 
$\Ind_7 = \Ind_{G_i \x G_{n-1}' \x A_{m-1}}^{G_n \x \Am}$. 
Select
and fix $m_1,m_2$ and $\cO$ for which there exists $\cO_1', \cO_2'$
satisfying
$\Ind_1 (\cO_i \x \cO) = \Ind_2 (\cO_i')$, $i=1,2$.  Such an $\cO$
exists by
the explicit results given in the previous section (B.5, C.5,
D.5).  
By part (a) of the lemma,
$\Ind_1 (\cO_2 \x \cO) \succeq \Ind_1(\cO_1 \x \cO)$ and by part (b)
of the
lemma $\cO_2' \succeq \cO_1'$.  Induction applied to Levi factors
yields
$\Ind_4 \Ind_3 \cO_2' \succeq \Ind_4 \Ind_3 \cO_1'$.  By the
transitivity
of induction $\Ind_4 \Ind_3 \cO_i' = \Ind_5 \Ind_6 \Ind_0 (\cO_i \x
\cO)$, 
$i=1,2$ so that 
$\Ind_5 \Ind_6 \Ind_0 (\cO_2 \x \cO) \succeq \Ind_5 \Ind_6 \Ind_0
(\cO_1 \x \cO)$.
The lemma allows us to drop ``$\Ind_5$'' to conclude that
$\Ind_6 \Ind_0 (\cO_2 \x \cO) \succeq \Ind_6 \Ind(\cO_1 \x \cO)$.  We
write
$\cO' = \Ind_{\Amone{1} \x \Amone{2}}^{A_{m_1+m_2-1}}(\cO)$.  Then
noting that
$\Ind_6$ acts only on $\cO$ we find that (with $m = m_1 + m_2$)
$\Ind_7 (\cO_2 \x \cO') = \Ind_6 \Ind_0 (\cO_2 \x \cO) \succeq
     \Ind_6 \Ind_0 (\cO_2 \x \cO) = \Ind_7 (\cO_1 \x \cO')$.  
This proves the proposition.

Now we turn to the proof of the lemma.  The validity of (b) was
observed
in~B.8, C.8 and~D.8.  Consider part (a) of the lemma. Consider
classes 
$\cO_\lam$, $\cO_{\lam'} \in UG_n.$  Let $\cO_\lam \succeq
\cO_{\lam'}$, 
$\lam = \lams$, $\lam' = \lamsprime$ (adding zeros as necessary).
Write
      \begin{gather}
\Lambda_i = \lam_{\ell-i+1} + \lam_{\ell-i} +\dots+ \lam_\ell \\
\Lambda_i' = \lam_{\ell-i+1}' + \lam_{\ell-i}' +\dots+ \lam_\ell'
     \end{gather}
 for $i \ge 1$.

Let $\underline{\mu} = (\overbrace{\mu ,\dots, \mu}^r)$ be a
partition
whose parts are all equal.  $\cO_{\underline{\mu}} \in U\Am$, 
$m = |\underline{\mu}| = \mu r$.  
If we show 
$\Ind (\cO_\lam \x \cO_{\underline{\mu}}) \succeq 
     \Ind (\cO_{\lam'} \x \cO_{\underline{\mu}})$ 
if and only if $\cO_\lam \succeq \cO_{\lam'}$ we are done by
induction, 
for every partition can be broken into a sum:
     %$$
     \begin{multline*}
     \mu' = (\mu_1 ,\dots, \mu_{\ell'}) = 
          \overbrace{(\mu_1 ,\dots, \mu_1)}^{\ell'}
          + (\mu_2-\mu_1 ,\dots, \mu_2-\mu_1) +\dots+ \\
     (\mu_3-\mu_2-\mu_1 ,\dots, \mu_3-\mu_2-\mu_1)
          +\dots+ (\mu_{\ell'} - \sum \mu_i ,\dots, \mu_{\ell'} -
\sum \mu_i).
     \end{multline*}
     %$$
of partitions $\underline{\mu}$ having identical parts.  For
example, 
$\mu' = 133558 = (111111) + (22222) + (222) + 3 = 1^6 + 2^5 + 2^3
+ 3^1
     = 1^6 + 1^5 + 1^5 + 1^3 + 1^3 + 1^1 + 1^1 + 1^1$.  
Notice that $\tilde\mu' = 11133556$.

By adding zeros as parts as necessary we may assume that $r \le
\ell$.  Now
let $N\lam$ be the partition such that
     $$
     \cO_{N\lam} = \Ind \cO_\lam \times \cO_{\underline{\mu}}.
     $$
Let $N\Lami = N\lam_{\ell-i+1} +\dots+ N\lam_\ell, i \ge 1$.
Define 
$N\lam'$ and $N\Lami'$ similarly.
A short calculation using the explicit relations ($\S 5$) gives:
     $$
N\Lami = \begin{cases} \Lami + 2i\mu &1 \le i<r. \\ \Lami + 2r\mu &r < i
\le \ell
\end{cases}
     $$
and $\NLam = 2r\mu + \Lam_r + \eta$, where $\eta \in \{0,-1\}$
(In the cases
that $\delta(\lambda,r)$ is defined, $\eta = -1$ if and only if 
$\delta(\lambda,r) > 0$).
Similar relations hold for the primed objects $N\Lami'$.
Now $\cO_{N\Lami} \succeq \cO_{N\Lam'}$ if and only if 
$N\Lami \ge N\Lami' \ge 0$ for all $i$, which holds if and only
if
     \begin{alignat*} {2}
     &\Lami - \Lami' \ge 0, &\quad &i<r      \\
     &\Lam_r - \Lam_r' + (\eta - \eta') = \NLam - N\Lambda_r' \ge
0 && \\
     &\Lami - \Lami' \ge 0, &\quad &i>r. 
     \end{alignat*}
So it is enough to prove that, provided we have $\Lami \ge
\Lami'$ for
$i \ne r$ then $\Lam_r - \Lam_r' + (\eta-\eta') \ge 0$ if and
only if
$\Lam_r - \Lam_r' \ge 0$.  A further calculation using $\S 5$
shows that
     \begin{equation}\tag{*}
     \NLam - \NLam' \in 2\Z.
\end{equation}
Also, since $\eta, \eta' \in \{0,-1\}$
     \begin{equation}\tag{**}
     |\eta-\eta'| \le 1.
\end{equation}
From (**) it follows that if $\Lam_r > \Lam_r'$ then $\NLam \ge
\NLam'$ and
if $\NLam > \NLam'$ then $\Lam_r \ge \Lam_r'$.  From (*) and (**)
if follows
that if $\Lam_r = \Lam_r'$ then $\NLam = \NLam'$.  We are left
with the
case $\NLam = \NLam'$.  It is enough to show that if $\eta' = -1$
then
$\eta = -1$.  

From $N\Lam_{r-1} \ge N\Lam_{r-1}'$ follows
$N\lam_{\ell-r+1}' \ge N\lam_{\ell-r+1} \ge N\lam_{\ell-r}$
and from $N\Lam_{r+1} \ge N\Lam_{r+1}'$ follows
$N\lam_{\ell-r} \ge N\lam_{\ell-r}'$.  Now $\eta' = -1$
if and only if $N\lam_{\ell-r+1}' - N\lam_{\ell-r}' < 2\mu$.  But
then 
$N\lam_{\ell-r+1} - N\lam_{\ell-r} < 2\mu$ so that $\eta = -1$
also.
\end{proof}

%\newpage

\section{Extension of theory to groups over fields other
than $\C$} %7

\bigskip

The classification of stable unipotent classes in terms of
partitions in a
reductive group $G$ is independent of the base field in
sufficiently large
characteristic~\cites{Sp,C}.  This allows us to extend our
construction 
to other base fields
$k$.  In this section we assume that $G$ is a reductive group
defined over a 
perfect field $k$ of sufficiently large characteristic.  Let 
$UH_{\text{rat}}^0$ denote 
the set of stable unipotent classes in $\UHo$ meeting $H(k)$ 
whose image
under induction is a stable class meeting $G(k)$, and let
$UG_{\text{rat}}$
denote the stable unipotent classes of $G$ meeting $G(k)$. 
Induction is
then defined from $UH_{\text{rat}}^0$ to $UG_{\text{rat}}$.

We give two examples.  Let $G = SL(r\ell)$ over $k$ and let $K$
be a cyclic 
extension
of $k$ with $[K:k] = \ell$.  Then there is an endoscopic group
$H$ which
is the subgroup of $\operatorname{Res}_{K/k} GL(r)$ (restriction
of scalars)
consisting of elements whose determinant has norm 1 by $K/k$. 
Over
an algebraic closure $\bar k$, $H$ is isomorphic to the subgroup
of 
elements $(x_1 ,\dots, x_\ell)$
in $\overbrace{GL(r) \times\dots\times GL(r)}^\ell$ such that
$\prod_i \det(x_i) = 1$.  This is a Levi factor of $SL(r\ell)$
over $\bar k$ 
so the Springer
correspondence coincides with ordinary induction of unipotent
classes.
Restricting to classes meeting $H(k)$ we obtain the map of
partitions
     $$
     \lam = (\lam_1 ,\dots, \lam_s) \longrightarrow 
     \ell\lam = (\ell\lam_1 ,\dots, \ell\lam_s)
     $$
where $\lam = (\lam_1 ,\dots, \lam_s)$, $|\lam| = r$ parametrizes
unipotent
classes in $GL(r,K)$ hence in $H(k)$.  Note that 
$\ell d^H(\lam) = d^G(\ell\lam)$.

For a second example, we take $G = B_6$, $H = {}^2(B_3 \times
B_3)$, so that
$H$ splits over a quadratic extension $K$ of $k$.  Classes of 
$UH_{\text{rat}}$ are in bijection with unipotent classes of
$UB_3$ so we have
     \begin{alignat*} {4}
     &\underline{d^H} &\qquad &\underline{UH_{\text{rat}}}
&\qquad
          &\underline{UG_{\text{rat}}} &\qquad &\underline{d^G}
\\
     &0 &\qquad &7 &\qquad &\underline{13} &\qquad &0 \\
     &1 &\qquad &51^2 &\qquad &931 &\qquad &2 \\
     &2 &\qquad &3^21 &\qquad &6^21 &\qquad &4 \\
     &3 &\qquad &32^2 &\qquad &54^2 &\qquad &6 \\
     &4 &\qquad &31^4 &\qquad &532^21 &\qquad &8 \\
     &5 &\qquad &(2^21^3) &\qquad &* &\qquad &* \\
     &9 &\qquad &1^7 &\qquad &2^61 &\qquad &18
     \end{alignat*}
Here $UH_{\text{rat}} \setminus UH_{\text{rat}}^0$ consists of a
single class
--- that with the partition $2^21^3$.  (More generally, for $G =
B_{2n}$,
$H = {}^2(B_n \times B_n)$ the set $UH_{\text{rat}}$ is in
bijection with
$UB_n$ so that
     $$
     2d^H(\lam) = d^G(\lam') \text{ where } \Ind_H^G \lam = \lam'
     $$
and $\lam$ is a partition describing a class of $B_n$.  Also, 
$UH_{\text{rat}}^0$ is in bijection with the stable special
unipotent
classes of $B_n$.)

%\newpage

\section{Exceptional Groups} %8

\bigskip

This section describes induction explicitly for exceptional
groups.  The
table $G_2$ is obtained using the Shoji's explicit description
of the Springer correspondence together with the character table
for $G_2$.  
The tables for $E_n (n = 6,7,8)$ follow directly from the
information on the Springer representation for $E_n$ contained 
in~\cite{AL}, at least for the special unipotent classes in
$\UHo$.
For the non-special classes we resort to the useful tables
of~\cite{A}.

The table $F_4$ is computed as follows.  There is an involution
$\iota$
of $W_{F_4}$~\cite{K}.  It arises from an automorphism of the
vector space
$V$ spanned by the roots which interchanges the short and long
roots.
Consequently the involution preserves the grading of polynomials
in
$S^*(V)$.  Since ${}^L H \subseteq {}^L F_4$ we have
$W_{{}^L H} \subseteq W_{{}^L F_4}$ and $W_H$ may be identified
with
$\iota(W_{{}^L H})$.  If $\rho \in \hat{W}_H\sphat$ let 
$\iota(\rho) \in \hat{W}_{{}^L H}\sphat$ be defined by 
$\iota(\rho)(x) = \rho( \iota^{-1} x)$.  It follows that
     \begin{equation}\tag{*}
     \iota \left( \I_{W_H}^{W_{F_4}} \rho \right) = 
          \I_{W_{{}^L H}}^{W_{{}^L F_4}} \iota(\rho)
\end{equation}
The induction on the right-hand side of (*) is contained
in~\cite{A}.  For
example, if $H = C_4$ (*) gives $\I_{W_{C_4}}^{W_{F_4}}$ in terms
of
$\I_{W_{B_4}}^{W_{F_4}}$.

We give induction for groups whose duals are maximal in ${}^L G$
and which are
not Levi subgroups of $G$  Call these groups ``$L$-maximal.''  The
group
$D_4$ in the $F_4$ table is not $L$-maximal because
${}^L D_4 \subseteq {}^L C_4 \subseteq {}^L F_4$ but is included
for 
reference.  Induction for endoscopic groups follows 
by the transitivity of induction for classes satisfying the
conditions of
the proposition of \S 4.

We denote unipotent conjugacy classes in simple classical groups
by the
corresponding partition.  As in~\cite{C} we identify unipotent
conjugacy classes in exceptional groups by a Levi subgroup of the

distinguished parabolic subgroup associated to each class by the
Bala-Carter
Theorem.  If the group $M$ is of the form $M_1 \x A_1$ and $x \in
UM_1$ we let
$x$ and $x$/ denote the unipotent classes $(x;2)$ and $(x;1^2)$
respectively
of $U(M_1 \x A_1) \overset{\sim}{\longrightarrow} UM_1 \x
UA_1$.  
Similarly
for groups of the form $M_1 \x A_2$ and $x \in UM_1$ we let $x$,
$x$/, and 
$x$// denote the classes $(x;3)$, $(x;21)$ and $(x;1^3)$
respectively of
$U(M_1 \x A_2) \overset{\sim}{\longrightarrow} UM_1 \x UA_2$.
In the case $M = A_2^3$ classes in $A_2$ are denoted by 0,1 and~3
instead
of 2, 21 and~$1^3$ respectively.

A tilde (for example $\tilde A_n$) indicates that the roots
of $A_n$ are identified with short roots of $G$.  
The first column of the tables list the degree
of homogeneity:  $d = \frac 12 (\dim C_G(u) - \rank G)$.  The
second column
lists the unipotent classes of the exceptional group $G$.  The
next column 
labelled $s$, giving
the dimension of the corresponding irreducible Weyl group
representation,
is included to make the relation of the following tables to those

of~\cite{AL} transparent.  Further columns list unipotent classes
in 
$L$-maximal groups $M$ and corresponding dimensions of
representations.
The unipotent classes in an $L$-maximal group $M$ listed in a row
or 
rows following a given
class $y$ in $UG$ are precisely the classes such that
$\Ind_M^G(x) = y$.
No relation between unipotent classes in different endoscopic
groups of $G$
is implied by these tables.  When $M$ is of the form $A_q^p$ then
the 
induction map does not depend on the order of the factors and
classes of $UM$
are listed only up to a permutation of factors.

The following list gives all classes in $UM \backslash UM^G$, $M$

$L$-maximal.  \newline
$S \overset {\text{def}}{=} UM \backslash UM^G = \emptyset$
unless    
     \begin{align*}
     &G = F_4, \quad M = C_4, \quad S = \{ 21^6 \} \\
     &G = F_4, \quad M = D_4 \text{ (not $L$-maximal)}, 
          \quad S = \{ 32^21, 1^8 \} \\
     &G = E_7, \quad M = D_6 + A_1, \quad S = \{ 32^2 1 / \} \\
     &G = E_8, \quad M = E_7 + A_1, \quad S = \{ A_5'/, 3A_1'/ \}
\\
     &G = E_8, \quad M = E_6 + A_2, \quad S = \{ 2A_2 + A_1// \}
\\
     &G = E_8, \quad M = D_8, \quad S = \{ 54^23, 4^232^21,
32^61, 32^21^9 \}.
     \end{align*}

The data in the tables that follow make it possible to complete
the proof
of Proposition~6.

\begin{cthm}{Lemma~1} Let $G$ be an exceptional group and let $H$ be
an
     endoscopic group of $G$.  If $\cO_x$ and $\cO_y$ are classes
in $UH^0$
     with $\cO_x \succeq \cO_y$ then $\Ind \cO_x \succeq \Ind \cO_y$.
     \end{cthm}
     
\begin{cthm}{Remark} The proof of this lemma assumes the validity of the
diagrams
in~\cite{Sp} giving the partial order on unipotent classes for
exceptional
groups.  The diagrams in~\cite{C} also purporting to give the
partial order
differ slightly from those of~\cite{Sp} for $E_7$ and~$E_8$. 
This lemma
is false if the partial orders in~\cite{C} are the correct ones.
\end{cthm}

\begin{proof} Let $M$ be one of the maximal groups in the tables. 
It follows
by comparing the tables against the diagrams giving the partial
order 
in~\cite{Sp} that if $\cO_u$ and $\cO_v$ are the classes in $UM^G$
with
$\cO_u \succeq \cO_v$ then $\Ind_M^G \cO_u \succeq \Ind_M^G \cO_v$.

Let $H$ be an endoscopic group of $G$.  By definition we have
${}^L H \subseteq {}^L M \subseteq {}^L G$ for some maximal ${}^L
M$.  By
the transitivity of induction we may assume there is no group
$H'$ with
${}^L H \subsetneqq {}^L H' \subseteqq {}^L M$ with $\cO_x, \cO_y
\in UH^{H'}$.
By the preceding remarks we may also assume that $H \ne M$ and
that
$\cO_x$ or $\cO_y$ belongs to $UH^G \backslash UH^M$.  We observe
that if
$G = E_8$, $M = E_7 + A_1$, $H \ne M$ and $UH^M \backslash UH \ne
\emptyset$
then $H \subseteq D_6 + A_1 + A_1 = D_6 + D_2 \subseteq D_8$. 
Thus in this
case we may replace $M$ by $M' = D_8$.  This allows us to assume
that $M$
contains no $E_7$ factor.  By the description of endoscopic
groups in \S 5
we may assume that if $M$ has a $B_n$ (resp. $C_n$, $D_n$) factor
then $H$
has a $B_{n-i} + B_i$ (resp. $C_{n-i} + D_i$, $D_{n-i} + D_i$)
factor.

The preceding reductions reduce the proof to the following list
of
possibilities for $G, M, H$ with $UH^G \backslash UH^M \ne
\emptyset$.
     \begin{equation}\tag{*}
     \begin{alignedat} 5
     &\underline{G} &\qquad &\underline{M} &\qquad
          &\underline{H} &\qquad &\underline{UH^G \backslash
UH^M} 
          &\qquad &\underline{\Ind_H^G (\cO)}\\
     &F_4 &\qquad &B_3 + \tilde A_1 &\qquad &B_2 + B_1 + \tilde
A_1 
          &\qquad &(2^21, 1^3) &\qquad &B_3 \\
     &E_7 &\qquad &D_6 + A_1 &\qquad &D_4 + D_2 + A_1 &\qquad
&(32^21,1^4,2)
          &\qquad &A_6 \\
     &E_8 &\qquad &D_8 &\qquad &D_4 + D_4 &\qquad &(32^21,51^3)
          &\qquad &E_8(a_5) \\
     &E_8 &\qquad &D_8 &\qquad &D_5 + D_3 &\qquad &(52^21,31^3)
          &\qquad &E_8(a_5) \\
     &&&&&&&(32^21^3,1^6) &\qquad &A_6 + A_1 \\
     &E_8 &\qquad &D_8 &\qquad &D_6 + D_2 &\qquad &(72^21,1^4)
          &\qquad &E_8(a_5) \\
     &&&&&&&(32^41,1^4) &\qquad &A_6 + A_1
     \end{alignedat}
\end{equation}

One now checks case by case that if $H$ is one of the groups
listed and
$\cO_x$ or $\cO_y$ is one of the unipotent classes in $UH^G
\backslash UH^M$
then $\Ind \cO_x \succeq \Ind \cO_y$.  The next paragraph gives
details for
$E_8$ the other cases being similar but much easier.  The proof
is now
complete for $G_2$ and $E_6$.

Let $G = E_8$.  By the reductions above, we may assume $M$ has no
$E_7$
factor.  By the tables, if $H'$ is an endoscopic group of $M =
E_6 + A_2$,
$2A_4$, or $A_8$ then $U{H'}^M = UH'$.  Thus we may assume $M =
D_8$,
$H = D_i + D_{8-i}$, $6 \ge i \ge 4$.  Notice that $D_7 + D_1 =
D_7 + T^1$
is a Levi subfactor of $D_8$ and may be excluded.  Also the
classes in
$D_2,D_3$ are all special.  Let $(\lam^1,\lam^2) \in UH
\backslash UH^M$;
then $\delta(\lam^1,2j) + \delta(\lam^2,2j) > 2$ for some $j$ so
that
in particular $\lam^1$ is not special.  This yields the
possibilities
     \begin{alignat*} {3}
     &D_4 + D_4 &\qquad &(\lam^1 = 32^21; 
          &\quad &\lam^2 = 51^3,32^21,31^5,1^8) \\
     &D_5 + D_3 &\qquad &(\lam^1 = 52^21,32^21;
          &\quad &\lam^2 = 31^3,1^6) \\
     &D_6 + D_2 &\qquad &(\lam^1 = 72^21,52^21^3,3^22^21^5,32^41;
          &\quad &\lam^2 = 1^4)
     \end{alignat*}
By \cite{A} and D.5 the only classes in this list which lie in
$UH^G$ are
the five pairs of possibilities listed above in (*).

To show that if $\cO_x$ or $\cO_y \in UH^G \backslash UH^M$ then
$\Ind_H^G \cO_x \succeq \Ind_H^G \cO_y$ argue as follows.  Consider
first the
case $\Ind(\cdot) = E_8(a_5) \in UE_8$.  An examination of the
partial order
on $UE_8$ revelas that for $\cO \in UE_8$, $\cO \succ E_8(a_5)$ if
and only
if $d_\cO < 6 = d_{E_8(a_5)}$ and that $\cO \prec E_8(a_5)$ if and
only if
$d_\cO > 6$.  Thus in this case the partial order is preserved
because
$d_\cO = d_{\cO'}$ if $\Ind \cO' = \cO$.

Finally, let $\cO_z$ be one of the classes of $UH^G \backslash
UH^M$ in (*)
with $\Ind \cO_z = A_6 + A_1$.  The classes $\cO \in UE_8$ with 
$d_\cO \ne d_{A_6 + A_1} = 14$, $\overline{\cO} \cap (A_6 + A_1) =
\emptyset$
and $\cO \cap \overline{A_6 + A_1} = \emptyset$ are $\cO = E_6$, 
$E_6(a_1)$, $D_6(a_1)$, $D_5 + A_1$, and $D_5$.  By the tables,
the classes
in $UD_8$ which map to one of these are $91^7,731^6$, $532^4$,
$4^22^4b$
and~$71^9$.  It is a simple matter to check that if $\cO_w \in
UH^G$
and $\overline{\cO}_w \cap \cO_z \ne \emptyset$ or $\cO_w \cap
\overline{\cO}_z$
then $\Ind_H^{D_8} (\cO_w) \ne 91^7,731^6,532^4,4^22^4b,71^9$. 
The
cases $G = F_4,E_7$ proceed similarly.
\end{proof}
\vfill\newpage

% END OF HHALES1C.TEX

 %% E6 FILE:

%\magnification=\magstep1

%Paper: Hales1
%Template: Table2
%Title: The Group E6

%\nologo

% \setstrut
% compute \strut as a \vrule of
%	width = 0
%	height = 70% of the current value of \baselineskip
%	depth = the remaining 30% of \baselineskip
\newdimen\strutskip
\renewcommand\strut{\vrule height 0.7\strutskip
					 depth 0.3\strutskip
					 width 0.2pt}%
\newcommand\setstrut{%
	\strutskip = \baselineskip
	}

\baselineskip = 14pt

$$
\vbox{
\setstrut
\offinterlineskip
%
			%tab0 and tabn are both \hfil
			\tabskip = 0pt plus 1fil
\halign to \hsize{
	#\hfil		\tabskip = 1.0 cm&	%Col 1	= d
	\hfil #		\tabskip = 0.25cm&	%2		= E6
	#\hfil		\tabskip = 0.25cm&	%3		= s
	\strut#		\tabskip = 0.25cm&  %4		= dividing line
	\hfil #		\tabskip = 0.25cm&	%5		= A23
	#\hfil		\tabskip = 1.0 cm&	%6		= s
	\hfil #		\tabskip = 0.25cm&	%7		= A1+A5
	#\hfil		\tabskip = 0pt plus 1fil\cr	%8	= s
%End preamble
%Row 1 + \hrule begins here
d&  $E_6$ class&  s& \omit& $3A_2$&  s&  $A_1+A_5$&  s\cr\noalign{\vskip 0.2cm
							\hrule height 2pt
							\vskip 0.2cm}
%Row 2 begins here
0&  $E_6$&  1&&  0+0+0&  1&  6&  1\cr
1&  $E_6(a_1)$&  6&&  0+0+1&  2&  51&  5\cr
&&&&&&  6/&  1\cr
2&  $D_5$&  20&&  0+1+1&  4&  42&  9\cr
&&&&&&  51/&  5\cr
3&  $E_6(a_3)$&  30&&  1+1+1&  8&  $41^2$&  10\cr
&&&&  0+0+3&  1&  $3^2$&  5\cr
&&&&&& $42/$&  9\cr
4&  $A_5$&  15&&&&  $3^2$/&  5\cr
4&  $D_5(a_1)$&  64&&  0+1+3&  2&  321&  16\cr
&&&&&&  $41^2$/&  10\cr
5&  $A_4+A_1$&  60&&  1+1+3&  4&  321/&  16\cr
6&  $D_4$&  24&&  0+3+3&  1&  $2^3$&  5\cr
6&  $A_4$&  81&&&&  $31^3$&  10\cr
7&  $D_4(a_1)$&  80&&  1+3+3&  2&  $2^21^2$&  9\cr
&&&&&&  $31^3$/&  10\cr
&&&&&&  $2^3$/&  5\cr
8&  $A_3+A_1$&  60&&&&  $2^21^2$/&  9\cr
9&  $2A_2+A_1$&  10&&  3+3+3&  1&& \cr
10&  $A_3$&  81&&&&  $21^4$&  5\cr
11&  $A_2+2A_1$&  60&&&&  $21^4$/&  5\cr
12&  $2A_2$&  24&&&&& \cr
13&  $A_2+A_1$&  64&&&&& \cr
15&  $A_2$&  30&&&&  $1^6$&  1\cr
16&  $3A_1$&  15&&&&  $1^6$/&  1\cr
20&  $2A_1$&  20&&&&& \cr
25&  $A_1$&  6&&&&& \cr
36&  1&  1&&&&& \cr
}%	end \halign
}$$%	end \vbox

%\end
\vfill\newpage

% END OF E6 FILE

% E7 FILE

%Paper: Hales1
%Template: Table3
%Title: The Group E7

%\nologo
%\hsize = 7 truein
%\vsize = 9 truein

% \setstrut
% compute \strut as a \vrule of
%	width = 0
%	height = 70% of the current value of \baselineskip
%	depth = the remaining 30% of \baselineskip
\newdimen\strutskip
\renewcommand\strut{\vrule height 0.7\strutskip
					 depth 0.3\strutskip
					 width 0.2pt}%
%\newcommand\setstrut{%
%	\strutskip = \baselineskip
%	}

\baselineskip = 12pt

$$
\vbox{
\setstrut
\offinterlineskip
%
			%tab0 and tabn are both \hfil
			\tabskip = 0pt plus 1fil
\halign to \hsize{
	#\hfil		\tabskip = 1.0 cm&	%Col 1	= d
	\hfil #		\tabskip = 0.5 cm&	%2		= E7
	#\hfil		\tabskip = 0.25cm&	%3		= s
	\strut#		\tabskip = 0.25cm&	%4		= dividing line
	\hfil #		\tabskip = 0.5 cm&	%5		= A7
	#\hfil		\tabskip = 1.0 cm&	%6		= s
	\hfil #		\tabskip = 0.5 cm&	%7		= A5+A2		
	#\hfil		\tabskip = 1.0 cm&  %8		= s
	\hfil #		\tabskip = 0.5 cm&	%9		= D6+A1
	#\hfil		\tabskip = 0pt plus 1fil\cr	%10	= s
%End preamble
%Row 1 + \hrule begins here
d&  $E_7$ class&  s& \omit& $A_7$&  s&  $A_5 + A_2$&  s&  $D_6+A_1$&
				s\cr\noalign{\vskip 0.2cm
				 	 		\hrule height 2pt
							\vskip 0.2cm}
%Row 2 begins here
0&  $E_7$&  1&&  8&  1&  6&  1& $\underline{11},1$& 1\cr
1&  $E_7(a_1)$&  7&&  71&  7&  51&  5& 93& 6\cr
&&&&&&  6/&  2& $\underline{11},1$/& 1\cr
2&  $E_7(a_2)$&  27&&  62&  20&  42&  9& $91^3$& 5\cr
&&&&&&  51/&  10& 75& 15\cr
&&&&&&&& 93/& 6\cr
3&  $E_7(a_3)$&  56&&  $61^2$&  21&  $41^2$&  10& $731^2$& 24\cr
&&&&  53&  28&  42/&  18& $6^2$a& 10\cr
&&&&&&  6//&  1& 75/& 15\cr
3&  $E_6$&  21&&&&  $3^2$&  5& $6^2$b& 10\cr
&&&&&&&& $91^3$/& 5\cr
4&  $E_6(a_1)$&  120&&  521&  64&  321&  16& $5^21^2$& 45\cr
&&&&&&  $41^2$/&  20& $731^2$/& 24\cr
&&&&&&  $3^2$/&  10& $6^2$b/& 10\cr
&&&&&&  51//&  5&&\cr
4&  $D_6$&  35&&  $4^2$&  14&&& $6^2$a/& 10\cr
&&&&&&&& $72^21$& 9\cr
5&  $E_7(a_4)$&  189&&  431&  70&  321/&  32& $53^21$& 30\cr
&&&&&&  42//&  9& $5^21^2$/& 45\cr
&&&&&&&& $72^21$/& 9\cr
6&  $D_6(a_1)$&  210&&  $51^3$&  35&  $31^3$&  10& $71^5$& 10\cr
&&&&&&&& $532^2$& 45\cr
6&  $D_5+A_1$&  168&&  $42^2$&  56&  $2^3$&  5& $4^231$& 30\cr
&&&&&&  $41^2$//&  10&&\cr
6&  $A_6$&  105&&&&  $3^2$//&  5& $53^21$/& 30\cr
7&  $E_7(a_5)$&  315&&  $421^2$&  90&  $31^3$/&  20& $531^4$& 36\cr
&&&&  $3^22$&  42&  $2^3$/&  10& $4^22^2$a& 40\cr
&&&&&&  321//&  16& $4^231$/& 30\cr
&&&&&&&& $532^2$/& 45\cr
7&  $D_5$&  189&&&&  $2^21^2$&  9& $4^22^2$b& 40\cr
&&&&&&&& $71^5$/& 10\cr
8&  $E_6(a_3)$&  405&&&&  $2^21^2$/&  18& $4^21^4$& 45\cr
&&&&&&&& $3^4$& 30\cr
&&&&&&&& $531^4$/& 36\cr
&&&&&&&& $4^22^2$b/& 40\cr
8&  $D_6(a_2)$&  280&&  $3^21^2$&  56&&& $4^22^2$a/& 40\cr
&&&&&&&& $52^21^3$&  16\cr
9&  $D_5(a_1)+A_1$&  378&&  $32^21$&  70&  $31^3$//&  10& $4^21^4$/& 45\cr
&&&&&&&& $3^31^3$& 30\cr
9&  $A_5+A_1$&  70&&&&  $2^3$//&  5&&\cr
9&  $(A_5)^\prime$&  216&&&&&& $3^4$/& 30\cr
&&&&&&&& $52^21^3$&  16\cr
10&  $A_4+A_2$&  210&&&&  $2^21^2$//&  9& $3^31^3$/& 30\cr
10&  $D_5(a_1)$&  420&&  $41^4$&  35&  $21^4$&  5& $3^22^21^2$& 45\cr
}% 	end \halign
}$$%	end \vbox

%\newpage
%The Group E7, page 2

$$
\vbox{
\setstrut
\offinterlineskip
%
			%tab0 and tabn are both \hfil
			\tabskip = 0pt plus 1fil
\halign to \hsize{
	#\hfil		\tabskip = 1.0 cm&	%Col 1	= d
	\hfil #		\tabskip = 0.5 cm&	%2		= E7
	#\hfil		\tabskip = 0.25cm&	%3		= s
	\strut#		\tabskip = 0.25cm&	%4		= dividing line
	\hfil #		\tabskip = 0.5 cm&	%5		= A7
	#\hfil		\tabskip = 1.0 cm&	%6		= s
	\hfil #		\tabskip = 0.5 cm&	%7		= A5+A2		
	#\hfil		\tabskip = 1.0 cm&  %8		= s
	\hfil #		\tabskip = 0.5 cm&	%9		= D6+A1
	#\hfil		\tabskip = 0pt plus 1fil\cr	%10	= s
%End preamble
%Row 1 + \hrule begins here
d&  $E_7$ class&  s& \omit& $A_7$&  s&  $A_5 + A_2$&  s&  $D_6+A_1$&
				s\cr\noalign{\vskip 0.2cm
				 	 		\hrule height 2pt
							\vskip 0.2cm}
%Row 2 begins here
11&  $A_4+A_1$&  512&&  $321^3$&  64&  $21^4$/&  10& $3^22^21^2$/& 45\cr
12&  $D_4+A_1$&  84&&  $2^4$&  14&&&  $32^41$&  5\cr
12&  $(A_5)^{\prime\prime}$&  105&&&&&& $51^7$& 10\cr
13&  $A_3+A_2+A_1$&  210&&  $2^31^2$&  28&  $21^4$//&  5&  $32^41$/&  5\cr
13&  $A_4$&  420&&&&&& $3^21^6$& 24\cr
&&&&&&&& $51^7$/& 10\cr
14&  $A_3+A_2$&  378&&&&&& $3^21^6$/& 24\cr
&&&&&&&& $32^21^5$&  9\cr
15&  $D_4(a_1)+A_1$&  405&&  $31^5$&  21&&& $2^6$a& 10\cr
15&  $D_4$&  105&&&&  $1^6$&  1& $2^6$b& 10\cr
16&  $A_3+2A_1$&  216&&  $2^21^4$&  20&&& $2^6$a/& 10\cr
16&  $D_4(a_1)$&  315&&&&  $1^6$/&  2& $2^41^4$& 15\cr
&&&&&&&& $2^6$b/& 10\cr
17&  $(A_3+A_1)^\prime$&  280&&&&&& $2^41^4$/& 15\cr
18&  $2A_2+A_1$&  70&&&&  $1^6$//&  1&&\cr
20&  $(A_3+A_1)^{\prime\prime}$&  189&&&&&& $31^9$& 5\cr
21&  $A_2+3A_1$&  105&&  $21^6$&  7&&&&\cr
21&  $2A_2$&  168&&&&&& $31^9$/& 5\cr
21&  $A_3$&  210&&&&&& $2^21^8$& 6\cr
22&  $A_2+2A_1$&  189&&&&&& $2^21^8$/& 6\cr
25&  $A_2+A_1$&  120&&&&&&&\cr
28&  $4A_1$&  15&&  $1^8$&  1&&&&\cr
30&  $A_2$&  56&&&&&& $1^{12}$& 1\cr
31&  $(3A_1)^\prime$&  35&&&&&& $1^{12}$/& 1\cr
36&  $(3A_1)^{\prime\prime}$&  21&&&&&&&\cr
37&  $2A_1$&  27&&&&&&&\cr
46&  $A_1$&  7&&&&&&&\cr
63&  1&  1&&&&&&&\cr
}% 	end \halign
}$$%	end \vbox

\vfill\newpage
%\end

% END OF E7

% START OF E8
%Paper: Hales1
%Template: Table4
%Title: The Group E8

\newdimen\strutskip
\renewcommand\strut{\vrule height 0.7\strutskip
					 depth 0.3\strutskip
					 width 0.2pt}%
\baselineskip = 14pt

\newenvironment{e8}{}{}

\begin{e8}
\hskip-2.5cm{
\vbox{
\setstrut
\offinterlineskip
%
			%tab0 and tabn are both \hfil
			\tabskip = 0pt plus 1fil
\halign to \hsize{
	#\hfil		\tabskip = 0.5 cm&	%Col 1 	= d
	\hfil #		\tabskip = 0.25cm&	%2		= E8
	#\hfil		\tabskip = 0.25cm&	%3		= s
	\strut#		\tabskip = 0.25cm&	%4		= dividing line
	\hfil #		\tabskip = 0.25cm&	%5		= E7+A1
	#\hfil		\tabskip = 0.5 cm&	%6		= s	
	\hfil #		\tabskip = 0.25cm&	%7		= E6+A2
	#\hfil		\tabskip = 0.5 cm&	%8		= s
	\hfil #		\tabskip = 0.25cm&	%9		= A24
	#\hfil		\tabskip = 0.5 cm&	%10		= s
	\hfil #		\tabskip = 0.25cm&	%11		= A8
	#\hfil		\tabskip = 0.5 cm&	%12		= s
	\hfil #		\tabskip = 0.25cm&	%13		= D8
	#\hfil		\tabskip = 0pt plus 1fil\cr	%14	= s
%End preamble
%Row 1 + \hrule begins here
d&  $E_8$ class&  s& \omit& $E_7+A_1$&  s&  $E_6+A_2$&  s&  $2A_4$&  
	s&  $A_8$&  s&  $D_8$&  s\cr\noalign{\vskip 0.2cm
						\hrule height 2pt
						\vskip 0.2cm}
%Row 2 begins here
0&  $E_8$&  1&&  $E_7$&  1&  $E_6$&  1&  5,5&  1&  9&  1&
	\underbar{15},1&  1\cr
1&  $E_8(a_1)$&  8&&  $E_7$/&  1&  $E_6$/&  2&  5,41&  4&  81&  8&
	\underbar{13},3&  8\cr
&&&&  $E_7(a_1)$&  7&  $E_6(a_1)$&  6&&&&&&\cr
2&  $E_8(a_2)$&  35&&  $E_7(a_1)$/&  7&  $E_6(a_1)$/&  12&  5,32&  5&
	72&  27&  \underbar{13},$1^3$&  7\cr
&&&&  $E_7(a_2)$&  27&  $D_5$&  20&  41,41&  16&&&  \underbar{11},5&  28\cr
3&  $E_8(a_3)$&  112&&  $E_7(a_2)$/&  27&  $E_6$//&  1&  $5,31^2$&  6&
	$71^2$&  28&  \underbar{11},$31^2$&  48\cr
&&&&  $E_7(a_3)$&  56&  $D_5$/&  40&  41,32&  20&  63&  48&  97&  56\cr
&&&& $E_6$&  21&  $E_6(a_3)$&  30&&&&&&\cr
4&  $E_8(a_4)$&  210&&  $E_7(a_3)$/&  56&  $E_6(a_1)$//&  6&  $5,2^21$&  5&
	621&  105&  $8^2$a&  35\cr
&&&&  $E_6(a_1)$&  120&  $E_6(a_3)$/&  60&  $41,31^2$&  24&  54&  42&  
	$951^2$&  140\cr
&&&&&&  $D_5(a_1)$&  64&  32,32&  25&&&&\cr
4&  $E_7$&  84&&  $E_6$/&  21& $A_5$& 15&&&&&  $8^2$b&  35\cr
&&&&  $D_6$&  35&&&&&&&  $\underline{11},2^21$&  20\cr
5&  $E_8(b_4)$&  560&&  $E_6(a_1)$/&  120&  $D_5$//&  20&  $41,2^21$&
	20&  531&  162&  $93^21$&  112\cr
&&&&  $E_7(a_4)$&  189&  $D_5(a_1)$/&  128&  $32,31^2$&  30&&&  $7^21^2$& 224\cr
&&&&  $D_6$/&  35&  $A_4+A_1$&  60&&&&&&\cr
&&&&&&  $A_5$/&  30&&&&&&\cr
6&  $E_8(a_5)$&  700&&  $E_7(a_4)$/&  189&  $E_6(a_3)$//&  30&  $32,2^21$&
	25&  522&  120&  7531&  252\cr
&&&&  $D_5+A_1$&  168& $A_4+A_1$/&  120&  $31^2,31^2$&  36&  $4^21$&  84&& \cr
&&&&  $A_6$&  105&  $D_4$&  24&&&&&&\cr
6&  $E_7(a_1)$&  567&&  $D_6(a_1)$&  210&  $A_4$&  81&  $5,21^3$& 
	4&  $61^3$&  56&  \underbar{11},$1^5$&  21\cr
&&&&&&&&&&&&  $932^2$&  140\cr
7&  $E_8(b_5)$&  1400&&  $D_6(a_1)$/&  210&  $D_5(a_1)$//&  64&  $41,21^3$&
	16&  $521^2$&  189&  $931^4$&  120\cr
&&&&  $D_5+A_1$/&  168&  $A_4$/&  162&  $31^2,2^21$&  30&  432&  168&
	$6^231$&  280\cr
&&&&  $E_7(a_5)$&  315&  $D_4$/&  48&&&&&  $752^2$&  448\cr
&&&&  $D_5$&  189&  $D_4(a_1)$&  80&&&&&&\cr
7&  $D_7$&  400&&  $A_6$/&  105&  $A_5$//&  15&&&&&  $74^21$&  112\cr
8&  $E_8(a_6)$&  1400&&  $E_7(a_5)$/&  315&  $A_4+A_1$//&  60&  $32,21^3$&
	20&  $431^2$&  216&  $751^4$&  280\cr
&&&&  $E_6(a_3)$&  405&  $D_4(a_1)$&  160&  $2^21,2^21$&  25&&&  
	$73^3$&  252\cr
&&&&&&&&&&&&  $6^22^2$a&  315\cr
&&&&&&&&&&&&  $5^31$&  140\cr
8&  $E_7(a_2)$&  1344&&  $D_5$/&  189&  $A_3+A_1$&  60&&&&&  $6^22^2$b&  315\cr
&&&&  $D_6(a_2)$&  280&&&&&&&  $92^21^3$&  64\cr
9&  $E_6+A_1$&  448&&  $A_5+A_1$&  70&  $D_4$//&  24&&&  $3^3$&  42&&\cr
&&&&&&  $2A_2+A_1$&  10&&&&&&\cr
9&  $D_7(a_1)$&  3240&&  $E_6(a_3)$/&  405&  $A_4$//&  81&  $31^2,21^3$&
	24&  $42^21$&  216&  $73^21^3$&  280\cr
&&&&  $A_5'$&  216&  $A_3+A_1$/&  120&&&&&  $6^21^4$&  336\cr
&&&&  $D_6(a_2)$&  280&&&&&&&  $5^23^2$& 560\cr
&&&&  $D_5(a_1)+A_1$&  378&&&&&&&&\cr
10&  $E_8(b_6)$&  2240&&  $A_4+A_2$&  210&  $D_4(a_1)$//&  80&  $2^21,21^3$&
	20&  $3^221$&  168&  $5^231^3$&  448\cr
&&&&  $D_5(a_1)+A_1$/&  378&  $2A_2+A_1$/&  20&&&&&&\cr
&&&&  $A_5+A_1$/&  70&&&&&&&&\cr
10&  $E_7(a_3)$&  2268&&  $D_5(a_1)$&  420&  $A_3$&  81&  $5,1^5$&  1&  
	$51^4$&  70&  $732^21^2$&  280\cr
11&  $E_6(a_1)+A_1$&  4096&&  $D_5(a_1)$/&  420&  $A_3$/&  162&  $41,1^5$&
	4&  $421^3$&  189&  $5^22^21^2$&  672\cr
&&&&  $A_4+A_1$&  512&  $A_2+2A_1$&  60&&&&&&\cr
11&  $A_7$&  1400&&  $A_4+A_2$/&  210&  $A_3+A_1$//&  60&&&&& $54^21^3$& 168\cr
}%	end \halign
}%	end \vbox
}% end \hskip
\end{e8}

%\newpage

\begin{e8}
\hskip-2.5cm{
\vbox{
\setstrut
\offinterlineskip
%
			%tab0 and tabn are both \hfil
			\tabskip = 0pt plus 1fil
\halign to \hsize{
	#\hfil		\tabskip = 0.5 cm&	%Col 1 	= d
	\hfil #		\tabskip = 0.25cm&	%2		= E8
	#\hfil		\tabskip = 0.25cm&	%3		= s
	\strut#		\tabskip = 0.25cm&	%4		= dividing line
	\hfil #		\tabskip = 0.25cm&	%5		= E7+A1
	#\hfil		\tabskip = 0.5 cm&	%6		= s	
	\hfil #		\tabskip = 0.25cm&	%7		= E6+A2
	#\hfil		\tabskip = 0.5 cm&	%8		= s
	\hfil #		\tabskip = 0.25cm&	%9		= A24
	#\hfil		\tabskip = 0.5 cm&	%10		= s
	\hfil #		\tabskip = 0.25cm&	%11		= A8
	#\hfil		\tabskip = 0.5 cm&	%12		= s
	\hfil #		\tabskip = 0.25cm&	%13		= D8
	#\hfil		\tabskip = 0pt plus 1fil\cr	%14	= s
%End preamble
%Row 1 + \hrule begins here
d&  $E_8$ class&  s& \omit& $E_7+A_1$&  s&  $E_6+A_2$&  s&  $2A_4$&  
	s&  $A_8$&  s&  $D_8$&  s\cr\noalign{\vskip 0.2cm
						\hrule height 2pt
						\vskip 0.2cm}
%Row 2 begins here
12&  $D_7(a_2)$&  4200&&  $A_4+A_1$/&  512&  $A_2+2A_1$/&  120&  $32,1^5$&  5&
	$3^21^3$&  120&  $53^31^2$&  448\cr
&&&&&&&&  $21^3,21^3$&  16&  $32^3$&  84&  $4^2a$&  140\cr
12&  $E_6$&  525&&  $(A_5)^{\prime\prime}$&  105&  $2A_2$&  24&&&&&  $91^7$&  35\cr
12&  $D_6$&  972&&  $D_4+A_1$&  84&&&&&&&  $4^2b$&  140\cr
&&&&&&&&&&&&  $72^41$&  56\cr
13&  $D_5+A_2$&  4536&&  $A_3+A_2+A_1$&  210&  $A_3$//&  81&
	$31^2,1^5$&  6&  $32^21^2$&  162&  $4^23^21^2$&  560\cr
&&&& $D_4+A_1$/&  84&&&&&&&  $53^22^21$&  168\cr
13&  $E_6(a_1)$&  2800&&  $(A_5)^{\prime\prime}$/&  105&  $2A_2$/&  48&&&&&  
	$731^6$&  160\cr
&&&&  $A_4$&  420&  $A_2+A_1$&  64&&&&&&\cr
14&  $E_7(a_4)$&  6075&&  $A_4$/&  420&  $A_2+A_1$/&  128&&&&&  $5^21^6$& 280\cr
&&&&  $A_3+A_2$&  378&&&&&&&  $72^21^5$&  90\cr
14&  $A_6+A_1$&  2835&&  $A_3+A_2+A_1$/&  210&  $A_2+2A_1$//&  60&  $2^21,1^5$&
	5&&&&\cr
15&  $D_6(a_1)$&  5600&&  $D_4(a_1)+A_1$&  405&  $A_2$&  30&&&  $41^5$&  56&
	$532^4$&  336\cr
&&&&  $D_4$&  105&&&&&&&&\cr
15&  $A_6$&  4200&&  $A_3+A_2$/&  378&  $2A_2$//&  24&&&&& $53^21^5$& 280\cr
16&  $E_8(a_7)$&  4480&&  $D_4(a_1) + A_1$/&  405&  $A_2$/&  60&  $21^3,1^5$&
	4&  $321^4$&  105&  $4^22^4a$&  315\cr
&&&&  $D_4(a_1)$&  315&  $A_2+A_1$//&  64&&&  $2^41$&  42&  $532^21^4$&  280\cr
&&&&&&&&&&&&  $4^231^5$&  252\cr
&&&&&&&&&&&&  $3^51$&  140\cr
16&  $D_5+A_1$&  3200&&  $D_4$/&  105&  $3A_1$&  15&&&&&  $4^22^4b$&  315\cr
&&&&  $A_3+2A_1$&  216&&&&&&&&\cr
17&  $E_7(a_5)$&  7168&&  $D_4(a_1)$/&  315&  $3A_1$/&  30&&&&&  $4^22^21^4$&
	448\cr
&&&&  $(A_3+A_1)'$&  280&&&&&&&  $3^42^2$&  280\cr
&&&&  $A_3+2A_1$/&  216&&&&&&&&\cr
18&  $E_6(a_3)+A_1$&  3150&&  $2A_2+A_1$&  70&  $A_2$//&  30&&&  
	$2^31^3$&  48&&\cr
18&  $D_6(a_2)$&  4200&&  $(A_3+A_1)'$/&  280&&&&&&&  $3^41^4$&  252\cr
&&&&&&&&&&&&  $52^41^3$&  70\cr
19&  $D_5(a_1)+A_2$&  1344&&&&&&&&&&  $3^32^21^3$&  112\cr
19&  $A_5+A_1$&  2016&&  $2A_2+A_1$/&  70&  $3A_1$//&  15&&&&&&\cr
20&  $A_4+A_3$&  420&&&&&&  $1^5,1^5$&  1&&&&\cr
20&  $D_5$&  2100&&  $(A_3+A_1)^{\prime\prime}$&  189&  $2A_1$&  20&&&&&  $71^9$&  35\cr
21&  $E_6(a_3)$&  5600&&  $(A_3+A_1)^{\prime\prime}$/&  189&  $2A_1$/&  40&&&&&  
	$531^8$& 120\cr
&&&&  $2A_2$&  168&&&&&&&&\cr
&&&&  $A_3$&  210&&&&&&&&\cr
21&  $D_4+A_2$&  4200&&  $A_2+3A_1$&  105&&&&&  $31^6$&  28& $3^22^41^2$& 224\cr
22&  $A_4+A_2+A_1$&  2835&&  $A_2+3A_1$/&  105&&&&&  $2^21^5$&  27&&\cr
22&  $D_5(a_1)+A_1$&  6075&&  $A_3$/&  210&&&&&&&  $4^21^8$&  140\cr
&&&&  $A_2+2A_1$&  189&&&&&&&&\cr
22&  $A_5$&  3200&&  $2A_2$/&  168&&&&&&&  $52^21^7$&  64\cr
23&  $A_4+A_2$&  4536&&  $A_2+2A_1$/&  189&  $2A_1$//&  20&&&&& $3^31^7$& 112\cr
24&  $A_4+2A_1$&  4200&&&&&&&&&&  $3^22^21^6$&  140\cr
25&  $D_5(a_1)$&  2800&&  $A_2+A_1$&  120&  $A_1$&  6&&&&&&\cr
}%	end \halign
}%	end \vbox
}% end \hskip
\end{e8}

%\newpage


\vbox{
\setstrut
\offinterlineskip
%
			%tab0 and tabn are both \hfil
			\tabskip = 0pt plus 1fil
\halign to \hsize{
	#\hfil		\tabskip = 0.5 cm&	%Col 1 	= d
	\hfil #		\tabskip = 0.25cm&	%2		= E8
	#\hfil		\tabskip = 0.25cm&	%3		= s
	\strut#		\tabskip = 0.25cm&	%4		= dividing line
	\hfil #		\tabskip = 0.25cm&	%5		= E7+A1
	#\hfil		\tabskip = 0.5 cm&	%6		= s	
	\hfil #		\tabskip = 0.25cm&	%7		= E6+A2
	#\hfil		\tabskip = 0.5 cm&	%8		= s
	\hfil #		\tabskip = 0.25cm&	%9		= A24
	#\hfil		\tabskip = 0.5 cm&	%10		= s
	\hfil #		\tabskip = 0.25cm&	%11		= A8
	#\hfil		\tabskip = 0.5 cm&	%12		= s
	\hfil #		\tabskip = 0.25cm&	%13		= D8
	#\hfil		\tabskip = 0pt plus 1fil\cr	%14	= s
%End preamble
%Row 1 + \hrule begins here
d&  $E_8$ class&  s& \omit& $E_7+A_1$&  s&  $E_6+A_2$&  s&  $2A_4$&  
	s&  $A_8$&  s&  $D_8$&  s\cr\noalign{\vskip 0.2cm
						\hrule height 2pt
						\vskip 0.2cm}
%Row 2 begins here
26&  $2A_3$&  840&&&&&&&&&&  $32^41^5$&  28\cr
26&  $A_4+A_1$&  4096&&  $A_2+A_1$/&  120&  $A_1$/&  12&&&&&&\cr
28&  $D_4(a_1)+A_2$&  2240&&&&  $A_1$//&  6&&&  $21^7$&  8&  $2^8a$&  35\cr
28&  $D_4+A_1$&  700&&  $4A_1$&  15&&&&&&&  $2^8b$&  35\cr
29&  $A_3+A_2+A_1$&  1400&&  $4A_1$/&  15&&&&&&&  $2^61^4$&  56\cr
30&  $A_4$&  2268&&  $A_2$&  56&&&&&&&  $51^{11}$&  21\cr
31&  $A_3+A_2$&  3240&&  $A_2$/&  56&&&&&&&  $3^21^{10}$&  48\cr
&&&&  $(3A_1)'$&  35&&&&&&&&\cr
32&  $D_4(a_1)+A_1$&  1400&&&&&&&&&&&\cr
34&  $A_3+2A_1$&  1050&&&&&&&&&&  $2^41^8$&  28\cr
36&  $2A_2+2A_1$&  175&&&&&&&&  $1^9$&  1&&\cr
36&  $D_4$&  525&&  $(3A_1)^{\prime\prime}$&  21&  1&  1&&&&&&\cr
37&  $D_4(a_1)$&  1400&&  $(3A_1)^{\prime\prime}$/&  21&  1/&  2&&&&&&\cr
&&&&  $2A_1$&  27&&&&&&&&\cr
38&  $A_3+A_1$&  1344&&  $2A_1$/&  27&&&&&&&&\cr
39&  $2A_2+A_1$&  448&&&&  1//&  1&&&&&&\cr
42&  $2A_2$&  700&&&&&&&&&&  $31^{13}$&  7\cr
43&  $A_2+3A_1$&  400&&&&&&&&&&  $2^21^{12}$&  8\cr
46&  $A_3$&  567&&  $A_1$&  7&&&&&&&&\cr
47&  $A_2+2A_1$&  560&&  $A_1$/&  7&&&&&&&&\cr
52&  $A_2+A_1$&  210&&&&&&&&&&&\cr
56&  $4A_1$&  50&&&&&&&&&&  $1^{16}$&  1\cr
63&  $A_2$&  112&&  1&  1&&&&&&&&\cr
64&  $3A_1$&  84&&  1/&  1&&&&&&&&\cr
74&  $2A_1$&  35&&&&&&&&&&&\cr
91&  $A_1$&  8&&&&&&&&&&&\cr
120&  1&  1&&&&&&&&&&&\cr
}%	end \halign
}%	end \vbox

\vfill\newpage
%\end

% END OF E8

% START OF F4
%\magnification=\magstep1

%Paper: Hales1
%Template: Table1
%Title: The Group F4

%\nologo

% \setstrut
% compute \strut as a \vrule of
%	width = 0
%	height = 70% of the current value of \baselineskip
%	depth = the remaining 30% of \baselineskip
%\newdimen\strutskip
%\redefine\strut{\vrule height 0.7\strutskip
%					 depth 0.3\strutskip
%					 width 0.2pt}%
%\newcommand\setstrut{%
%	\strutskip = \baselineskip
%	}

\baselineskip = 14pt

$$
\vbox{
\setstrut
\offinterlineskip
%
			%tab0 and tabn are both \hfil
			\tabskip = 0pt plus 1fil
\halign to \hsize{
	#\hfil		\tabskip = 0.5 cm&	%Col 1  = d
	\hfil #		\tabskip = 0.25cm&	%2		= F4
	#\hfil		\tabskip = 0.25cm&	%3		= s
	\strut#		\tabskip = 0.25cm&	%4		= dividing line
	\hfil #		\tabskip = 0.25cm&	%5		= C4
	#\hfil		\tabskip = 0.5 cm&	%6		= s
	\hfil #		\tabskip = 0.25cm&	%7		= D4
	#\hfil		\tabskip = 0.5 cm&	%8		= s
	\hfil #		\tabskip = 0.25cm&	%9		= B3+A1
	#\hfil		\tabskip = 0.5 cm&	%10		= s
	\hfil #		\tabskip = 0.25cm&	%11		= A2+A2
	#\hfil		\tabskip = 0pt plus 1fil\cr	%12	= s
%End preamble
%Row 1 + \hrule begins here
d&  $F_4$ class&  s& \omit&  $C_4$&  s&  $D_4$&  s&  $B_3+\tilde A_1$&  
	s&  $A_2+\tilde A_2$&  s\cr\noalign{\vskip 0.2cm
						\hrule height 2pt
						\vskip 0.2cm}
%Row 2 begins here
0&  $F_4$&  1&&  8&  1&  71&  1&  7&  1&  $3,\tilde 3$&  1\cr
1&  $F_4(a_1)$&  4&&  62&  4&  53&  4&  $51^2$&  3&  $21,\tilde 3$&  2\cr
&&&&&&&&  7/&  1&  $3,\widetilde{21}$& 2\cr
2&  $F_4(a_2)$&  9&&  $61^2$&  3&  $51^3$&  3&  $3^21$&  3&  
	$21,\widetilde{21}$&  4\cr
&&&&  $4^2$&  6&  $4^2$&  3+3&  $51^2$/&  3&  &\cr
3&  $C_3$&  8&&&&&& $32^2$&  3&  $1^3,\tilde 3$&  1\cr
3&  $B_3$&  8&&  $42^2$&  8&  $3^21^2$&  8&  $3^21$/&  3&  $3,\tilde 1^3$&  1\cr
4&  $F_4(a_3)$&  12&&  $421^2$&  6&&& $31^4$&  3&  $21,\tilde 1^3$& 2\cr
&&&&  $3^22$&  6&&&  $32^2$/&  3&  $1^3,\widetilde{21}$&  2\cr
5&  $C_3(a_1)$&  16&&  $3^21^2$& 8&&&  $2^21^3$&  2&&\cr
&&&&&&&&  $31^4$/&  3&&\cr
6&  $\tilde A_2+A_1$&  6&&&&&&&&  $1^3,\tilde 1^3$&  1\cr
6&  $B_2$&  9&&  $41^4$&  3&  $31^5$&  3&  $2^21^3$/&  2&&\cr
&&&&  $2^4$&  6&  $2^4$&  3+3&&&&\cr
7&  $A_2+\tilde A_1$&  4&&  $2^31^2$&  4&  $2^21^4$&  4&&&&\cr
9&  $\tilde A_2$&  8&&&&&&  $1^7$&  1&&\cr
9&  $A_2$&  8&&  $2^21^4$&  4&&&&&&\cr
10&  $A_1+\tilde A_1$&  9&&&&&&  $1^7$/&  1&&\cr
13&  $\tilde A_1$&  4&&&&&&&&&\cr
16&  $A_1$&  2&&  $1^8$& 1&&&&&&\cr
24&  1&  1&&&&&&&&&\cr
}%	end \halign
}$$%	end \vbox

%\end

% END OF F4

% START OF G2

%\magnification=\magstep1

%Paper: Hales1
%Template: Table6
%Title: The Group G2

%\nologo

% \setstrut
% compute \strut as a \vrule of
%	width = 0
%	height = 70% of the current value of \baselineskip
%	depth = the remaining 30% of \baselineskip
%\newdimen\strutskip
%\redefine\strut{\vrule height 0.7\strutskip
%					 depth 0.3\strutskip
%					 width 0.2pt}%
%\newcommand\setstrut{%
%	\strutskip = \baselineskip
%	}

\baselineskip = 14pt

$$
\vbox{
\setstrut
\offinterlineskip
%
			%tab0 and tabn are both \hfil
			\tabskip = 0pt plus 1fil
\halign to \hsize{
	#\hfil		\tabskip = 1.0 cm&	%Col 1	= d 
	\hfil #		\tabskip = 0.25cm&	%2		= G2 Class
	#\hfil		\tabskip = 0.25cm&	%3		= s
	\strut#		\tabskip = 0.25cm&  %4		= dividing line
	\hfil #		\tabskip = 0.25cm&	%5		= A2
	#\hfil		\tabskip = 1.0 cm&	%6		= s
	\hfil #		\tabskip = 0.25cm&	%7		= A2xA1
	#\hfil		\tabskip = 0pt plus 1fil\cr	%8 = s
%End preamble
%Row 1 + \hrule begins here
d&  $G_2$ class&  s& \omit& $A^2$&  s&  $A_1 + \tilde A_1$&  
			   s\cr\noalign{\vskip 0.2cm
							\hrule height 2pt
							\vskip 0.2cm}
%Row 2 begins here
0&  $G_2$&  1&&  3&  1&  $2,\tilde 2$&  1\cr
1&  $G_2(a_1)$&  2&&  21&  2&  $1^2,\tilde 2$&  1\cr
&&&&&&  $2,\widetilde{1^2}$&  1\cr
2&  $\tilde A_1$&  2&&&&  $1^2,\widetilde{1^2}$&  1\cr
3&  $A_1$&  1&&  $1^3$&  1&&\cr
6&  1&  1&&&&&\cr
}%	end \halign
}$$%	end \vbox

%\end

% END OF G2

% REFERENCES:

%MSRI Book References File
%Typist: David Mostardi
%Begun
\vfill\newpage
\Refs

\ref \key [{\bf A}]
\by D. Alvis
\paper Induce/Restrict Matrices for Exceptional Weyl Groups
\paperinfo preprint
\endref

\ref \key [{\bf AL}]
\by D. Alvis and G. Lusztig
\paper On Springer's Correspondance for Simple Groups of Type 
	$E_n (n = 6,7,8)$
\jour Math. Proc. Camb. Phil. Soc.  \vol 92
\yr 1982  \pages 65--78
\endref

\ref \key [{\bf BL}]
\by W.M. Benyon and G. Lusztig
\paper Some Numerical Results on the Characters of Exceptional Weyl Groups
\jour Math. Proc. Camb. Phil. Soc.  \vol 84
\yr 1978  \pages 417--426
\endref

\ref \key [{\bf BM1}]
\by W. Borho and R. MacPherson
\paper Repr\'esentations des Groupes de Weyl et Homologie d'Intersection
	pour les Vari\'et\'es Nilpotentes
\jour C.R. Acad. Sc. Paris  \vol 292 s\'er. I
\yr 27 avril 1981  \pages 707--710
\endref

\ref \key [{\bf BM2}]
\by W. Borho and R. MacPherson
\paper Partial Resolutions of Nilpotent Varieties
\jour Ast\'eriques  \vol 100
\endref

\ref \key [{\bf C}]
\by R. Carter
\book Finite Groups of Lie Type: Conjugacy Classes and Complex Characters
\publ John Wiley {\it \&\/} Sons
\yr 1985
\endref

\ref \key [{\bf H}]
\by R. Hotta
\paper On Springer's Representations
\jour J. Fac. Sci. Univ. Tokyo  \vol 82
\yr 1981  \pages 863--876
\endref

\ref \key [{\bf K}]
\by T. Kondo
\paper The Characters of the Weyl Group of Type $F_4$
\jour J. Fac. Sci. Univ. Tokyo  \vol 11
\yr 1965  \pages 145--153
\endref

\ref \key [{\bf L}]
\by R.P. Langlands
\book Les D\'ebuts d'une Formule des Traces Stables
\publ Publ. Math. de l'Univ. de Paris  VII, vol. 13
\yr 1983
\endref

\ref \key [{\bf LS}]
\by G. Lusztig and N. Spaltenstein
\paper Induced Unipotent Classes
\jour J. London Math. Soc. II  \vol 19
\yr 1979  \pages 41--52
\endref

\ref \key [{\bf Lu}]
\by G. Lusztig
\book Characters of Reductive Groups over a Finite Field
\publ Princeton University Press
\yr 1984
\endref

\ref \key [{\bf M1}]
\by I.G. MacDonald
\paper Some Irreducible Representations of Weyl Groups
\jour Bull. London Math. Soc.  \vol 4
\yr 1972  \pages 148--150
\endref

\ref \key [{\bf M2}]
\by I.G. MacDonald
\book Symmetric Functions and Hall Polynomials
\publ Clarendon Press
\publaddr Oxford
\yr 1979
\endref

\ref \key [{\bf Sh1}]
\by T. Shoji
\paper On the Springer Representations of the Weyl Groups of Classical
	Algebraic Groups
\jour Comm. in Alg. VII  \vol 16
\yr 1979  \pages 1713--1745
\endref

\ref \key [{\bf Sh2}]
\by T. Shoji
\paper On the Springer Representations of Chevalley Groups of Type $F_4$
\jour Comm. in Alg. VIII  \vol 5
\yr 1980  \pages 409--440
\endref

\ref \key [{\bf Sh3}]
\by T. Shoji
\paper On the Green Polynomials of Classical Groups
\jour Inv. Math.  \vol 74
\yr 1983  \pages 239--267
\endref

\ref \key [{\bf Sh4}]
\by T. Shoji
\paper On the Green polynomials of a Chevalley Groups of Type $F_4$.
\jour Comm. in Alg. X  \vol 5
\yr 1982  \pages 505-543
\endref

\ref \key [{\bf Sl1}]
\by P. Slodowy
\paper Four Lectures on Simple Groups and Singularities
\jour Comm. of the Math. Inst. Rijksuniversiteit Utrecht II
\yr 1980
\endref

\ref \key [{\bf Sl2}]
\by P. Slodowy
\book Simple Singularities and Simple Algebraic Groups
\publ Springer Lecture Notes in Math., vol 815
\yr 1980
\endref

\ref \key [{\bf Sp}]
\by N. Spaltenstein
\book Classes Unipotentes et Sous-Groupes de Borel
\publ Springer Lecture Notes in Math., vol 946
\yr 1982
\endref

\ref \key [{\bf Spr1}]
\by T.A. Springer
\paper A Construction of Representations of Weyl Groups
\jour Inv. Math.  \vol 44
\yr 1978  \pages 279--293
\endref

\ref \key [{\bf Spr2}]
\by T.A. Springer
\paper Trigonometric Sums, Green Functions of Finite Groups and 
	Representations of Weyl Groups
\jour Inv. Math.  \vol 36
\yr 1976  \pages 173--207
\endref

\ref \key [{\bf Spr3}]
\by T.A. Springer
\paper A Purity Result for Fixed Point Varieties in Flag Manifolds
\jour J. Fac. Sci. Univ. Tokyo  \vol Sec. 1A, Math. 31
\yr 1984  \pages 271--282
\endref

\ref \key [{\bf Sr}]
\by B. Srinivasan
\paper Green Polynomials of Finite Classical Groups
\jour Comm. in Alg. V  \vol 12
\yr 1977  \pages 1241--1258
\endref

\end{document}

